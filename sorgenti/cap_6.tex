\chapter{Model checking del trattamento della deficienza GAMT}\label{cap:modelchecking}
In questo capitolo vengono descritte le difficolt\`a tecniche riscontrate durante il model checking (sez. \ref{sez:limiti}), le propriet\`a CSL verificate sui modelli PRISM e l'esito del model checking per ognuna di esse (sez. \ref{sez:propepa} per le propriet\`a generate da Bio-PEPA e sez. \ref{sez:propcustom} per le propriet\`a definite manualmente).

\section{Limiti e requisiti del model checking}\label{sez:limiti}
Poich\'e la sperimentazione del trattamento per la deficienza GAMT \`e ancora in fase iniziale, le simulazioni si sono rivelate sufficienti a riprodurre gli esperimenti effettuati.
Sebbene possa essere interessante confrontare gli esiti simulativi con quelli di verifica formale, l'elevata dimensione del modello rende impraticabile tale confronto.

In termini predittivi, risulta di particolare interesse un'analisi di tipo prestazionale, che identifichi le reazioni limitanti della via metabolica, e di tipo temporale, per valutare l'efficacia della terapia.
Oltre a tali propriet\`a, pu\`o essere interessante, a scopo didattico, verificare le propriet\`a predefinite da Bio-PEPA, le quali pur essendo di tipo quantitativo, tuttavia non forniscono informazioni aggiuntive rispetto alle simulazioni.

I modelli PRISM, generati a partire dal modello Bio-PEPA, vengono convertiti in CTMC di dimensioni intrattabili per un personal computer.
Si \`e pertanto tentato di costruire le CTMC in macchine via via pi\`u ricche di memoria principale:
\begin{itemize}
	\item macchina virtuale con \SI{4}{GB} di RAM,
	\item macchina fisica con \SI{16}{GB} di RAM,
	\item server virtuale con \SI{64}{GB} di RAM,
	\item server virtuale con \SI{128}{GB} di RAM.
\end{itemize}
Utilizzando l'engine \emph{hybrid} (model checking ibrido) sul modello \ref{mod:14}, le prime tre macchine non sono riuscite a portare a termine la costruzione del modello, andando in crash per memoria insufficiente.
La quarta \`e invece riuscita a costruire un MTBDD da \SI{94}{GB} in 7325.603 secondi, costituito da 38814405 nodi (di cui 409378 raggiungibili), equivalente a una CTMC con 42156649 stati e 125726277 transizioni, ma si \`e dovuto interrompere il calcolo dopo circa quindici giorni di blocco.

Un tentativo con l'engine \emph{explicit} (model checking tradizionale) si \`e rivelato pi\`u fruttuoso, costruendo la matrice delle transizioni da \SI{71}{GB} in 699.519 secondi.
Tuttavia non \`e attualmente supportata la verifica di propriet\`a allo steady state (contenenti gli operatori $\mathbb{S}$ o $\mathbb{R}[\mathcal{S}]$) o contenenti operatori innestati (ad esempio $\mathcal{F} \mathcal{G} \phi$) per l'engine explicit.

\section{Propriet\`a generate da Bio-PEPA}\label{sez:propepa}
Relativamente al trattamento per la deficienza GAMT, sono stati generati automaticamente da Bio-PEPA:
\begin{itemize}
	\item reward transitori relativi alle reazioni ($\_sams$, $\_gamt$ e $\_uptake$), utili per contare quante volte avvengono;
	\item reward stazionari relativi alle specie ($\_SAM$, $\_SAH$, $\_GAA\_INT$, $\_GAA\_EXT$ e $\_CR$), che permettono di quantificarle;
	\item reward quadratici (ad esempio $\_SAM\_squared = \_SAM^2$), utili per il calcolo di varianze e scarti quadratici;
	\item reward basati sui contatori definiti ($\_UM\_SAM = \_SAM / 10$, $\_UM\_SAH = \_SAH / 10$, $\_UM\_GAA\_EXT = \_GAA\_EXT / 10$, $\_UM\_GAA\_INT = \_GAA\_INT / 10$ e $\_UM\_CR = \_CR / 10$).
\end{itemize}

Sono stati inoltre generati i predicati:
\begin{itemize}
	\item specie al massimo (ad esempio $\_SAM\_at\_maximum: \_SAM = MAX$), che rendono pi\`u leggibili le propriet\`a relative alla saturazione della specie;
	\item specie esaurite (ad esempio $\_SAM\_depleted: \_SAM = 0$), che rendono pi\`u leggibili le propriet\`a relative all'esaurimento.
\end{itemize}

La variabile $MAX$ \`e definita, nei modelli PRISM, come la somma di tutte le quantit\`a iniziali, per evitare di generare modelli infiniti.
Vengono generate altre tre variabili, nel file di propriet\`a, da inizializzare a valori arbitrari prima della verifica formale:
\begin{itemize}
	\item $T$ istante temporale;
	\item $i$ quantit\`a di una specie;
	\item $rew$ reward generico.
\end{itemize}
Vista la ridotta quantit\`a di informazioni aggiuntive ottenibili, rispetto alle simulazioni, dalle propriet\`a predefinite sono state inizializzate le variabili allo stesso valore per ogni propriet\`a: $T = 600$ (10 ore), $i = 10$ (\SI{1}{\mu M}), $rew$ non utilizzata.

Di seguito vengono elencate le propriet\`a, il tempo di verifica e l'esito, relativamente al solo modello \ref{mod:14}, che ha prodotto una CTMC da $42156649$ stati, $125726277$ transizioni e fattore di uniformizzazione pari a $17.4583$; il coefficiente d'errore per l'algoritmo di Fox-Glynn \`e pari a $1.25 \cdot 10^{-7}$.
I risultati vengono troncati alla quinta cifra decimale.

\begin{equation}
	\mathbb{P}_{=?} [ \mathcal{F}^{[T,T]} \_SAM=i ]
\end{equation}
``Qual \`e la probabilit\`a che ci siano $i$ molecole di S-adenosil metionina al tempo $T$?''\\
Risultato: 0\\
Tempo di elaborazione: 84309.627 secondi.

\begin{equation}
	\mathbb{P}_{=?} [ \mathcal{F}^{[T,T]} \_SAM\_at\_maximum ]
\end{equation}
``Qual \`e la probabilit\`a che l'S-adenosil metionina sia satura esattamente al tempo $T$?''\\
Risultato: 0.92757\\
Tempo di elaborazione: 57043.331 secondi.

\begin{equation}
	\mathbb{P}_{=?} [ \mathcal{F}^{ \leq T} \_SAM\_at\_maximum ]
\end{equation}
``Qual \`e la probabilit\`a che l'S-adenosil metionina sia satura entro tempo $T$?''\\
Risultato: 0.99999\\
Tempo di elaborazione: 67152.519 secondi.

\begin{equation}
	\mathbb{R}_{\{\_SAM\}=?} [ \mathcal{I}=T ]
\end{equation}
``Quanta S-adenosil metionina \`e presente al tempo $T$?''\\
Risultato: 584.92193\\
Tempo di elaborazione: 65439.354 secondi.

\begin{equation}
	\mathbb{P}_{=?} [ \mathcal{F}^{[T,T]} \_SAH=i ]
\end{equation}
``Qual \`e la probabilit\`a che ci siano $i$ molecole di S-adenosil omocisteina al tempo $T$?''\\
Risultato: 0\\
Tempo di elaborazione: 65624.93 secondi.

\begin{equation}
	\mathbb{P}_{=?} [ \mathcal{F}^{[T,T]} \_SAH\_at\_maximum ]
\end{equation}
``Qual \`e la probabilit\`a che l'S-adenosil omocisteina sia satura esattamente al tempo $T$?''\\
Risultato: 0\\
Tempo di elaborazione: 65311.243 secondi.

\begin{equation}
	\mathbb{P}_{=?} [ \mathcal{F}^{\leq T} \_SAH\_at\_maximum ]
\end{equation}
``Qual \`e la probabilit\`a che l'S-adenosil omocisteina sia satura entro tempo $T$?''\\
Risultato: 0\\
Tempo di elaborazione: 65306.103 secondi.

\begin{equation}
	\mathbb{R}_{\{\_SAH\}=?} [ \mathcal{I}=T ]
\end{equation}
``Quanta S-adenosil omocisteina \`e presente al tempo $T$?''\\
Risultato: 345.72163\\
Tempo di elaborazione: 65339.864.

\begin{equation}
	\mathbb{P}_{=?} [ \mathcal{F}^{[T,T]} \_GAA\_INT=i ]
\end{equation}
``Qual \`e la probabilit\`a che ci siano $i$ molecole di guanidinoacetato intracellulare al tempo $T$?''\\
Risultato: $2.69092 \cdot 10^{-13}$\\
Tempo di elaborazione: 65357.881 secondi.

\begin{equation}
	\mathbb{P}_{=?} [ \mathcal{F}^{[T,T]} \_GAA\_INT\_at\_maximum ]
\end{equation}
``Qual \`e la probabilit\`a che il guanidinoacetato intracellulare sia saturo esattamente al tempo $T$?''\\
Risultato: 0\\
Tempo di elaborazione: 65071.259 secondi.

\begin{equation}
	\mathbb{P}_{=?} [ \mathcal{F}^{\leq T} \_GAA\_INT\_at\_maximum ]
\end{equation}
``Qual \`e la probabilit\`a che il guanidinoacetato intracellulare sia saturo entro il tempo $T$?''\\
Risultato: 0\\
Tempo di elaborazione: 65088.718 secondi.

\begin{equation}
	\mathbb{R}_{\{\_GAA\_INT\}=?} [ \mathcal{I}=T ]
\end{equation}
``Quanto guanidinoacetato intracellulare \`e presente al tempo $T$?''\\
Risultato: 96.15829\\
Tempo di elaborazione: 65317.685 secondi.

\begin{equation}
	\mathbb{P}_{=?} [ \mathcal{F}^{[T,T]} \_GAA\_EXT=i ]
\end{equation}
``Qual \`e la probabilit\`a che ci siano $i$ molecole di guanidinoacetato plasmatico al tempo $T$?''\\
Risultato: $4.63631 \cdot 10^{-17}$\\
Tempo di elaborazione: 66352.333 secondi.

\begin{equation}
	\mathbb{P}_{=?} [ \mathcal{F}^{[T,T]} \_GAA\_EXT\_at\_maximum ]
\end{equation}
``Qual \`e la probabilit\`a che il guanidinoacetato plasmatico sia saturo esattamente al tempo $T$?''\\
Risultato: 0\\
Tempo di elaborazione: 65089.469 secondi.

\begin{equation}
	\mathbb{P}_{=?} [ \mathcal{F}^{\leq T} \_GAA\_EXT\_at\_maximum ]
\end{equation}
``Qual \`e la probabilit\`a che il guanidinoacetato plasmatico sia saturo entro tempo $T$?''\\
Risultato: 0\\
Tempo di elaborazione: 65337.322 secondi.

\begin{equation}
	\mathbb{R}_{\{\_GAA\_EXT\}=?} [ \mathcal{I}=T ]
\end{equation}
``Quanto guanidinoacetato plasmatico \`e presente al tempo $T$?''\\
Risultato: 108.12003\\
Tempo di elaborazione: 66145.619 secondi.

\begin{equation}
	\mathbb{P}_{=?} [ \mathcal{F}^{[T,T]} \_CR=i ]
\end{equation}
``Qual \`e la probabilit\`a che ci siano $i$ molecole di creatina al tempo $T$?''\\
Risultato: $4.09024 \cdot 10^{-161}$\\
Tempo di elaborazione: 65139.591 secondi.

\begin{equation}
	\mathbb{P}_{=?} [ \mathcal{F}^{[T,T]} \_CR\_at\_maximum ]
\end{equation}
``Qual \`e la probabilit\`a che la creatina sia satura esattamente al tempo $T$?''\\
Risultato: 0\\
Tempo di elaborazione: 65127.878 secondi.

\begin{equation}
	\mathbb{P}_{=?} [ \mathcal{F}^{\leq T} \_CR\_at\_maximum ]
\end{equation}
``Qual \`e la probabilit\`a che la creatina sia satura entro il tempo $T$?''\\
Risultato: 0\\
Tempo di elaborazione: 65093.444 secondi.

\begin{equation}
	\mathbb{R}_{\{\_CR\}=?} [ \mathcal{I}=T ]
\end{equation}
``Quanta creatina \`e presente al tempo $T$?''\\
Risultato: 332.72163\\
Tempo di elaborazione: 65481.934 secondi.

\begin{equation}
	\mathbb{R}_{\{\_UM\_SAM\}=?} [ \mathcal{I}=T ]
\end{equation}
``Quanta S-adenosil metionina \`e presente al tempo $T$, in $\mu M$?''\\
Risultato: 58.49219\\
Tempo di elaborazione: 65468.829 secondi.

\begin{equation}
	\mathbb{R}_{\{\_UM\_SAH\}=?} [ \mathcal{I}=T ]
\end{equation}
``Quanta S-adenosil omocisteina \`e presente al tempo $T$, in $\mu M$?''\\
Risultato: 34.57216\\
Tempo di elaborazione: 66046.642 secondi.

\begin{equation}
	\mathbb{R}_{\{\_UM\_GAA\_INT\}=?} [ \mathcal{I}=T ]
\end{equation}
``Quanto guanidinoacetato intracellulare \`e presente al tempo $T$, in $\mu M$?''\\
Risultato: 9.61583\\
Tempo di elaborazione: 65381.08 secondi.

\begin{equation}
	\mathbb{R}_{\{\_UM\_GAA\_EXT\}=?} [ \mathcal{I}=T ]
\end{equation}
``Quanto guanidinoacetato plasmatico \`e presente al tempo $T$, in $\mu M$?''\\
Risultato: 10.812000\\
Tempo di elaborazione: 65364.411 secondi.

\begin{equation}
	\mathbb{R}_{\{\_UM\_CR\}=?} [ \mathcal{I}=T ]
\end{equation}
``Quanta creatina \`e presente al tempo $T$, in $\mu M$?''\\
Risultato: 33.27216\\
Tempo di elaborazione: 65343.736 secondi.

\begin{equation}
	\mathbb{R}_{\{\_sams\}=?} [ \mathcal{C} \leq T ]
\end{equation}
``Quante volte \`e avvenuta la reazione ad opera della SAM sintasi entro il tempo $T$?''\\
Risultato: N/A\\
Tempo di elaborazione: N/A\\
Risultato ottenuto per via simulativa: 882.566.

\begin{equation}
	\mathbb{R}_{\{\_gamt\}=?} [ \mathcal{C} \leq T ]
\end{equation}
``Quante volte \`e avvenuta la reazione ad opera della guanidinoacetato metiltransferasi entro il tempo $T$?''\\
Risultato: N/A\\
Tempo di elaborazione: N/A\\
Risultato ottenuto per via simulativa: 332.971.

\begin{equation}
	\mathbb{R}_{\{\_uptake\}=?} [ \mathcal{C} \leq T ]
\end{equation}
``Quante volte il guanidinoacetato \`e entrato dentro gli eritrociti entro il tempo $T$?''\\
Risultato: N/A\\
Tempo di elaborazione: N/A\\
Risultato ottenuto per via simulativa: 392.443.

\section{Propriet\`a definite}\label{sez:propcustom}
Di seguito sono descritte propriet\`a pi\`u interessanti di quelle generate automaticamente (anche in questo caso verificate per $T = 600$ e $soglia = 10$).

Essendo l'S-adenosil metionina il reagente limitante, \`e necessario controllare e quantificare i casi in cui l'intera via metabolica vada in \emph{starvation}, bloccando la produzione di creatina:
\begin{equation}
\mathbb{P}_{=?} [\mathcal{F} \mathcal{G} \_SAM = 0]
\end{equation}
``Qual \`e la probabilit\`a che prima o poi l'S-adenosil metionina si esaurisca e rimanga a zero per sempre?''\\
Risultato: 0\\
Tempo di elaborazione: 342.842 secondi.

Anche nei casi in cui l'S-adenosil metionina non si esaurisca, \`e importante quantificare pure eventuali cali prestazionali e capacit\`a di recupero da cali temporanei:
\begin{equation}
\mathbb{P}_{=?} [\mathcal{F} \mathcal{G} \_SAM \leq soglia]
\end{equation}
``Qual \`e la probabilit\`a che prima o poi l'S-adenosil metionina scenda sotto la $soglia$ e vi rimanga per sempre?''\\
Risultato: 0\\
Tempo di elaborazione: 292.23 secondi.
\begin{equation}
\mathbb{P}_{=?} [\mathcal{G} (\_SAM = 0 \rightarrow \mathcal{F}^{\leq T} \_SAM > soglia)]
\end{equation}
``Qual \`e la probabilit\`a che, se si esaurisce l'S-adenosil metionina, questa riesca sempre a tornare oltre la $soglia$ entro un tempo $T$?''\\
Risultato: Propriet\`a non verificabile con l'engine explicit, n\'e per via simulativa\\
Tempo di elaborazione: N/A.

Vista la citotossicit\`a del guanidinoacetato, \`e interessante quantificare eventuali accumuli negli eritrociti (\`e importante tuttavia notare che il modello non rappresenta una possibile fuoriuscita del guanidinoacetato e che quindi l'esito della propriet\`a seguente potrebbe non essere significativo):
\begin{equation}
\mathbb{P}_{=?} [\mathcal{F}\mathcal{G}^{\leq T} \_GAA\_INT > soglia]
\end{equation}
``Qual \`e la probabilit\`a che il guanidinoacetato intracellulare rimanga sopra la $soglia$ almeno per un tempo $T$?''\\
Risultato: Propriet\`a non verificabile con l'engine explicit, n\'e per via simulativa\\
Tempo di elaborazione: N/A.

Per valutare l'efficacia della cura, \`e utile predire il tempo necessario a ripristinare livelli fisiologici di guanidinoacetato nel plasma:
\begin{equation}
\mathbb{P}_{=?} [\mathcal{F}^{\leq T} \_GAA\_EXT \leq soglia]
\end{equation}
``Qual \`e la probabilit\`a che il guanidinoacetato plasmatico scenda sotto la $soglia$ entro un tempo $T$?''\\
Risultato: $5.06063 \cdot 10^{-17}$\\
Tempo di elaborazione: 64474.518 secondi.
\\\\