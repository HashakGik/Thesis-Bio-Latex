\chapter{Costruzione del modello per il trattamento della deficienza GAMT}\label{cap:costruzione}
In questo capitolo viene presentata la fase di descrizione formale del caso di studio, riassumendo brevemente le caratteristiche specifiche del metabolismo (sezione \ref{sez:via}) e mostrando il modello Bio-PEPA (sezione \ref{sez:modpepa}) che lo descrive e i modelli PRISM generati a partire da quest'ultimo (sezione \ref{sez:modprism}).

\section{Via metabolica da modellare}\label{sez:via}
Riassumendo i dati raccolti nel capitolo \ref{cap:casostudio}, si parte dalla seguente via:
\begin{align}
		Arg + Gly &\xrightleftharpoons{AGAT} Orn + GAA\label{eq:agat2}\\
		Met + ATP &\xrightarrow{SAMS} SAM + P_i + PP_i\label{eq:sams2}\\
		GAA_{fuori} &\xrightleftharpoons{trasporto} GAA_{dentro}\label{eq:trasporto2}\\
		GAA_{dentro} + SAM &\xrightarrow{GAMT} Cr + SAH,\label{eq:gamt2}
\end{align}

Assumendo che la produzione di guanidinoacetato avvenga molto pi\`u lentamente della sua degradazione, si pu\`o astrarre dalla reazione \ref{eq:agat2}, considerando una quantit\`a iniziale di guanidinoacetato libera nel plasma che non viene rifornita.
Vista l'elevata efficacia dei meccanismi di regolazione, si pu\`o assumere che i metabolismi energetici e di biosintesi riescano a compensare ampiamente l'aumentato fabbisogno di metionina e ATP degli eritrociti, di conseguenza si possono assumere livelli costanti di questi ultimi nella reazione \ref{eq:sams2}.
L'equilibrio della reazione \ref{eq:trasporto2} risulta essere spostato verso i prodotti, a causa del sequestro immediato del guanidinoacetato intracellulare ad opera della reazione \ref{eq:gamt2}, di conseguenza \`e possibile semplificare la reazione rendendola unidirezionale (uptake).

Le cinetiche coinvolte nelle tre reazioni rimaste sono:
\begin{labeling}{trasporto}
	\item [SAMS] catalisi con formazione di complessi ternari ad ordine di associazione obbligata, inibita da S-adenosil metionina;
	\item [GAMT] catalisi con formazione di complessi ternari ad ordine di associazione obbligata, inibita da S-adenosil omocisteina;
	\item [trasporto] diffusione passiva attraverso la membrana, dal plasma, verso l'interno degli eritrociti (equazione \ref{eq:diffusione}).
\end{labeling}

Per limitare l'esplosione dello spazio degli stati del modello, \`e necessario eliminare tutto ci\`o che non \`e rilevante ai fini delle cinetiche (ad eccezione della creatina, di cui \`e richiesta la quantificazione), di conseguenza le reazioni sono semplificate come segue:
\begin{align}
	Met_{costante} + ATP_{costante} &\xrightarrow{SAMS} SAM\label{eq:sams3}\\
	GAA_{fuori} &\xrightarrow{uptake} GAA_{dentro}\label{eq:uptake3}\\
	GAA_{dentro} + SAM &\xrightarrow{GAMT} Cr + SAH\label{eq:gamt3}.
\end{align}

\subsection{Cinetiche di reazione}
Per modellare fedelmente il caso di studio, \`e necessario ricavare le velocit\`a a cui dovranno avvenire le azioni che modellano le reazioni precedentemente descritte.
Di seguito vengono ricavate le cinetiche utili alla modellazione del caso di studio.

\subsubsection{Catalisi enzimatica con formazione di complessi ternari}
Gli enzimi che catalizzano le reazioni \ref{eq:sams3} e \ref{eq:gamt3} sono nella forma:
\begin{equation*}
	A + B \xrightarrow{E} P + Q,
\end{equation*}
	che costituisce in realt\`a una rappresentazione astratta (che considera solo gli istanti iniziale e finale, ignorando i passaggi intermedi) della sequenza di reazioni pi\`u complessa:
\begin{align*}
	E + A &\rightleftharpoons EA\\
	EA + B &\rightleftharpoons EAB\\
	EAB &\rightarrow EPQ\\
	EPQ &\rightleftharpoons EQ + P\\
	EQ &\rightleftharpoons E + Q,
\end{align*}
	dove i due substrati $A$ e $B$ sono convertiti nei due prodotti $P$ e $Q$ ed \`e possibile un solo ordine di associazione (prima si forma il complesso binario $EA$ e poi quello ternario $EAB$, analogamente, prima si dissocia il prodotto $P$ e poi quello $Q$).
	
	Ai fini del calcolo della cinetica, risulta essere pi\`u conveniente ragionare sui singoli passi della reazione, anzich\'e sulla rappresentazione astratta. Una volta ricavata l'equazione della cinetica, si pu\`o applicare quest'ultima direttamente alla reazione astratta, per semplificare il modello.

Utilizzando metodi come quello di King-Altman~\cite{king1956schematic}, che semplificano il calcolo della cinetica nonostante l'elevato numero di reazioni coinvolte, si ottiene l'equazione:
\begin{equation}\label{eq:michaelistern}
v =  \frac{\frac{V_{max}}{km_B + [B]} \cdot [A]}{\frac{ki \cdot km_B + km_A \cdot [B]}{km_B + [B]} + [A]}.
\end{equation}
Risulta evidente che:
\begin{align*}
	V^{app}_{max} &= \frac{V_{max}}{km_B + [B]}\\
	K^{app}_M &= \frac{ki \cdot km_B + km_A \cdot [B]}{km_B + [B]},
\end{align*}
dove $k_i$ indica un'eventuale costante di inibizione, $km_A$ e $km_B$ indicano le affinit\`a dell'enzima rispetto ad $A$ e $B$, rispettivamente, e $V_{max}$ indica la velocit\`a massima dell'enzima.

Come precedentemente affermato, la funzione \ref{eq:michaelistern} definisce l'andamento della velocit\`a delle reazioni catalizzate da SAMS e GAMT.

\begin{figure}
	\center
	\begin{tikzpicture}
	\begin{axis}[xmin=0, xmax=20, ymin=0, ymax=1.1,domain=0:20,samples=100, xtick={1}, xticklabels={$K_m$}, ytick={0.5,1}, yticklabels={$\frac{V_{max}}{2}$, $V_{max}$}, xlabel={$[S]$}, ylabel={$v$}]
	\addplot[black]{x/(1+x)};
	\addplot[black, dashed]{1};
	\path[draw=black, dashed] (10, 50) -- (10, 0);
	\end{axis}
	\end{tikzpicture}
	\caption{Andamento della velocit\`a di reazione in funzione della concentrazione di substrato in un enzima di Michaelis-Menten}
	\label{fig:michaelis}
\end{figure}

\section{Modello Bio-PEPA}\label{sez:modpepa}
Il modello Bio-PEPA per il trattamento della deficienza GAMT a base di eritrociti ingegnerizzati, prodotto a partire dalle reazioni \ref{eq:sams3}, \ref{eq:uptake3} e \ref{eq:gamt3}, \`e il seguente:
\begin{align}
	sams &= \left [10 \cdot \frac{\frac{v\_sams}{km\_sams\_atp + atp} \cdot met}{\frac{ki\_sams\_sam \cdot km\_sams\_atp + km\_sams\_met \cdot atp}{km\_sams\_atp + atp} + met} \right ];\nonumber\\
	gamt &= \left [10 \cdot \frac{\frac{v\_gamt}{km\_gamt\_sam + \frac{SAM}{10}} \cdot \frac{GAA\_INT}{10}}{\frac{ki\_gamt\_sah \cdot km\_gamt\_sam + km\_gamt\_gaa\_int \cdot \frac{SAM}{10}}{km\_gamt\_sam + \frac{SAM}{10}} + \frac{GAA\_INT}{10}}\right ];\nonumber\\
	uptake &= \left [v\_uptake \cdot \frac{GAA\_EXT - GAA\_INT}{10} \right ];\nonumber\\
	UM\_SAM &= \left [\frac{SAM}{10}\right ];\nonumber\\
	UM\_SAH &= \left [\frac{SAH}{10}\right ];\nonumber\\
	UM\_GAA\_INT &= \left [\frac{GAA\_INT}{10}\right ];\nonumber\\
	UM\_GAA\_EXT &= \left [\frac{GAA\_EXT}{10}\right ];\nonumber\\
	UM\_CR &= \left [\frac{CR}{10}\right ];\label{mod:rate}
\end{align}
\begin{align}
	SAM &= sams \uparrow SAM + gamt \downarrow SAM;\nonumber\\
	SAH &= gamt \uparrow SAH;\nonumber\\
	GAA\_INT &= uptake \uparrow GAA\_INT + gamt \downarrow GAA\_INT;\nonumber\\
	GAA\_EXT &= uptake \downarrow GAA\_EXT;\nonumber\\
	CR &= gamt \uparrow CR;\label{mod:cost}
\end{align}
\begin{equation}
(SAM \underset{gamt}{\bowtie} SAH \underset{gamt}{\bowtie} GAA\_INT \underset{gamt}{\bowtie} CR)\underset{uptake}{\bowtie} GAA\_EXT.\label{mod:sincr}
\end{equation}

La porzione \ref{mod:rate} descrive rate e contatori definiti sul modello.
I rate sono associati alle tre azioni definite, che coincidono con le reazioni da modellare, ognuno \`e definito in base alla cinetica coinvolta ed \`e espresso in $10^{-7}$ $M \cdot min^{-1}$ (mentre le costanti sono definite in $\mu M$ e $\mu M \cdot min^{-1}$, motivo per cui \`e necessario includere fattori di conversione).
I contatori sono utilizzati per convertire le concentrazioni delle specie chimiche del modello in $\mu M$.

Le costanti di processo sono definite nella porzione \ref{mod:cost} e ognuna rappresenta il comportamento di ogni specie chimica, osservata in aliquote da $10^{-7}$ $M$.
Il comportamento dei processi \`e definito in termini di aumenti e diminuzioni a seguito di ogni reazione.
Il processo $SAM$, ad esempio, pu\`o aumentare a seguito della partecipazione alla reazione \ref{eq:sams3} (catalizzata da SAMS), oppure diminuire partecipando alla reazione \ref{eq:gamt3} (catalizzata da GAMT).

Il modello vero e proprio \`e costituito dalla porzione \ref{mod:sincr}, che sincronizza tutti i processi. La semantica dell'operatore di composizione parallela modella fedelmente il comportamento reale, poich\'e ogni reazione \`e costituita dalla trasformazione simultanea di tutte le specie coinvolte, ma, allo stesso tempo, due specie non coinvolte nella stessa reazione possono trasformarsi indipendentemente l'una dall'altra.
Essendo il risultato della composizione di due processi un nuovo processo, \`e possibile costruire il modello incrementalmente, sincronizzando man mano le singole specie, si pu\`o dunque immaginare la porzione \ref{mod:sincr} costruita come:
\begin{enumerate}
	\item $SAM$ e $SAH$ sincronizzati in un unico processo $reagenti\_gamt$;
	\item $GAA\_INT$ e $CR$ sincronizzati nel processo $prodotti\_gamt$;
	\item $reagenti\_gamt$ e $prodotti_gamt$ sincronizzati nel processo $reazione\_gamt$ (dove l'azione $gamt$ non richiede ulteriori sincronizzazioni, dato che tutti i processi che la utilizzano sono stati accorpati);
	\item $reazione\_gamt$ e $GAA\_EXT$ sincronizzati nel modello completo (dato che l'azione $uptake$ va sincronizzata solo tra $GAA\_INT$, che \`e incluso in $reazioni\_gamt$ e $GAA\_EXT$ e dato che l'azione $sams$ non ha bisogno di sincronizzazioni, essendo eseguita dal solo processo $SAM$).
	
\end{enumerate}

Dallo stesso modello \`e possibile analizzare diversi aspetti della terapia semplicemente modificando il valore delle quantit\`a iniziali e delle costanti. Le tabelle seguenti indicano i parametri utilizzati per ogni analisi.

\subsection{Ingresso di guanidinoacetato negli eritrociti}
In questa serie di istanze del modello si replica l'esperimento descritto nella sezione \ref{sez:uptake}, per validare la fedelt\`a del modello rispetto ai dati sperimentali.
Di queste istanze verr\`a fatto un utilizzo esclusivamente simulativo.

\begin{table}[H]
	\centering
	\begin{tabular}{| c | c | c | c |}
	\hline
	Parametro & Valore & Unit\`a & Note \\
	\hline
	$SAM$ & $0$ & \si{10^{-7} M} & \\
	\hline
	$SAH$ & $0$ & \si{10^{-7} M} & \\
	\hline
	$GAA\_INT$ & $5$ & \si{10^{-7} M} & valore sperimentale \\
	\hline
	$GAA\_EXT$ & $100$ & \si{10^{-7} M} & valore sperimentale \\
	\hline
	$CR$ & $0$ & \si{10^{-7} M} & \\
	\hline
	$met$ & $0$ & \si{\mu M} & \\
	\hline
	$atp$ & $0$ & \si{\mu M} & \\
	\hline
	$v\_sams$ & $0$ & \si{\mu M / min} & \\
	\hline
	$v\_gamt$ & $0$ & \si{\mu M / min} & \\
	\hline
	$km\_sams\_atp$ & $0$ & \si{\mu M} & \\
	\hline
	$km\_sams\_met$ & $0$ & \si{\mu M} & \\
	\hline
	$ki\_sams\_sam$ & $0$ & \si{\mu M} & \\
	\hline
	$km\_gamt\_sam$ & $0$ & \si{\mu M} & \\
	\hline
	$km\_gamt\_gaa\_int$ & $0$ & \si{\mu M} & \\
	\hline
	$ki\_gamt\_sah$ & $0$ & \si{\mu M} & \\
	\hline
	$v\_uptake$ & $0.02$ & \si{\mu M/min} & valore sperimentale \\
	\hline
\end{tabular}
\caption{Esperimento di uptake con \SI{10}{\mu M} di guanidinoacetato extracellulare}
\label{mod:1}
\end{table}

\begin{table}[H]
	\centering
	\begin{tabular}{| c | c | c | c |}
	\hline
	Parametro & Valore & Unit\`a & Note \\
		\hline
		$SAM$ & $0$ & \si{10^{-7} M} & \\
		\hline
		$SAH$ & $0$ & \si{10^{-7} M} & \\
		\hline
		$GAA\_INT$ & $12$ & \si{10^{-7} M} & valore sperimentale \\
		\hline
		$GAA\_EXT$ & $250$ & \si{10^{-7} M} & valore sperimentale \\
		\hline
		$CR$ & $0$ & \si{10^{-7} M} & \\
		\hline
		$met$ & $0$ & \si{\mu M} & \\
		\hline
		$atp$ & $0$ & \si{\mu M} & \\
		\hline
		$v\_sams$ & $0$ & \si{\mu M / min} & \\
		\hline
		$v\_gamt$ & $0$ & \si{\mu M / min} & \\
		\hline
		$km\_sams\_atp$ & $0$ & \si{\mu M} & \\
		\hline
		$km\_sams\_met$ & $0$ & \si{\mu M} & \\
		\hline
		$ki\_sams\_sam$ & $0$ & \si{\mu M} & \\
		\hline
		$km\_gamt\_sam$ & $0$ & \si{\mu M} & \\
		\hline
		$km\_gamt\_gaa\_int$ & $0$ & \si{\mu M} & \\
		\hline
		$ki\_gamt\_sah$ & $0$ & \si{\mu M} & \\
		\hline
		$v\_uptake$ & $0.02$ & \si{\mu M/min} & \\
		\hline
	\end{tabular}
	\caption{Esperimento di uptake con \SI{25}{\mu M} di guanidinoacetato extracellulare}
	\label{mod:2}
\end{table}

\begin{table}[H]
	\centering
	\begin{tabular}{| c | c | c | c |}
	\hline
	Parametro & Valore & Unit\`a & Note \\
		\hline
		$SAM$ & $0$ & \si{10^{-7} M} & \\
		\hline
		$SAH$ & $0$ & \si{10^{-7} M} & \\
		\hline
		$GAA\_INT$ & $25$ & \si{10^{-7} M} & valore sperimentale \\
		\hline
		$GAA\_EXT$ & $500$ & \si{10^{-7} M} & valore sperimentale \\
		\hline
		$CR$ & $0$ & \si{10^{-7} M} & \\
		\hline
		$met$ & $0$ & \si{\mu M} & \\
		\hline
		$atp$ & $0$ & \si{\mu M} & \\
		\hline
		$v\_sams$ & $0$ & \si{\mu M / min} & \\
		\hline
		$v\_gamt$ & $0$ & \si{\mu M / min} & \\
		\hline
		$km\_sams\_atp$ & $0$ & \si{\mu M} & \\
		\hline
		$km\_sams\_met$ & $0$ & \si{\mu M} & \\
		\hline
		$ki\_sams\_sam$ & $0$ & \si{\mu M} & \\
		\hline
		$km\_gamt\_sam$ & $0$ & \si{\mu M} & \\
		\hline
		$km\_gamt\_gaa\_int$ & $0$ & \si{\mu M} & \\
		\hline
		$ki\_gamt\_sah$ & $0$ & \si{\mu M} & \\
		\hline
		$v\_uptake$ & $0.02$ & \si{\mu M/min} & \\
		\hline
	\end{tabular}
	\caption{Esperimento di uptake con \SI{50}{\mu M} di guanidinoacetato extracellulare}
	\label{mod:3}
\end{table}

\begin{table}[H]
	\centering
	\begin{tabular}{| c | c | c | c |}
	\hline
	Parametro & Valore & Unit\`a & Note \\
		\hline
		$SAM$ & $0$ & \si{10^{-7} M} & \\
		\hline
		$SAH$ & $0$ & \si{10^{-7} M} & \\
		\hline
		$GAA\_INT$ & $52$ & \si{10^{-7} M} & valore sperimentale \\
		\hline
		$GAA\_EXT$ & $1000$ & \si{10^{-7} M} & valore sperimentale \\
		\hline
		$CR$ & $0$ & \si{10^{-7} M} & \\
		\hline
		$met$ & $0$ & \si{\mu M} & \\
		\hline
		$atp$ & $0$ & \si{\mu M} & \\
		\hline
		$v\_sams$ & $0$ & \si{\mu M / min} & \\
		\hline
		$v\_gamt$ & $0$ & \si{\mu M / min} & \\
		\hline
		$km\_sams\_atp$ & $0$ & \si{\mu M} & \\
		\hline
		$km\_sams\_met$ & $0$ & \si{\mu M} & \\
		\hline
		$ki\_sams\_sam$ & $0$ & \si{\mu M} & \\
		\hline
		$km\_gamt\_sam$ & $0$ & \si{\mu M} & \\
		\hline
		$km\_gamt\_gaa\_int$ & $0$ & \si{\mu M} & \\
		\hline
		$ki\_gamt\_sah$ & $0$ & \si{\mu M} & \\
		\hline
		$v\_uptake$ & $0.02$ & \si{\mu M/min} & \\
		\hline
	\end{tabular}
	\caption{Esperimento di uptake con \SI{100}{\mu M} di guanidinoacetato extracellulare}
	\label{mod:4}
\end{table}

\subsection{Sintesi di creatina in eritrociti incapsulati con SAM esogena}
In questa serie di istanze del modello si replica l'esperimento descritto nella sezione \ref{sez:dosaggio}, per validare la fedelt\`a del modello rispetto ai dati sperimentali.
Anche queste istanze verranno sottoposte esclusivamente ad analisi simulativa.

\begin{table}[H]
	\centering
	\begin{tabular}{| c | c | c | c |}
	\hline
	Parametro & Valore & Unit\`a & Note \\
		\hline
		$SAM$ & $100$ & \si{10^{-7} M} & valore sperimentale \\
		\hline
		$SAH$ & $13$ & \si{10^{-7} M} & \cite{oden1983s} \\
		\hline
		$GAA\_INT$ & $660$ & \si{10^{-7} M} & valore sperimentale \\
		\hline
		$GAA\_EXT$ & $1000$ & \si{10^{-7} M} & valore sperimentale \\
		\hline
		$CR$ & $26$ & \si{10^{-7} M} & valore sperimentale \\
		\hline
		$met$ & $12$ & \si{\mu M} &\cite{oden1983s} \\
		\hline
		$atp$ & $1100$ & \si{\mu M} & \cite{oden1983s} \\
		\hline
		$v\_sams$ & $0.033$ & \si{\mu M / min} & \cite{kim1974s} \\
		\hline
		$v\_gamt$ & $33$ & \si{\mu M / min} & valore sperimentale \\
		\hline
		$km\_sams\_atp$ & $80$ & \si{\mu M} & \cite{oden1983s} \\
		\hline
		$km\_sams\_met$ & $2$ & \si{\mu M} & \cite{oden1983s} \\
		\hline
		$ki\_sams\_sam$ & $2$ & \si{\mu M} & \cite{oden1983s} \\
		\hline
		$km\_gamt\_sam$ & $14.8$ & \si{\mu M} & \cite{ilas2000guanidinoacetate} \\
		\hline
		$km\_gamt\_gaa\_int$ & $78$ & \si{\mu M} & \cite{ilas2000guanidinoacetate} \\
		\hline
		$ki\_gamt\_sah$ & $0.4$ & \si{\mu M} &\cite{brendagamt} \\
		\hline
		$v\_uptake$ & $0.02$ & \si{\mu M/min} & \\
		\hline
	\end{tabular}
	\caption{Esperimento di sintesi della creatina con \SI{10}{\mu M} di S-adenosil metionina incapsulata}
	\label{mod:5}
\end{table}

\begin{table}[H]
	\centering
	\begin{tabular}{| c | c | c | c |}
	\hline
	Parametro & Valore & Unit\`a & Note \\
		\hline
		$SAM$ & $500$ & \si{10^{-7} M} & valore sperimentale \\
		\hline
		$SAH$ & $13$ & \si{10^{-7} M} & \\
		\hline
		$GAA\_INT$ & $663$ & \si{10^{-7} M} & valore sperimentale \\
		\hline
		$GAA\_EXT$ & $1000$ & \si{10^{-7} M} & \\
		\hline
		$CR$ & $73$ & \si{10^{-7} M} & valore sperimentale \\
		\hline
		$met$ & $12$ & \si{\mu M} & \\
		\hline
		$atp$ & $1100$ & \si{\mu M} & \\
		\hline
		$v\_sams$ & $0.033$ & \si{\mu M / min} & \\
		\hline
		$v\_gamt$ & $33$ & \si{\mu M / min} & \\
		\hline
		$km\_sams\_atp$ & $80$ & \si{\mu M} & \\
		\hline
		$km\_sams\_met$ & $2$ & \si{\mu M} & \\
		\hline
		$ki\_sams\_sam$ & $2$ & \si{\mu M} & \\
		\hline
		$km\_gamt\_sam$ & $14.8$ & \si{\mu M} & \\
		\hline
		$km\_gamt\_gaa\_int$ & $78$ & \si{\mu M} & \\
		\hline
		$ki\_gamt\_sah$ & $0.4$ & \si{\mu M} & \\
		\hline
		$v\_uptake$ & $0.02$ & \si{\mu M/min} & \\
		\hline
	\end{tabular}
	\caption{Esperimento di sintesi della creatina con \SI{50}{\mu M} di S-adenosil metionina incapsulata}
	\label{mod:6}
\end{table}

\begin{table}[H]
	\centering
	\begin{tabular}{| c | c | c | c |}
	\hline
	Parametro & Valore & Unit\`a & Note \\
		\hline
		$SAM$ & $35$ & \si{10^{-7} M} & \cite{oden1983s} \\
		\hline
		$SAH$ & $13$ & \si{10^{-7} M} & \\
		\hline
		$GAA\_INT$ & $598$ & \si{10^{-7} M} &  valore sperimentale \\
		\hline
		$GAA\_EXT$ & $1000$ & \si{10^{-7} M} & \\
		\hline
		$CR$ & $1$ & \si{10^{-7} M} &  valore sperimentale \\
		\hline
		$met$ & $12$ & \si{\mu M} & \\
		\hline
		$atp$ & $1100$ & \si{\mu M} & \\
		\hline
		$v\_sams$ & $0.033$ & \si{\mu M / min} & \\
		\hline
		$v\_gamt$ & $0$ & \si{\mu M / min} & non incapsulata \\
		\hline
		$km\_sams\_atp$ & $80$ & \si{\mu M} & \\
		\hline
		$km\_sams\_met$ & $2$ & \si{\mu M} & \\
		\hline
		$ki\_sams\_sam$ & $2$ & \si{\mu M} & \\
		\hline
		$km\_gamt\_sam$ & $14.8$ & \si{\mu M} & \\
		\hline
		$km\_gamt\_gaa\_int$ & $78$ & \si{\mu M} & \\
		\hline
		$ki\_gamt\_sah$ & $0.4$ & \si{\mu M} & \\
		\hline
		$v\_uptake$ & $0.02$ & \si{\mu M/min} & \\
		\hline
	\end{tabular}
	\caption{Esperimento di controllo della sintesi della creatina con S-adenosil metionina endogena}
	\label{mod:7}
\end{table}

\subsection{Predizione della sintesi di creatina in eritrociti incapsulati con SAM sintasi di \emph{E.\ coli}}
In questa serie di istanze del modello si prevede l'andamento di esperimenti non ancora effettuati, dove si valuta la quantit\`a di SAM sintasi di \emph{E.\ coli} (con affinit\`a e costanti di equilibrio diverse rispetto a quella eritrocitaria, usata come controllo) da incapsulare per fare in modo che l'S-adenosil metionina non sia il reagente limitante del metabolismo.
Queste istanze verranno utilizzate sia per l'applicazione delle tecniche simulative che di verifica.

\begin{table}[H]
	\centering
	\begin{tabular}{| c | c | c | c |}
	\hline
	Parametro & Valore & Unit\`a & Note \\
		\hline
		$SAM$ & $35$ & \si{10^{-7} M} & \\
		\hline
		$SAH$ & $13$ & \si{10^{-7} M} & \\
		\hline
		$GAA\_INT$ & $37$ & \si{10^{-7} M} & livelli patologici~\cite{kikuchi1981liquid} \\
		\hline
		$GAA\_EXT$ & $500$ & \si{10^{-7} M} & concentrazione massima ipotizzata \\
		\hline
		$CR$ & $0$ & \si{10^{-7} M} & \\
		\hline
		$met$ & $12$ & \si{\mu M} & \\
		\hline
		$atp$ & $1100$ & \si{\mu M} & \\
		\hline
		$v\_sams$ & $0.033$ & \si{\mu M / min} & \\
		\hline
		$v\_gamt$ & $15$ & \si{\mu M / min} & assumendo un incapsulamento del \\&&&30\% circa di \SI{1}{mg/ml} \\
		\hline
		$km\_sams\_atp$ & $2$ & \si{\mu M} & \\
		\hline
		$km\_sams\_met$ & $80$ & \si{\mu M} & \\
		\hline
		$ki\_sams\_sam$ & $2$ & \si{\mu M} & \\
		\hline
		$km\_gamt\_sam$ & $14.8$ & \si{\mu M} & \\
		\hline
		$km\_gamt\_gaa\_int$ & $78$ & \si{\mu M} & \\
		\hline
		$ki\_gamt\_sah$ & $0.4$ & \si{\mu M} & \\
		\hline
		$v\_uptake$ & $0.02$ & \si{\mu M/min} & \\
		\hline
	\end{tabular}
	\caption{Esperimento di controllo della sintesi della creatina con SAM sintasi endogena}
	\label{mod:8}
\end{table}

\begin{table}[H]
	\centering
	\begin{tabular}{| c | c | c | c |}
	\hline
	Parametro & Valore & Unit\`a & Note \\
		\hline
		$SAM$ & $35$ & \si{10^{-7} M} & \\
		\hline
		$SAH$ & $13$ & \si{10^{-7} M} & \\
		\hline
		$GAA\_INT$ & $37$ & \si{10^{-7} M} & \\
		\hline
		$GAA\_EXT$ & $500$ & \si{10^{-7} M} & \\
		\hline
		$CR$ & $0$ & \si{10^{-7} M} & \\
		\hline
		$met$ & $20$ & \si{\mu M} & \\
		\hline
		$atp$ & $1100$ & \si{\mu M} & \\
		\hline
		$v\_sams$ & $20$ & \si{\mu M / min} & \cite{brendasams}, \SI{0.05}{mg} incapsulati \\
		\hline
		$v\_gamt$ & $15$ & \si{\mu M / min} & \\
		\hline
		$km\_sams\_atp$ & $73$ & \si{\mu M} & \cite{brendasams} \\
		\hline
		$km\_sams\_met$ & $75$ & \si{\mu M} & \cite{brendasams} \\
		\hline
		$ki\_sams\_sam$ & $0$ & \si{\mu M} & \cite{brendasams} \\
		\hline
		$km\_gamt\_sam$ & $14.8$ & \si{\mu M} & \\
		\hline
		$km\_gamt\_gaa\_int$ & $78$ & \si{\mu M} & \\
		\hline
		$ki\_gamt\_sah$ & $0.4$ & \si{\mu M} & \\
		\hline
		$v\_uptake$ & $0.02$ & \si{\mu M/min} & \\
		\hline
	\end{tabular}
	\caption{Esperimento di sintesi della creatina con \SI{0.05}{mg} di SAM sintasi di \emph{E.\ coli} incapsulati}
	\label{mod:9}
\end{table}

\begin{table}[H]
	\centering
	\begin{tabular}{| c | c | c | c |}
	\hline
	Parametro & Valore & Unit\`a & Note \\
		\hline
		$SAM$ & $35$ & \si{10^{-7} M} & \\
		\hline
		$SAH$ & $13$ & \si{10^{-7} M} & \\
		\hline
		$GAA\_INT$ & $37$ & \si{10^{-7} M} & \\
		\hline
		$GAA\_EXT$ & $500$ & \si{10^{-7} M} & \\
		\hline
		$CR$ & $0$ & \si{10^{-7} M} & \\
		\hline
		$met$ & $20$ & \si{\mu M} & \\
		\hline
		$atp$ & $1100$ & \si{\mu M} & \\
		\hline
		$v\_sams$ & $40$ & \si{\mu M / min} & \SI{0.1}{mg} incapsulati \\
		\hline
		$v\_gamt$ & $15$ & \si{\mu M / min} & \\
		\hline
		$km\_sams\_atp$ & $73$ & \si{\mu M} & \\
		\hline
		$km\_sams\_met$ & $75$ & \si{\mu M} & \\
		\hline
		$ki\_sams\_sam$ & $0$ & \si{\mu M} & \\
		\hline
		$km\_gamt\_sam$ & $14.8$ & \si{\mu M} & \\
		\hline
		$km\_gamt\_gaa\_int$ & $78$ & \si{\mu M} & \\
		\hline
		$ki\_gamt\_sah$ & $0.4$ & \si{\mu M} & \\
		\hline
		$v\_uptake$ & $0.02$ & \si{\mu M/min} & \\
		\hline
	\end{tabular}
	\caption{Esperimento di sintesi della creatina con \SI{0.1}{mg} di SAM sintasi di \emph{E.\ coli} incapsulati}
	\label{mod:10}
\end{table}

\begin{table}[H]
	\centering
	\begin{tabular}{| c | c | c | c |}
	\hline
	Parametro & Valore & Unit\`a & Note \\
		\hline
		$SAM$ & $35$ & \si{10^{-7} M} & \\
		\hline
		$SAH$ & $13$ & \si{10^{-7} M} & \\
		\hline
		$GAA\_INT$ & $37$ & \si{10^{-7} M} & \\
		\hline
		$GAA\_EXT$ & $500$ & \si{10^{-7} M} & \\
		\hline
		$CR$ & $0$ & \si{10^{-7} M} & \\
		\hline
		$met$ & $20$ & \si{\mu M} & \\
		\hline
		$atp$ & $1100$ & \si{\mu M} & \\
		\hline
		$v\_sams$ & $200$ & \si{\mu M / min} & \SI{0.5}{mg} incapsulati \\
		\hline
		$v\_gamt$ & $15$ & \si{\mu M / min} & \\
		\hline
		$km\_sams\_atp$ & $73$ & \si{\mu M} & \\
		\hline
		$km\_sams\_met$ & $75$ & \si{\mu M} & \\
		\hline
		$ki\_sams\_sam$ & $0$ & \si{\mu M} & \\
		\hline
		$km\_gamt\_sam$ & $14.8$ & \si{\mu M} & \\
		\hline
		$km\_gamt\_gaa\_int$ & $78$ & \si{\mu M} & \\
		\hline
		$ki\_gamt\_sah$ & $0.4$ & \si{\mu M} & \\
		\hline
		$v\_uptake$ & $0.02$ & \si{\mu M/min} & \\
		\hline
	\end{tabular}
	\caption{Esperimento di sintesi della creatina con \SI{0.5}{mg} di SAM sintasi di \emph{E.\ coli} incapsulati}
	\label{mod:11}
\end{table}

\begin{table}[H]
	\centering
	\begin{tabular}{| c | c | c | c |}
	\hline
	Parametro & Valore & Unit\`a & Note \\
		\hline
		$SAM$ & $35$ & \si{10^{-7} M} & \\
		\hline
		$SAH$ & $13$ & \si{10^{-7} M} & \\
		\hline
		$GAA\_INT$ & $37$ & \si{10^{-7} M} & \\
		\hline
		$GAA\_EXT$ & $500$ & \si{10^{-7} M} & \\
		\hline
		$CR$ & $0$ & \si{10^{-7} M} & \\
		\hline
		$met$ & $20$ & \si{\mu M} & \\
		\hline
		$atp$ & $1100$ & \si{\mu M} & \\
		\hline
		$v\_sams$ & $400$ & \si{\mu M / min} & \SI{1}{mg} incapsulati \\
		\hline
		$v\_gamt$ & $15$ & \si{\mu M / min} & \\
		\hline
		$km\_sams\_atp$ & $73$ & \si{\mu M} & \\
		\hline
		$km\_sams\_met$ & $75$ & \si{\mu M} & \\
		\hline
		$ki\_sams\_sam$ & $0$ & \si{\mu M} & \\
		\hline
		$km\_gamt\_sam$ & $14.8$ & \si{\mu M} & \\
		\hline
		$km\_gamt\_gaa\_int$ & $78$ & \si{\mu M} & \\
		\hline
		$ki\_gamt\_sah$ & $0.4$ & \si{\mu M} & \\
		\hline
		$v\_uptake$ & $0.02$ & \si{\mu M/min} & \\
		\hline
	\end{tabular}
	\caption{Esperimento di sintesi della creatina con \SI{1}{mg} di SAM sintasi di \emph{E.\ coli} incapsulati}
	\label{mod:12}
\end{table}

\begin{table}[H]
	\centering
	\begin{tabular}{| c | c | c | c |}
	\hline
	Parametro & Valore & Unit\`a & Note \\
		\hline
		$SAM$ & $35$ & \si{10^{-7} M} & \\
		\hline
		$SAH$ & $13$ & \si{10^{-7} M} & \\
		\hline
		$GAA\_INT$ & $37$ & \si{10^{-7} M} & \\
		\hline
		$GAA\_EXT$ & $500$ & \si{10^{-7} M} & \\
		\hline
		$CR$ & $0$ & \si{10^{-7} M} & \\
		\hline
		$met$ & $20$ & \si{\mu M} & \\
		\hline
		$atp$ & $1100$ & \si{\mu M} & \\
		\hline
		$v\_sams$ & $1000$ & \si{\mu M / min} & \SI{2.5}{mg} incapsulati \\
		\hline
		$v\_gamt$ & $15$ & \si{\mu M / min} & \\
		\hline
		$km\_sams\_atp$ & $73$ & \si{\mu M} & \\
		\hline
		$km\_sams\_met$ & $75$ & \si{\mu M} & \\
		\hline
		$ki\_sams\_sam$ & $0$ & \si{\mu M} & \\
		\hline
		$km\_gamt\_sam$ & $14.8$ & \si{\mu M} & \\
		\hline
		$km\_gamt\_gaa\_int$ & $78$ & \si{\mu M} & \\
		\hline
		$ki\_gamt\_sah$ & $0.4$ & \si{\mu M} & \\
		\hline
		$v\_uptake$ & $0.02$ & \si{\mu M/min} & \\
		\hline
	\end{tabular}
	\caption{Esperimento di sintesi della creatina con \SI{2.5}{mg} di SAM sintasi di \emph{E.\ coli} incapsulati}
	\label{mod:13}
\end{table}

\begin{table}[H]
	\centering
	\begin{tabular}{| c | c | c | c |}
	\hline
	Parametro & Valore & Unit\`a & Note \\
		\hline
		$SAM$ & $35$ & \si{10^{-7} M} & \\
		\hline
		$SAH$ & $13$ & \si{10^{-7} M} & \\
		\hline
		$GAA\_INT$ & $37$ & \si{10^{-7} M} & \\
		\hline
		$GAA\_EXT$ & $500$ & \si{10^{-7} M} & \\
		\hline
		$CR$ & $0$ & \si{10^{-7} M} & \\
		\hline
		$met$ & $20$ & \si{\mu M} & \\
		\hline
		$atp$ & $1100$ & \si{\mu M} & \\
		\hline
		$v\_sams$ & $2000$ & \si{\mu M / min} & \SI{5}{mg} incapsulati \\
		\hline
		$v\_gamt$ & $15$ & \si{\mu M / min} & \\
		\hline
		$km\_sams\_atp$ & $73$ & \si{\mu M} & \\
		\hline
		$km\_sams\_met$ & $75$ & \si{\mu M} & \\
		\hline
		$ki\_sams\_sam$ & $0$ & \si{\mu M} & \\
		\hline
		$km\_gamt\_sam$ & $14.8$ & \si{\mu M} & \\
		\hline
		$km\_gamt\_gaa\_int$ & $78$ & \si{\mu M} & \\
		\hline
		$ki\_gamt\_sah$ & $0.4$ & \si{\mu M} & \\
		\hline
		$v\_uptake$ & $0.02$ & \si{\mu M/min} & \\
		\hline
	\end{tabular}
	\caption{Esperimento di sintesi della creatina con \SI{5}{mg} di SAM sintasi di \emph{E.\ coli} incapsulati}
	\label{mod:14}
\end{table}

\section{Modelli PRISM}\label{sez:modprism}
I modelli PRISM compilati a partire dalle istanze del modello Bio-PEPA per effettuare l'analisi sono tutti della stessa forma, cambiando soltanto le costanti definite a inizio file e le quantit\`a iniziali dei moduli.

Per semplicit\`a viene mostrato un unico modello parametrizzato, indicando in corsivo gli identificatori Bio-PEPA con cui sostituire i parametri (ad esempio, nel modello \ref{mod:1} \`e necessario sostituire $met$ con $0$, nel modello \ref{mod:5} con $12$ e cos\`i via).
Per ragioni di spazio, le cinetiche sono indicate con $sams$, $gamt$ e $uptake$ (anzich\'e con le formule gi\`a indicate nel modello Bio-PEPA), sebbene siano le stesse per ogni modello.

In questo modello parametrico vengono definiti un modulo per ogni specie (dove nel caso di un aumento viene controllato che non sia raggiunta la saturazione e nel caso di una diminuzione viene controllato che la specie non sia esaurita), e un modulo complessivo per tutti i rate.
Vengono inoltre definite tutte le strutture di reward che verranno utilizzate per l'analisi e che corrispondono al numero di reazioni effettuate, alla quantit\`a di ogni substrato, al quadrato della quantit\`a e ai contatori Bio-PEPA definiti.

\begin{lstlisting}[mathescape=true,language=prism]
ctmc
	const double _met = $met$;
	const double _atp = $atp$;
	const double _v_sams = $v\_sams$;
	const double _v_gamt = $v\_gamt$;
	const double _km_sams_atp = $km\_sams\_atp$;
	const double _km_sams_met = $km\_sams\_met$;
	const double _ki_sams_sam = $ki\_sams\_sam$;
	const double _km_gamt_sam = $km\_gamt\_sam$;
	const double _km_gamt_gaa_int = $km\_gaa\_int$;
	const double _ki_gamt_sah = $km\_gamt\_sah$;
	const double _v_uptake = $v\_uptake$;
	
	module Rates
		[_sams] $sams$ : true;
		[_gamt] $gamt$ : true;
		[_uptake] $uptake$ : true;
	endmodule
	
	const int MAX = $SAM$ + $SAH$ + $GAA\_INT$ + $GAA\_EXT$ + $CR$;
	
	module _SAM
		_SAM : [0..MAX] init $SAM$;
		[_sams] (_SAM + 1 $\leq$ MAX) $\rightarrow$ 1 : (_SAM' = _SAM + 1);
		[_gamt] (_SAM $\geq$ 1) $\rightarrow$ 1 : (_SAM' = _SAM - 1);
	endmodule
	
	module _SAH
		_SAH : [0..MAX] init $SAH$;
		[_gamt] (_SAH + 1 $\leq$ MAX) $\rightarrow$ 1 : (_SAH' = _SAH + 1);
	endmodule
	
	module _GAA_INT
	_GAA_INT : [0..MAX] init $GAA\_INT$;
		[_uptake] (_GAA_INT + 1 $\leq$ MAX) $\rightarrow$ 1 : (_GAA_INT' = _GAA_INT + 1);
		[_gamt] (_GAA_INT $\geq$ 1) $\rightarrow$ 1 : (_GAA_INT' = _GAA_INT - 1);
	endmodule
	
	module _GAA_EXT
		_GAA_EXT : [0..MAX] init $GAA\_EXT$;
		[_uptake] (_GAA_EXT $\geq$ 1) $\rightarrow$ 1 : (_GAA_EXT' = _GAA_EXT - 1);
	endmodule
	
	module _CR
		_CR : [0..MAX] init $CR$;
		[_gamt] (_CR + 1 $\leq$ MAX) $\rightarrow$ 1 : (_CR' = _CR + 1);
	endmodule
	
	
	rewards "_sams"
		[_sams] true : 1;
	endrewards
	
	rewards "_gamt"
		[_gamt] true : 1;
	endrewards
	
	rewards "_uptake"
		[_uptake] true : 1;
	endrewards
	
	rewards "_SAM"
		true : _SAM;
	endrewards
	
	rewards "_SAM_squared"
		true : _SAM * _SAM;
	endrewards
	
	rewards "_SAH"
		true : _SAH;
	endrewards
	
	rewards "_SAH_squared"
		true : _SAH * _SAH;
	endrewards
	
	rewards "_GAA_INT"
		true : _GAA_INT;
	endrewards
	
	rewards "_GAA_INT_squared"
		true : _GAA_INT * _GAA_INT;
	endrewards
	
	rewards "_GAA_EXT"
		true : _GAA_EXT;
	endrewards
	
	rewards "_GAA_EXT_squared"
		true : _GAA_EXT * _GAA_EXT;
	endrewards
	
	rewards "_CR"
		true : _CR;
	endrewards
	
	rewards "_CR_squared"
		true : _CR * _CR;
	endrewards
	
	rewards "_UM_SAM"
		true : (_SAM / 10.0);
	endrewards
	
	rewards "_UM_SAH"
		true : (_SAH / 10.0);
	endrewards
	
	rewards "_UM_GAA_INT"
		true : (_GAA_INT / 10.0);
	endrewards
	
	rewards "_UM_GAA_EXT"
		true : (_GAA_EXT / 10.0);
	endrewards
	
	rewards "_UM_CR"
		true : (_CR / 10.0);
	endrewards\end{lstlisting}