\chapter{Analisi simulativa del trattamento della deficienza GAMT}\label{cap:simulazione}
In questo capitolo viene presentato l'esito delle simulazioni effettuate sui modelli sotto forma di grafici relativi all'andamento delle varie specie in funzione del tempo.
Pi\`u in dettaglio, sono stati ripetuti gli esperimenti gi\`a effettuati \emph{in vitro}, relativamente all'uptake di guanidinoacetato (sez. \ref{sez:simup}) e al dosaggio della creatina prodotta (sez. \ref{sez:simsint}), e gli esperimenti non ancora effettuati relativi all'incapsulamento di SAMS e GAMT (sez. \ref{sez:simpred}).


\section{Simulazione dell'esperimento sull'uptake del guanidinoacetato}\label{sez:simup}
I modelli \ref{mod:1}-\ref{mod:4} sono stati utilizzati per simulare l'esperimento di uptake di guanidinoacetato (sezione \ref{sez:uptake}).
Ogni modello \`e stato simulato per 60 minuti, con un campionamento degli eventi ogni 10 minuti.
La figura \ref{fig:uptake} raggruppa le quattro simulazioni e le sovrappone ai dati sperimentali.
\begin{figure}[H]
	\center
	\resizebox{\textwidth}{!}{
		\begin{tikzpicture}[spy using outlines={rectangle, magnification=3.0}, connect spies]
		\begin{axis}[axis lines=middle, xmin=0, xmax=60, ymin=0, ymax=70,samples=1000, xtick={0,10,...,60}, xlabel={$t (min)$}, ylabel={$[GAA*] (\mu M)$}, legend style={font=\tiny},
		x label style={at={(axis description cs:0.5,-0.1)},anchor=north},
		y label style={at={(axis description cs:-0.1,.5)},rotate=90,anchor=south},
		tick label style={font=\tiny},
		label style={font=\tiny},
		legend pos=outer north east,
		legend entries={$10 \mu M$ GAA*, $25 \mu M$ GAA*, $50 \mu M$ GAA*, $100 \mu M$ GAA*, Sperimentali}]
		
		\addplot[
		ultra thin,
		scatter,forget plot,
		point meta=explicit symbolic,
		scatter/classes={
			gaa10={mark=asterisk,draw=black,mark size=2},
			gaa25={mark=pentagon, draw=black,mark size=2},
			gaa50={mark=diamond, draw=black,mark size=2},
			gaa100={mark=o,draw=black,mark size=2},
			uptakeexp={mark=+,draw=black,mark size=2}}
		]
		file{../Dati/001-004/001.txt};
		
		\addplot[
		ultra thin,
		scatter,forget plot,
		point meta=explicit symbolic,
		scatter/classes={
			gaa10={mark=asterisk,draw=black,mark size=2},
			gaa25={mark=pentagon, draw=black,mark size=2},
			gaa50={mark=diamond, draw=black,mark size=2},
			gaa100={mark=o,draw=black,mark size=2},
			uptakeexp={mark=+,draw=black,mark size=2}}
		]
		file{../Dati/001-004/002.txt};
		
		
		
		\addplot[
		ultra thin,
		scatter,forget plot,
		point meta=explicit symbolic,
		scatter/classes={
			gaa10={mark=asterisk,draw=black,mark size=2},
			gaa25={mark=pentagon, draw=black,mark size=2},
			gaa50={mark=diamond, draw=black,mark size=2},
			gaa100={mark=o,draw=black,mark size=2},
			uptakeexp={mark=+,draw=black,mark size=2}}
		]
		file{../Dati/001-004/003.txt};
		
		\addplot[
		ultra thin,
		scatter,forget plot,
		point meta=explicit symbolic,
		scatter/classes={
			gaa10={mark=asterisk,draw=black,mark size=2},
			gaa25={mark=pentagon, draw=black,mark size=2},
			gaa50={mark=diamond, draw=black,mark size=2},
			gaa100={mark=o,draw=black,mark size=2},
			uptakeexp={mark=+,draw=black,mark size=2}}
		]
		file{../Dati/001-004/004.txt};
		
		\addplot[
		ultra thin,
		scatter,
		point meta=explicit symbolic,
		only marks,
		forget plot,
		scatter/classes={
			gaa10={mark=asterisk,draw=black,mark size=2},
			gaa25={mark=pentagon, draw=black,mark size=2},
			gaa50={mark=diamond, draw=black,mark size=2},
			gaa100={mark=o,draw=black,mark size=2},
			uptakeexp={mark=+,draw=black,mark size=2}}
		]
		file{../Dati/uptake.txt};
		
		\addlegendimage{mark=asterisk}
		\addlegendimage{mark=pentagon}
		\addlegendimage{mark=diamond}
		\addlegendimage{mark=o}
		\addlegendimage{only marks, mark=+}
		
		
		
		\end{axis}
		\end{tikzpicture}
	}
	\caption{Validazione dell'esperimento di uptake di guanidinoacetato}
	\label{fig:uptake}
\end{figure}
Si pu\`o notare come la fedelt\`a delle simulazioni rispetto ai dati sperimentali sia molto alta per tutte le concentrazioni iniziali considerate, ad eccezione del caso dell'incubazione con $100 \mu M$ di guanidinoacetato.
Tale curva presenta una leggera sovrastima rispetto ai dati sperimentali da attribuire, probabilmente, a una sottospecifica del modello.
Il volume degli eritrociti \`e infatti finito ed \`e ragionevole pensare che si presentino fenomeni di saturazione non dipendenti dal meccanismo di trasporto (che essendo una diffusione passiva non pu\`o essere, per sua natura, saturabile), bens\`i dal poco spazio a disposizione all'interno degli eritrociti per contenere grandi quantit\`a di guanidinoacetato.

Questa sovrastima \`e comunque accettabile, considerando il fatto che le aliquote di eritrociti ingegnerizzati utilizzate per questa serie di esperimenti risulta molto bassa rispetto a quelle utilizzate per gli esperimenti successivi.

\section{Simulazione dell'esperimento sulla sintesi di creatina}\label{sez:simsint}
	I modelli \ref{mod:5}-\ref{mod:7} simulano l'esperimento di dosaggio della creatina prodotta da eritrociti ingegnerizzati con GAMT (sezione \ref{sez:dosaggio}).
	Ogni modello \`e stato simulato per 10 ore, con un campionamento ogni minuto.
	Nei grafici \ref{fig:syntgaa} e \ref{fig:syntcr} sono mostrati gli andamenti delle concentrazioni del guanidinoacetato intracellulare e della creatina nel tempo.
	Il numero di marker \`e stato ridotto per aumentare la leggibilit\`a dei grafici.
	
	Si pu\`o osservare dalle curve ``Unloaded'', relative al modello \ref{mod:5}, che gli eritrociti nativi non sono dotati di alcuna attivit\`a di sintesi della creatina, quindi si limitano ad accumulare guanidinoacetato al loro interno.
	
	Le altre curve presentano invece due regioni separate da un brusco cambiamento di tendenza che coincide con l'esaurimento di S-adenosil metionina incapsulata.
	Nella prima si osservano una decrescita del guanidinoacetato e una crescita della creatina prodotta, indicando l'utilizzo dell'S-adenosil metionina incapsulata.
	La seconda mostra invece un andamento costante della creatina e un accumulo di guanidinoacetato, indicando che una volta esaurita l'S-adenosil metionina, la SAMS nativa non \`e in grado di produrne di nuova a una velocit\`a sufficientemente elevata da far riprendere la reazione di sintesi della creatina. 
	
\clearpage
\begin{sidewaysfigure}
	\center
	\resizebox{\textwidth}{!}{
		\begin{tikzpicture}[spy using outlines={rectangle, magnification=3.0}, connect spies]
		\begin{axis}[axis lines=middle, xmin=0, xmax=300, ymin=0, ymax=85,samples=1000, xtick={0,60,...,300}, xlabel={$t (min)$}, ylabel={$[Gaa_{int}] (\mu M)$}, legend style={font=\tiny},
		x label style={at={(axis description cs:0.5,-0.1)},anchor=north},
		y label style={at={(axis description cs:-0.1,.5)},rotate=90,anchor=south},
		tick label style={font=\tiny},
		label style={font=\tiny},
		legend pos=outer north east, mark repeat={10}]
		
		\addplot[
		ultra thin,
		scatter,
		point meta=explicit symbolic,
		scatter/classes={
			sam10={mark=asterisk,draw=black,mark size=2},
			sam50={mark=pentagon, draw=black,mark size=2},
			unload={mark=diamond, draw=black,mark size=2}}
		]
		file{../Dati/005-007/005-gaa.txt};
		
		
		
		\addplot[
		ultra thin,
		scatter,
		point meta=explicit symbolic,
		scatter/classes={
			sam10={mark=asterisk,draw=black,mark size=2},
			sam50={mark=pentagon, draw=black,mark size=2},
			unload={mark=diamond, draw=black,mark size=2}}
		]
		file{../Dati/005-007/006-gaa.txt};
		
		\addplot[
		ultra thin,
		scatter,
		point meta=explicit symbolic,
		scatter/classes={
			sam10={mark=asterisk,draw=black,mark size=2},
			sam50={mark=pentagon, draw=black,mark size=2},
			unload={mark=diamond, draw=black,mark size=2}}
		]
		file{../Dati/005-007/007-gaa.txt};
		
		
		\legend{$10 \mu M$ SAM, $50 \mu M$ SAM, Unloaded};
		
		\coordinate (spypoint) at (axis cs:10,65);
		\coordinate (spyviewer) at (axis cs:100,20);
		\end{axis}
		\spy [height=2.5cm, width=3cm,spy connection path={
			\begin{scope}[on background layer]
			\draw (tikzspyonnode.north east) -- (tikzspyinnode.north east);
			\draw (tikzspyonnode.north west) -- (tikzspyinnode.north west);
			\draw (tikzspyonnode.south west) -- (tikzspyinnode.south west);
			\draw (tikzspyonnode.south east) -- (tikzspyinnode.south east);
			\end{scope}
		}] on (spypoint)
		in node [fill=white] at (spyviewer);
		\end{tikzpicture}
	}
	\caption{Validazione dell'esperimento di sintesi di creatina: andamento del guanidinoacetato intracellulare}
	\label{fig:syntgaa}
\end{sidewaysfigure}

\begin{sidewaysfigure}
	\center
	\resizebox{\textwidth}{!}{
		\begin{tikzpicture}
		\begin{axis}[axis lines=middle, xmin=0, xmax=120, ymin=0, ymax=60,samples=1000, xtick={0,20,...,120}, xlabel={$t (min)$}, ylabel={$[Cr] (\mu M)$}, legend style={font=\tiny},
		x label style={at={(axis description cs:0.5,-0.1)},anchor=north},
		y label style={at={(axis description cs:-0.1,.5)},rotate=90,anchor=south},
		tick label style={font=\tiny},
		label style={font=\tiny},
		legend pos=outer north east, mark repeat={3}]
		
		\addplot[
		ultra thin,
		scatter,
		point meta=explicit symbolic,
		scatter/classes={
			sam10={mark=asterisk,draw=black,mark size=2},
			sam50={mark=pentagon, draw=black,mark size=2},
			unload={mark=diamond, draw=black,mark size=2}}
		]
		file{../Dati/005-007/005-cr.txt};
		
		
		
		\addplot[
		ultra thin,
		scatter,
		point meta=explicit symbolic,
		scatter/classes={
			sam10={mark=asterisk,draw=black,mark size=2},
			sam50={mark=pentagon, draw=black,mark size=2},
			unload={mark=diamond, draw=black,mark size=2}}
		]
		file{../Dati/005-007/006-cr.txt};
		
		\addplot[
		ultra thin,
		scatter,
		point meta=explicit symbolic,
		scatter/classes={
			sam10={mark=asterisk,draw=black,mark size=2},
			sam50={mark=pentagon, draw=black,mark size=2},
			unload={mark=diamond, draw=black,mark size=2}}
		]
		file{../Dati/005-007/007-cr.txt};
		
		
		\legend{$10 \mu M$ SAM, $50 \mu M$ SAM, Unloaded};
		
		\end{axis}
		\end{tikzpicture}
	}
	\caption{Validazione dell'esperimento di sintesi di creatina: andamento della creatina}
	\label{fig:syntcr}
\end{sidewaysfigure}
\clearpage

\section{Simulazione di eritrociti ingegnerizzati con GAMT e SAMS}\label{sez:simpred}
	I modelli \ref{mod:8}-\ref{mod:14} simulano esperimenti non ancora effettuati che prevedono l'incapsulamento di una quantit\`a fissa di GAMT e di una quantit\`a variabile di SAMS clonata da \emph{E.\ coli}.
	Ogni modello \`e stato simulato per 10 ore, con un campionamento ogni 10 minuti.
	Le figure \ref{fig:008}-\ref{fig:014} mostrano l'andamento di tutte le specie per ogni simulazione, mentre le figure \ref{fig:crsynt} e \ref{fig:gaacons} confrontano l'andamento, rispettivamente, della sintesi di creatina e del consumo di guanidinoacetato tra tutte le simulazioni, ad eccezione del modello \ref{mod:14} che, andando in saturazione, presenta dati non realistici.
	
	Si pu\`o osservare che il cambiamento di tendenza sulle curve del guanidinoacetato, appena visibile in figura \ref{fig:008} (e coincidente con l'esaurimento dell'S-adenosil metionina), viene mascherato nelle altre figure dall'attivit\`a della SAMS.
	L'S-adenosil metionina prodotta viene immediatamente sequestrata dalla GAMT, quindi la sintesi di creatina e il consumo di guanidinoacetato sono in grado di proseguire, nonostante i livelli di S-adenosil metionina siano mantenuti apparentemente a zero.
	
	L'attivit\`a della GAMT risulta essere proporzionale all'attivit\`a della SAMS incapsulata in un intervallo molto ampio.
	Nelle figure \ref{fig:013} e \ref{fig:014} si osserva invece una cinetica della SAMS che supera quella della GAMT, causando un accumulo di S-adenosil metionina, fino a raggiungere la saturazione.
	
	Le figure \ref{fig:crsynt} e \ref{fig:gaacons} evidenziano l'andamento proporzionale della cinetica della GAMT rispetto alla quantit\`a di SAMS incapsulata (e quindi della sua cinetica).

\begin{figure}[H]
	\center
	\resizebox{\textwidth}{!}{
		\begin{tikzpicture}
		\begin{axis}[axis lines=middle, xmin=0, xmax=240, ymin=0, ymax=55,samples=1000, xtick={0,60,...,240}, xlabel={$t (min)$}, ylabel={$[Specie] (\mu M)$}, legend style={font=\tiny},
		x label style={at={(axis description cs:0.5,-0.1)},anchor=north},
		y label style={at={(axis description cs:-0.1,.5)},rotate=90,anchor=south},
		tick label style={font=\tiny},
		label style={font=\tiny},
		legend pos=outer north east]

		
		\addplot[
		ultra thin,
		scatter,
		mark=asterisk,mark size=2,black,scatter/use mapped color={draw=black,fill=none}
			]
			file{../Dati/008-014/008-sam.txt};
			\addplot[
			ultra thin,
			scatter,
				mark=pentagon,mark size=2, black,scatter/use mapped color=
				{draw=black,fill=red}
				]
				file{../Dati/008-014/008-sah.txt};

				
				\addplot[
				ultra thin,
				scatter,
					mark=diamond,mark size=2,black,scatter/use mapped color={draw=black,fill=none}
					]
					file{../Dati/008-014/008-gaa_int.txt};
		
				\addplot[
				ultra thin,
				scatter, mark=o,mark size=2,black,scatter/use mapped color={draw=black,fill=none}
					]
					file{../Dati/008-014/008-gaa_ext.txt};
		
				\addplot[
				ultra thin,
				scatter, mark=x,mark size=2,black,scatter/use mapped color={draw=black,fill=none}
					]
					file{../Dati/008-014/008-cr.txt};
		
		\legend{$SAM$, $SAH$, $GAA_{int}$, $GAA_{ext}$, $CR$};
		
		\end{axis}
		\end{tikzpicture}
	}
	\caption{Simulazione di controllo con SAMS nativa (modello \ref{mod:8})}
	\label{fig:008}
\end{figure}

\begin{figure}[H]
	\center
	\resizebox{\textwidth}{!}{
		\begin{tikzpicture}
		\begin{axis}[axis lines=middle, xmin=0, xmax=240, ymin=0, ymax=55,samples=1000, xtick={0,60,...,240}, xlabel={$t (min)$}, ylabel={$[Specie] (\mu M)$}, legend style={font=\tiny},
		x label style={at={(axis description cs:0.5,-0.1)},anchor=north},
		y label style={at={(axis description cs:-0.1,.5)},rotate=90,anchor=south},
		tick label style={font=\tiny},
		label style={font=\tiny},
		legend pos=outer north east]
		
		
		\addplot[
		ultra thin,
		scatter,
		mark=asterisk,mark size=2,black,scatter/use mapped color={draw=black,fill=none}
		]
		file{../Dati/008-014/009-sam.txt};
		\addplot[
		ultra thin,
		scatter,
		mark=pentagon,mark size=2, black,scatter/use mapped color={draw=black,fill=none}
		]
		file{../Dati/008-014/009-sah.txt};
		
		\addplot[
		ultra thin,
		scatter,
		mark=diamond,mark size=2,black,scatter/use mapped color={draw=black,fill=none}
		]
		file{../Dati/008-014/009-gaa_int.txt};
		
		\addplot[
		ultra thin,
		scatter, mark=o,mark size=2,black,scatter/use mapped color={draw=black,fill=none}
		]
		file{../Dati/008-014/009-gaa_ext.txt};
		
		\addplot[
		ultra thin,
		scatter, mark=x,mark size=2,black,scatter/use mapped color={draw=black,fill=none}
		]
		file{../Dati/008-014/009-cr.txt};
		
		\legend{$SAM$, $SAH$, $GAA_{int}$, $GAA_{ext}$, $CR$};
		
		\end{axis}
		\end{tikzpicture}
	}
	\caption{Simulazione dell'incapsulamento di 0.05 mg di SAMS (modello \ref{mod:9})}
	\label{fig:009}
\end{figure}

\begin{figure}[H]
	\center
	\resizebox{\textwidth}{!}{
		\begin{tikzpicture}
		\begin{axis}[axis lines=middle, xmin=0, xmax=240, ymin=0, ymax=55,samples=1000, xtick={0,60,...,240}, xlabel={$t (min)$}, ylabel={$[Specie] (\mu M)$}, legend style={font=\tiny},
		x label style={at={(axis description cs:0.5,-0.1)},anchor=north},
		y label style={at={(axis description cs:-0.1,.5)},rotate=90,anchor=south},
		tick label style={font=\tiny},
		label style={font=\tiny},
		legend pos=outer north east]
		
		
		\addplot[
		ultra thin,
		scatter,
		mark=asterisk,mark size=2,black,scatter/use mapped color={draw=black,fill=none}
		]
		file{../Dati/008-014/010-sam.txt};
		\addplot[
		ultra thin,
		scatter,
		mark=pentagon,mark size=2, black,scatter/use mapped color={draw=black,fill=none}
		]
		file{../Dati/008-014/010-sah.txt};
		
		\addplot[
		ultra thin,
		scatter,
		mark=diamond,mark size=2,black,scatter/use mapped color={draw=black,fill=none}
		]
		file{../Dati/008-014/010-gaa_int.txt};
		
		\addplot[
		ultra thin,
		scatter, mark=o,mark size=2,black,scatter/use mapped color={draw=black,fill=none}
		]
		file{../Dati/008-014/010-gaa_ext.txt};
		
		\addplot[
		ultra thin,
		scatter, mark=x,mark size=2,black,scatter/use mapped color={draw=black,fill=none}
		]
		file{../Dati/008-014/010-cr.txt};
		
		\legend{$SAM$, $SAH$, $GAA_{int}$, $GAA_{ext}$, $CR$};
		
		\end{axis}
		\end{tikzpicture}
	}
	\caption{Simulazione dell'incapsulamento di 0.1 mg di SAMS (modello \ref{mod:10})}
	\label{fig:010}
\end{figure}

\begin{figure}[H]
	\center
	\resizebox{\textwidth}{!}{
		\begin{tikzpicture}[spy using outlines={rectangle, magnification=3.0}, connect spies]
		\begin{axis}[axis lines=middle, xmin=0, xmax=600, ymin=0, ymax=55,samples=1000, xtick={0,60,...,600}, xlabel={$t (min)$}, ylabel={$[Specie] (\mu M)$}, legend style={font=\tiny},
		x label style={at={(axis description cs:0.5,-0.1)},anchor=north},
		y label style={at={(axis description cs:-0.1,.5)},rotate=90,anchor=south},
		tick label style={font=\tiny},
		label style={font=\tiny},
		legend pos=outer north east,mark repeat={3}]
		
		
		\addplot[
		ultra thin,
		scatter,
		mark=asterisk,mark size=2,black,scatter/use mapped color={draw=black,fill=none}
		]
		file{../Dati/008-014/011-sam.txt};
		\addplot[
		ultra thin,
		scatter,
		mark=pentagon,mark size=2, black,scatter/use mapped color={draw=black,fill=none}
		]
		file{../Dati/008-014/011-sah.txt};
		
		\addplot[
		ultra thin,
		scatter,
		mark=diamond,mark size=2,black,scatter/use mapped color={draw=black,fill=none}
		]
		file{../Dati/008-014/011-gaa_int.txt};
		
		\addplot[
		ultra thin,
		scatter, mark=o,mark size=2,black,scatter/use mapped color={draw=black,fill=none}
		]
		file{../Dati/008-014/011-gaa_ext.txt};
		
		\addplot[
		ultra thin,
		scatter, mark=x,mark size=2,black,scatter/use mapped color={draw=black,fill=none}
		]
		file{../Dati/008-014/011-cr.txt};
		
		\legend{$SAM$, $SAH$, $GAA_{int}$, $GAA_{ext}$, $CR$};
		
		\end{axis}
		\end{tikzpicture}
	}
	\caption{Simulazione dell'incapsulamento di 0.5 mg di SAMS (modello \ref{mod:11})}
	\label{fig:011}
\end{figure}

\begin{figure}[H]
	\center
	\resizebox{\textwidth}{!}{
		\begin{tikzpicture}[spy using outlines={rectangle, magnification=3.0}, connect spies]
		\begin{axis}[axis lines=middle, xmin=0, xmax=600, ymin=0, ymax=55,samples=1000, xtick={0,60,...,600}, xlabel={$t (min)$}, ylabel={$[Specie] (\mu M)$}, legend style={font=\tiny},
		x label style={at={(axis description cs:0.5,-0.1)},anchor=north},
		y label style={at={(axis description cs:-0.1,.5)},rotate=90,anchor=south},
		tick label style={font=\tiny},
		label style={font=\tiny},
		legend pos=outer north east,mark repeat={3}]
		
		
		\addplot[
		ultra thin,
		scatter,
		mark=asterisk,mark size=2,black,scatter/use mapped color={draw=black,fill=none}
		]
		file{../Dati/008-014/012-sam.txt};
		\addplot[
		ultra thin,
		scatter,
		mark=pentagon,mark size=2, black,scatter/use mapped color={draw=black,fill=none}
		]
		file{../Dati/008-014/012-sah.txt};
		
		\addplot[
		ultra thin,
		scatter,
		mark=diamond,mark size=2,black,scatter/use mapped color={draw=black,fill=none}
		]
		file{../Dati/008-014/012-gaa_int.txt};
		
		\addplot[
		ultra thin,
		scatter, mark=o,mark size=2,black,scatter/use mapped color={draw=black,fill=none}
		]
		file{../Dati/008-014/012-gaa_ext.txt};
		
		\addplot[
		ultra thin,
		scatter, mark=x,mark size=2,black,scatter/use mapped color={draw=black,fill=none}
		]
		file{../Dati/008-014/012-cr.txt};
		
		\legend{$SAM$, $SAH$, $GAA_{int}$, $GAA_{ext}$, $CR$};
		
		\end{axis}
		\end{tikzpicture}
	}
	\caption{Simulazione dell'incapsulamento di 1 mg di SAMS (modello \ref{mod:12})}
	\label{fig:012}
\end{figure}

\begin{figure}[H]
	\center
	\resizebox{\textwidth}{!}{
		\begin{tikzpicture}[spy using outlines={rectangle, magnification=3.0}, connect spies]
		\begin{axis}[axis lines=middle, xmin=0, xmax=600, ymin=0, ymax=65,samples=1000, xtick={0,60,...,600}, xlabel={$t (min)$}, ylabel={$[Specie] (\mu M)$}, legend style={font=\tiny},
		x label style={at={(axis description cs:0.5,-0.1)},anchor=north},
		y label style={at={(axis description cs:-0.1,.5)},rotate=90,anchor=south},
		tick label style={font=\tiny},
		label style={font=\tiny},
		legend pos=outer north east,mark repeat={3}]
		
		
		\addplot[
		ultra thin,
		scatter,
		mark=asterisk,mark size=2,black,scatter/use mapped color={draw=black,fill=none}
		]
		file{../Dati/008-014/013-sam.txt};
		\addplot[
		ultra thin,
		scatter,
		mark=pentagon,mark size=2, black,scatter/use mapped color={draw=black,fill=none}
		]
		file{../Dati/008-014/013-sah.txt};
		
		\addplot[
		ultra thin,
		scatter,
		mark=diamond,mark size=2,black,scatter/use mapped color={draw=black,fill=none}
		]
		file{../Dati/008-014/013-gaa_int.txt};
		
		\addplot[
		ultra thin,
		scatter, mark=o,mark size=2,black,scatter/use mapped color={draw=black,fill=none}
		]
		file{../Dati/008-014/013-gaa_ext.txt};
		
		\addplot[
		ultra thin,
		scatter, mark=x,mark size=2,black,scatter/use mapped color={draw=black,fill=none}
		]
		file{../Dati/008-014/013-cr.txt};
		
		\legend{$SAM$, $SAH$, $GAA_{int}$, $GAA_{ext}$, $CR$};
		
				\coordinate (spypoint) at (axis cs:240,3);
				\coordinate (spyviewer) at (axis cs:650,20);
				\end{axis}
				\spy [height=2.5cm, width=4cm,spy connection path={
					\begin{scope}[on background layer]
					\draw (tikzspyonnode.north east) -- (tikzspyinnode.north east);
					\draw (tikzspyonnode.north west) -- (tikzspyinnode.north west);
					\draw (tikzspyonnode.south west) -- (tikzspyinnode.south west);
					\draw (tikzspyonnode.south east) -- (tikzspyinnode.south east);
					\end{scope}
				}] on (spypoint)
				in node [fill=white] at (spyviewer);
				\end{tikzpicture}
	}
	\caption{Simulazione dell'incapsulamento di 2.5 mg di SAMS (modello \ref{mod:13})}
	\label{fig:013}
\end{figure}

\begin{figure}[H]
	\center
	\resizebox{\textwidth}{!}{
		\begin{tikzpicture}
		\begin{axis}[axis lines=middle, xmin=0, xmax=600, ymin=0, ymax=65,samples=1000, xtick={0,60,...,600}, xlabel={$t (min)$}, ylabel={$[Specie] (\mu M)$}, legend style={font=\tiny},
		x label style={at={(axis description cs:0.5,-0.1)},anchor=north},
		y label style={at={(axis description cs:-0.1,.5)},rotate=90,anchor=south},
		tick label style={font=\tiny},
		label style={font=\tiny},
		legend pos=outer north east,mark repeat={3}]
		
		
		\addplot[
		ultra thin,
		scatter,
		mark=asterisk,mark size=2,black,scatter/use mapped color={draw=black,fill=none}
		]
		file{../Dati/008-014/014-sam.txt};
		\addplot[
		ultra thin,
		scatter,
		mark=pentagon,mark size=2, black,scatter/use mapped color={draw=black,fill=none}
		]
		file{../Dati/008-014/014-sah.txt};
		
		\addplot[
		ultra thin,
		scatter,
		mark=diamond,mark size=2,black,scatter/use mapped color={draw=black,fill=none}
		]
		file{../Dati/008-014/014-gaa_int.txt};
		
		\addplot[
		ultra thin,
		scatter, mark=o,mark size=2,black,scatter/use mapped color={draw=black,fill=none}
		]
		file{../Dati/008-014/014-gaa_ext.txt};
		
		\addplot[
		ultra thin,
		scatter, mark=x,mark size=2,black,scatter/use mapped color={draw=black,fill=none}
		]
		file{../Dati/008-014/014-cr.txt};
		
		\legend{$SAM$, $SAH$, $GAA_{int}$, $GAA_{ext}$, $CR$};
		
		\end{axis}

		\end{tikzpicture}
	}
	\caption{Simulazione dell'incapsulamento di 5 mg di SAMS (modello \ref{mod:14})}
	\label{fig:014}
\end{figure}

\begin{sidewaysfigure}
	\center
	\resizebox{\textwidth}{!}{
		\begin{tikzpicture}[spy using outlines={rectangle, magnification=3.0}, connect spies]
		\begin{axis}[axis lines=middle, xmin=0, xmax=600, ymin=0, ymax=55,samples=1000, xtick={0,60,...,600}, xlabel={$t (min)$}, ylabel={$[Cr] (\mu M)$}, legend style={font=\tiny},
		x label style={at={(axis description cs:0.5,-0.1)},anchor=north},
		y label style={at={(axis description cs:-0.1,.5)},rotate=90,anchor=south},
		tick label style={font=\tiny},
		label style={font=\tiny},
		legend pos=outer north east, mark repeat={3}]
		
		\addplot[
		ultra thin,
		scatter,
		point meta=explicit symbolic,
		scatter/classes={
			emat={mark=asterisk,draw=black,mark size=2},
			eco005={mark=pentagon, draw=black,mark size=2},
			eco01={mark=diamond, draw=black,mark size=2},
			eco05={mark=o,draw=black,mark size=2},
			eco1={mark=x, draw=black,mark size=2},
			eco25={mark=square,draw=black,mark size=2},
			eco5={mark=triangle,draw=black,mark size=2}}
		]
		file{../Dati/008-014/008-cr.txt};
		
		\addplot[
		ultra thin,
		scatter,
		point meta=explicit symbolic,
		scatter/classes={
			emat={mark=asterisk,draw=black,mark size=2},
			eco005={mark=pentagon, draw=black,mark size=2},
			eco01={mark=diamond, draw=black,mark size=2},
			eco05={mark=o,draw=black,mark size=2},
			eco1={mark=x, draw=black,mark size=2},
			eco25={mark=square,draw=black,mark size=2},
			eco5={mark=triangle,draw=black,mark size=2}}
		]
		file{../Dati/008-014/009-cr.txt};
		
		
		
		\addplot[
		ultra thin,
		scatter,
		point meta=explicit symbolic,
		scatter/classes={
			emat={mark=asterisk,draw=black,mark size=2},
			eco005={mark=pentagon, draw=black,mark size=2},
			eco01={mark=diamond, draw=black,mark size=2},
			eco05={mark=o,draw=black,mark size=2},
			eco1={mark=x, draw=black,mark size=2},
			eco25={mark=square,draw=black,mark size=2},
			eco5={mark=triangle,draw=black,mark size=2}}
		]
		file{../Dati/008-014/010-cr.txt};
		
		\addplot[
		ultra thin,
		scatter,
		point meta=explicit symbolic,
		scatter/classes={
			emat={mark=asterisk,draw=black,mark size=2},
			eco005={mark=pentagon, draw=black,mark size=2},
			eco01={mark=diamond, draw=black,mark size=2},
			eco05={mark=o,draw=black,mark size=2},
			eco1={mark=x, draw=black,mark size=2},
			eco25={mark=square,draw=black,mark size=2},
			eco5={mark=tria del modellongle,draw=black,mark size=2}}
		]
		file{../Dati/008-014/011-cr.txt};
		
		\addplot[
		ultra thin,
		scatter,
		point meta=explicit symbolic,
		scatter/classes={
			emat={mark=asterisk,draw=black,mark size=2},
			eco005={mark=pentagon, draw=black,mark size=2},
			eco01={mark=diamond, draw=black,mark size=2},
			eco05={mark=o,draw=black,mark size=2},
			eco1={mark=x, draw=black,mark size=2},
			eco25={mark=square,draw=black,mark size=2},
			eco5={mark=triangle,draw=black,mark size=2}}
		]
		file{../Dati/008-014/012-cr.txt};
		
		\addplot[
		ultra thin,
		scatter,
		point meta=explicit symbolic,
		scatter/classes={
			emat={mark=asterisk,draw=black,mark size=2},
			eco005={mark=pentagon, draw=black,mark size=2},
			eco01={mark=diamond, draw=black,mark size=2},
			eco05={mark=o,draw=black,mark size=2},
			eco1={mark=x, draw=black,mark size=2},
			eco25={mark=square,draw=black,mark size=2},
			eco5={mark=triangle,draw=black,mark size=2}}
		]
		file{../Dati/008-014/013-cr.txt};
		
		
		\legend{H. sapiens, $0.05mg$, $0.1mg$, $0.5mg$, $1.0mg$, $2.5mg$};
		
		\coordinate (spypoint) at (axis cs:30,5);
		\coordinate (spyviewer) at (axis cs:120,50);
		\end{axis}
		\spy [height=1.5cm, width=2cm,spy connection path={
			\begin{scope}[on background layer]
			\draw (tikzspyonnode.north east) -- (tikzspyinnode.north east);
			\draw (tikzspyonnode.north west) -- (tikzspyinnode.north west);
			\draw (tikzspyonnode.south west) -- (tikzspyinnode.south west);
			\draw (tikzspyonnode.south east) -- (tikzspyinnode.south east);
			\end{scope}
		}] on (spypoint)
		in node [fill=white] at (spyviewer);
		\end{tikzpicture}
	}
	\caption{Sintesi di creatina negli eritrociti ingegnerizzati con quantit\`a variabili di SAMS di \emph{E.\ coli}}
	\label{fig:crsynt}
\end{sidewaysfigure}

\begin{sidewaysfigure}
	\center
	\resizebox{\textwidth}{!}{
		\begin{tikzpicture}
		\begin{axis}[axis lines=middle, xmin=0, xmax=600, ymin=0, ymax=27
		,samples=1000, xtick={0,60,...,600}, xlabel={$t (min)$}, ylabel={$[GAA_{int}] (\mu M)$}, legend style={font=\tiny},
		x label style={at={(axis description cs:0.5,-0.1)},anchor=north},
		y label style={at={(axis description cs:-0.1,.5)},rotate=90,anchor=south},
		tick label style={font=\tiny},
		label style={font=\tiny},
		legend pos=outer north east,mark repeat={3}]
		
		\addplot[
		ultra thin,
		scatter,
		point meta=explicit symbolic,
		scatter/classes={
			emat={mark=asterisk,draw=black,mark size=2},
			eco005={mark=pentagon, draw=black,mark size=2},
			eco01={mark=diamond, draw=black,mark size=2},
			eco05={mark=o,draw=black,mark size=2},
			eco1={mark=x, draw=black,mark size=2},
			eco25={mark=square,draw=black,mark size=2},
			eco5={mark=triangle,draw=black,mark size=2}}
		]
		file{../Dati/008-014/008-gaa_int.txt};
		
		\addplot[
		ultra thin,
		scatter,
		point meta=explicit symbolic,
		scatter/classes={
			emat={mark=asterisk,draw=black,mark size=2},
			eco005={mark=pentagon, draw=black,mark size=2},
			eco01={mark=diamond, draw=black,mark size=2},
			eco05={mark=o,draw=black,mark size=2},
			eco1={mark=x, draw=black,mark size=2},
			eco25={mark=square,draw=black,mark size=2},
			eco5={mark=triangle,draw=black,mark size=2}}
		]
		file{../Dati/008-014/009-gaa_int.txt};
		
		\addplot[
		ultra thin,
		scatter,
		point meta=explicit symbolic,
		scatter/classes={
			emat={mark=asterisk,draw=black,mark size=2},
			eco005={mark=pentagon, draw=black,mark size=2},
			eco01={mark=diamond, draw=black,mark size=2},
			eco05={mark=o,draw=black,mark size=2},
			eco1={mark=x, draw=black,mark size=2},
			eco25={mark=square,draw=black,mark size=2},
			eco5={mark=triangle,draw=black,mark size=2}}
		]
		file{../Dati/008-014/010-gaa_int.txt};
		
		\addplot[
		ultra thin,
		scatter,
		point meta=explicit symbolic,
		scatter/classes={
			emat={mark=asterisk,draw=black,mark size=2},
			eco005={mark=pentagon, draw=black,mark size=2},
			eco01={mark=diamond, draw=black,mark size=2},
			eco05={mark=o,draw=black,mark size=2},
			eco1={mark=x, draw=black,mark size=2},
			eco25={mark=square,draw=black,mark size=2},
			eco5={mark=triangle,draw=black,mark size=2}}
		]
		file{../Dati/008-014/011-gaa_int.txt};
		
		\addplot[
		ultra thin,
		scatter,
		point meta=explicit symbolic,
		scatter/classes={
			emat={mark=asterisk,draw=black,mark size=2},
			eco005={mark=pentagon, draw=black,mark size=2},
			eco01={mark=diamond, draw=black,mark size=2},
			eco05={mark=o,draw=black,mark size=2},
			eco1={mark=x, draw=black,mark size=2},
			eco25={mark=square,draw=black,mark size=2},
			eco5={mark=triangle,draw=black,mark size=2}}
		]
		file{../Dati/008-014/012-gaa_int.txt};
		
		\addplot[
		ultra thin,
		scatter,
		point meta=explicit symbolic,
		scatter/classes={
			emat={mark=asterisk,draw=black,mark size=2},
			eco005={mark=pentagon, draw=black,mark size=2},
			eco01={mark=diamond, draw=black,mark size=2},
			eco05={mark=o,draw=black,mark size=2},
			eco1={mark=x, draw=black,mark size=2},
			eco25={mark=square,draw=black,mark size=2},
			eco5={mark=triangle,draw=black,mark size=2}}
		]
		file{../Dati/008-014/013-gaa_int.txt};

		\legend{H. sapiens, $0.05mg$, $0.1mg$, $0.5mg$, $1.0mg$, $2.5mg$};
		
		\end{axis}
		\end{tikzpicture}
	}
	\caption{Consumo di guanidinoacetato negli eritrociti ingegnerizzati con quantit\`a variabili di SAMS di \emph{E.\ coli}}
	\label{fig:gaacons}
\end{sidewaysfigure}