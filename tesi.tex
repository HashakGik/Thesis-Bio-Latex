\documentclass[a4paper,twoside,11pt]{report}
%\documentclass[a4paper,singleside,11pt]{report}

\usepackage{bio_urb_thesis}
\usepackage[italian]{babel}
\usepackage[latin1]{inputenc}
\usepackage[T1]{fontenc}
\usepackage{lmodern}

\usepackage[hidelinks]{hyperref}

\usepackage{amsmath,
			algorithm2e,
		    amsfonts,
		    amstext,
		    amssymb,
		    mathrsfs,
		    mathtools,
		    latexsym,
		    graphicx,
		    epsfig,
		    color,
		    tikz,
		    verbatim,
		    pgfplots,
		    pgfplotstable,
		    float,
		    scrextend,
		    url,
		    rotating}
	    
\usepackage[nounderscore]{syntax}
\usepackage[range-phrase = --]{siunitx}

\usetikzlibrary{positioning}
\usetikzlibrary{pgfplots.groupplots,spy,backgrounds,shadows}

\usepackage[backend=bibtex,style=trad-plain, sorting=nty,firstinits=true]{biblatex}
\addbibresource{sorgenti/biblio.bib}

\usepackage[formats]{listings}
\lstdefinelanguage{prism}{
	morekeywords={rewards, endrewards, ctmc, module, endmodule, const, double, true, init},
	sensitive=false, % keywords are not case-sensitive
}

\lstset{
	numbers = left,
	framexleftmargin=10mm,
	frame=none,
	backgroundcolor=\color[RGB]{245,245,244},
	showstringspaces=true,
	tabsize=2,
	breaklines=true
}

\begin{document}
	
	\titolo{Modellazione formale applicata ad analisi \emph{in silico} di reti metaboliche}
	\candidato{Luca Salvatore Lorello}
	\relatore{Chiar.mo Prof.~Mauro Magnani}
	\correlatori{Chiar.mo Prof.~Marco Bernardo\\Dott.ssa~Noemi Bigini\\Dott.ssa~Francesca Pierig\`e}
	\annoaccademico{2017-2018}
	
	\copertinatesi 
	\dedica{A quei pochi che leggeranno effettivamente.}
	\indice
	\indicefigure
	\indicetabelle
	\iniziatesto
	
	\chapter{Introduzione}\label{cap:introduzione}
	\section{Contesto della tesi}
		Nonostante i progressi nel campo biologico della ricerca \emph{in vitro}, i costi elevati e la potenziale assenza di informazioni preliminari su un dato fenomeno rendono al giorno d'oggi l'utilizzo delle tecniche \emph{in silico} sempre pi\`u interessanti, non solo come strumenti per l'osservazione e la replicazione del fenomeno, ma soprattutto in termini predittivi.
		
		In questa tesi viene descritto un approccio \emph{formale}, basato su algebre di processi, catene di Markov e logiche temporali, applicato alla costruzione di modelli di reti metaboliche, adatti sia all'aspetto replicativo che a quello predittivo tipici di un ``esperimento'' \emph{in silico}, facendo riferimento in particolare alle tecniche di analisi basate su \emph{simulazione} e \emph{verifica formale}.
		
		I metodi formali risultano particolarmente efficaci per l'analisi di sistemi complessi, eliminando ogni sorgente di errore attraverso appositi linguaggi che permettono altres\`i di automatizzare lo studio dei sistemi stessi.
		Partendo da una rappresentazione corretta del metabolismo oggetto d'esame, che catturi aspetti quantitativi, interattivi e cinetici, si possono ottenere, tra le altre, informazioni di natura quantitativa, temporale e prestazionale che non soffrono dell'incertezza tipica degli esperimenti \emph{in vitro}.
		
		Sebbene l'approccio simulativo abbia goduto di notevole successo grazie alla sua semplicit\`a di applicazione, non risulta tanto potente quanto quello basato su verifica, mancando dell'esaustivit\`a che contraddistingue quest'ultimo.
		Viceversa, l'approccio basato su verifica risulta computazionalmente oneroso e richiede strumenti matematici pi\`u complessi.
		Queste differenze rendono auspicabile un utilizzo sinergico di entrambi gli approcci, per ottenere il maggior numero di informazioni possibile, a partire da una singola descrizione formale del metabolismo.
		
		In questa tesi viene affrontata la modellazione di patologie di origine genetica.
		Esse sono caratterizzate da disordini metabolici permanenti che si riflettono su alterazioni pi\`u o meno estese delle vie coinvolte.
		Tra le terapie proposte per contrastare tali alterazioni, l'utilizzo di bioreattori circolanti nel flusso ematico si presenta come un approccio promettente, essendo il sangue in grado di raggiungere l'intero organismo.
		Bioreattori ideali per circolare nel sangue sono gli eritrociti che, se opportunamente ingegnerizzati, sono in grado di incapsulare enzimi in forma nativa che sopperiscono alla mancanza di funzionalit\`a di quelli patologici.

		La patologia utilizzata come esempio per questa tesi \`e la deficienza dell'enzima guanidinoacetato metiltransferasi (GAMT), necessario per la catalisi dell'ultima tappa della via biosintetica della creatina.
		Oltre ai deficit, a livello cognitivo e muscolare, causati dalla carenza di creatina, l'accumulo del guanidinoacetato, precursore della creatina che nei mammiferi non \`e impiegato in nessun'altra reazione, causa danni al sistema nervoso.
		La semplice supplementazione di creatina con la dieta non \`e dunque sufficiente ad alleviare i sintomi della patologia.
		Per una terapia efficace risulta quindi necessario limitare  la produzione del guanidinoacetato o, pi\`u auspicabilmente, eliminarlo attivamente.
	
	\section{Obiettivi e contributi della tesi}
	Questa tesi si prefigge l'obiettivo di descrivere ed esemplificare le tecniche di modellazione e verifica formale pi\`u adatte alla cattura di aspetti interattivi e quantitativi di un sistema biochimico.

	L'approccio descritto \`e utilizzato per \emph{validare} esperimenti \emph{in vitro} gi\`a effettuati e per \emph{predire} l'andamento di esperimenti futuri, relativamente a un nuovo approccio farmacologico, basato su eritrociti ingegnerizzati per il trattamento della deficienza GAMT, usato come caso di studio.
	
	Pi\`u in dettaglio, viene proposto un modello espresso nel linguaggio di modellazione algebrico Bio-PEPA, che descrive le interazioni tra il guanidinoacetato dell'ambiente plasmatico di pazienti affetti da deficienza GAMT ed eritrociti modificati da un apparato, chiamato Red Cell Loader, per funzionare da bioreattori in grado di sequestrare guanidinoacetato e liberare creatina.
	I parametri del modello sono stati successivamente istanziati a valori tali da riprodurre esperimenti \emph{in vitro} gi\`a effettuati, allo scopo di validare la fedelt\`a del modello, e a valori corrispondenti ad esperimenti programmati ma non ancora effettuati, per la predizione dell'andamento.
	
	Le istanze del modello sono state automaticamente convertite in un linguaggio di modellazione di pi\`u basso livello, adeguato all'analisi tramite simulazioni ad eventi discreti e model checking utilizzando il software PRISM.
	Sebbene il modello proposto presenti assunzioni non necessariamente soddisfatte \emph{in vivo}, le simulazioni prodotte a partire da tale modello fungeranno da validazione degli esperimenti che verranno realizzati \emph{in vitro} mentre le propriet\`a verificate tramite model checking forniranno una quantificazione preliminare dell'efficacia della terapia.

	\section{Organizzazione della tesi}
	Nel capitolo \ref{cap:modellazione} sono descritte in maniera generale le caratteristiche dei metabolismi e le tecniche di modellazione e analisi pi\`u adatte a catturare tali caratteristiche. Il capitolo \ref{cap:casostudio} presenta come caso di studio la deficienza GAMT, un difetto genetico nella via di sintesi di creatina, e il principio di funzionamento del trattamento proposto a base di eritrociti ingegnerizzati tramite Red Cell Loader.
	
	Nei capitoli successivi viene mostrato il processo di sperimentazione \emph{in silico} nelle due fasi di creazione del modello (cap. \ref{cap:costruzione}) e analisi, tramite simulazioni a eventi discreti e model checking (capp. \ref{cap:simulazione} e \ref{cap:modelchecking} rispettivamente), sia per validare risultati gi\`a ottenuti che per predire comportamenti non ancora osservati \emph{in vitro}.
	
	Il capitolo \ref{cap:conclusioni} conclude il lavoro riassumendo vantaggi e svantaggi delle due tecniche di analisi e descrive potenziali sviluppi futuri, sempre in relazione al caso di studio preso in esame.
	
	Si consiglia a un lettore con formazione biologica di cominciare la lettura del capitolo \ref{cap:modellazione} a partire dalla sezione \ref{sez:silico}.
	Un lettore con formazione informatica pu\`o limitarsi a leggere le sezioni \ref{sez:caratt} e \ref{sez:biopepa}.
	\chapter{Modellazione formale}\label{cap:modellazione}
	In questo capitolo vengono illustrate le caratteristiche ``interattive'' comuni a tutti i metabolismi biochimici (sezione \ref{sez:caratt}) e gli strumenti \emph{in silico} utili per la descrizione (sezione \ref{sez:silico}), con particolare enfasi verso la modellazione formale tramite catene di Markov (sezione \ref{sez:markov}), algebre di processi (sezione \ref{sez:processalgebra}) e logiche temporali (sezione \ref{sez:logiche}), e l'analisi, tramite simulazioni (sezione \ref{sez:simul}) e model checking (sezione \ref{sez:modelcheck}) di tali caratteristiche.

	\section{Caratteristiche dei metabolismi biochimici}\label{sez:caratt}
		Una rete metabolica \`e per sua natura costituita da un insieme di reazioni che \emph{interagiscono} tra loro in termini di quantit\`a e cinetiche.
		
		In generale una singola reazione pu\`o essere espressa come:
		\begin{equation*}
			S_1 + S_2 + \dots + S_n \xrightarrow{E , c_1, \dots, c_n} P_1 + P_2 + \dots + P_n,
		\end{equation*}
		dove l'insieme $\mathbb{S} = \{S_1, S_2, \dots, S_n\}$ costituisce i \emph{substrati} (o reagenti) della reazione e l'insieme $\mathbb{P} = \{P_1, P_2, \dots, P_n\}$ ne costituisce i \emph{prodotti}; i substrati sono specie chimiche che diminuiscono di quantit\`a, mentre i prodotti aumentano, essendo i primi trasformati nei secondi.
		L'insieme, opzionale, $\mathbb{C} = \{E, c_1, \dots, c_n\}$ rappresenta i \emph{catalizzatori} (tipicamente costituiti da almeno un enzima e zero o pi\`u cofattori) della reazione, sostanze che \emph{accelerano} la cinetica di reazione, ma che non vengono modificate in quantit\`a; in termini biologici, le reazioni non catalizzate avvengono cos\`i lentamente che si pu\`o assumere che non possano avvenire affatto, in assenza di catalizzatori.
		
		Affinch\'e avvenga una reazione, secondo la teoria degli urti, \`e necessario che si verifichi una collisione tra le molecole di substrato, con sufficiente energia e corretto orientamento spaziale.
		Una reazione chimica \`e quindi un processo aleatorio, la cui velocit\`a (rate) dipende da fattori quali temperatura (energia cinetica delle molecole) e concentrazione di substrati.
		
		La catalisi altera la cinetica, aumentando la probabilit\`a delle collisioni, grazie alla divisione della reazione in sottoreazioni termodinamicamente pi\`u favorite.
		Per meccanismi di tipo allosterico (ovvero dove modifiche di forma dell'enzima si riflettono in alterazioni della cinetica), gli enzimi possono essere modulati in presenza di \emph{promotori} o \emph{inibitori}, che aumentano o diminuiscono, rispettivamente, il rate di reazione.
		
		Una o pi\`u reazioni costituiscono un \emph{metabolismo} se interagiscono tra loro.
		Di seguito sono esemplificate alcune interazioni:
		\begin{itemize}
			\item se due reazioni sono in successione, i prodotti della prima costituiscono i reagenti della seconda: globalmente la reazione pi\`u lenta pu\`o costituire un collo di bottiglia;
			\item se due reazioni utilizzano uno stesso substrato, vi competono;
			\item se una reazione produce o sottrae un modulatore per l'enzima di un'altra reazione, la cinetica di quest'ultima viene alterata;
			\item se una reazione produce o distrugge l'enzima (o suoi attivatori e inattivatori) di un'altra reazione, la cinetica di quest'ultima viene alterata;
			\item se sono presenti sistemi di feedback all'interno di una stessa reazione (e.g.\ se il prodotto \`e anche un inibitore dell'enzima), la cinetica viene alterata nel tempo.
		\end{itemize}
		
		In termini modellistici, un metabolismo costituisce un \emph{sistema reattivo} dove le singole componenti operano su un pool condiviso di risorse su cui competono con modalit\`a stocastiche e dove le risorse possono costituire esse stesse nuove componenti (e.g.\ una stessa sostanza pu\`o comportarsi da enzima in una reazione e da substrato in un'altra).
		L'elevata dimensione del pool si riflette in un elevato grado di parallelismo, che rende il numero di interazioni estremamente elevato e, conseguentemente, l'analisi del sistema estremamente complicata e tipicamente non trattabile senza l'ausilio di \emph{astrazioni} quali:
		\begin{itemize}
			\item ragionare in termini di aliquote di sostanze, anzich\'e in termini di molecole (e.g.\ esprimendo le quantit\`a in moli, si ha un numero di interazioni di $6.022 \times 10^{23}$ volte minori rispetto al sistema reale, introducendo errori di campionamento tipicamente trascurabili);
			\item eliminare componenti non reattive (ad esempio molecole che non cambiano di quantit\`a, come i cofattori, o che non alterano la cinetica di altre reazioni);
			\item eliminare componenti poco reattive (i cui effetti sono trascurabili nel contesto preso in esame);
			\item eliminare interazioni trascurabili (come inibizioni di entit\`a molto limitata);
			\item aggregare pi\`u reazioni, eliminando tutte le sostanze intermedie e calcolando una cinetica complessiva.
		\end{itemize}
		
	\subsection{Cinetica enzimatica di Michaelis-Menten}
	La maggior parte degli enzimi \`e accomunata da una cinetica bifasica descritta genericamente dalle seguenti reazioni:
	\begin{equation}
	E + S \xrightleftharpoons[k_{-1}]{k_1} ES \xrightarrow{k_2} E + P,
	\label{eq:cinetica}
	\end{equation}
	dove $E$ indica l'enzima ed $S$ il substrato da trasformare nel prodotto $P$ e dove la prima reazione causa la formazione reversibile del complesso enzima-substrato ($ES$, con costanti di equilibrio $k_1$ per la reazione diretta e $k_{-1}$ per quella inversa) mentre la seconda causa la formazione di prodotto (con costante di equilibrio $k_2$).
	Come \`e noto, la soluzione dell'equazione differenziale che descrive la cinetica del sistema, dal punto di vista della concentrazione del complesso $ES$, sar\`a:
	\begin{equation*}
		\frac{d[ES]}{dt} = k_1 [E][S] - (k_{-1} + k_2) [ES],
	\end{equation*}
	essendo la velocit\`a complessiva pari alla differenza tra la velocit\`a di sintesi  e quella di degradazione.
	
	Indicando con $e_0$ la quantit\`a complessiva di enzima nel sistema, sar\`a vero in ogni momento: $[E] = e_0 - [ES]$ e sar\`a possibile ricavare $[ES]$ allo steady state, eguagliando le velocit\`a di sintesi e degradazione:
	\begin{align*}
		k_1[E][S] &= (k_{-1} + k_2)[ES]\\
		[ES] &= \frac{k_1 e_0 [S]}{k_{-1} + k_2 + k_1[S]}.
	\end{align*}
	
	La velocit\`a di formazione del prodotto sar\`a:
	\begin{equation*}
		\frac{d[P]}{dt} = k_2 [ES] = \frac{k_1 k_2 e_0 [S]}{k_{-1} + k_2 + k_1[S]}.
	\end{equation*}
	Ponendo $V_{max} = k_2 e_0$ e $K_M = \frac{k_{-1} + k_2}{k_1}$, \`e possibile riscrivere l'equazione come:
	\begin{equation}
	\frac{d[P]}{dt} = \frac{V_{max} [S]}{K_M + [S]}.
	\label{eq:michaelis}
	\end{equation}
	
	L'equazione \ref{eq:michaelis} prende il nome di \emph{equazione di Michaelis-Menten} e l'andamento della velocit\`a di sintesi del prodotto, in funzione della concentrazione di substrato, \`e descritto dalla figura \ref{fig:michaelis}.
	
	Per molti enzimi risulta spesso necessario arricchire la cinetica \ref{eq:cinetica} per descrivere pi\`u fedelmente il comportamento: un enzima comune pu\`o presentare substrati multipli, che formano complessi con svariati meccanismi, e interazioni con sostanze in grado di modulare la reazione (inibitori e promotori se, rispettivamente, diminuiscono o aumentano la velocit\`a di reazione).
	Tuttavia, mantenendo costanti le concentrazioni di tutte le sostanze ad eccezione di quella di interesse, si osserva un andamento identico a quello descritto dall'equazione \ref{eq:michaelis}, dove le informazioni relative alle altre sostanze sono incluse nelle ``costanti'' $V_{max}$ e $K_M$, che diventano quindi funzioni di tutte le altre concentrazioni e sono dette \emph{costanti apparenti}.
	
	\subsection{Diffusione passiva}
	Il fenomeno di diffusione di una sostanza \`e interamente governato dal suo \emph{potenziale chimico} e, allo steady state, vale la \emph{prima legge di Fick}:
	\begin{equation*}
		\mathbf{J} = -D \nabla \varphi,
	\end{equation*}
	che afferma che il flusso della sostanza varia proporzionalmente all'opposto del gradiente di concentrazione ($\nabla \varphi$) di quella sostanza, con $D$ \emph{coefficiente di diffusivit\`a}.
	
	Nel caso di un ambiente interno, separato dall'esterno da una membrana (o analogamente due compartimenti separati da una membrana), il gradiente pu\`o essere approssimato dalla differenza tra le concentrazioni tra l'interno e l'esterno:
	\begin{equation*}
		J = -D ([X_{dentro}] - [X_{fuori}]),
	\end{equation*}
	ma, essendo il flusso la derivata parziale della velocit\`a rispetto alla superficie ($\mathbf{J} = \frac{\partial\mathbf{v}}{\partial\mathbf{A}}$) e ponendo costante (in forma e dimensioni) la superficie attraversata, \`e possibile riscrivere l'equazione come:
	\begin{equation}\label{eq:diffusione}
	v_{ingresso} = k \cdot ([X_{fuori}] - [X_{dentro}]),
	\end{equation}
	dove $k$ \`e una costante, ricavabile sperimentalmente, che aggrega diffusivit\`a e superficie della membrana.
	
		
	\section{Analisi \emph{in silico}}\label{sez:silico}
		Le informazioni che possono essere inferite dalla semplice osservazione di un fenomeno sono tipicamente insufficienti per permettere una comprensione adeguata del fenomeno stesso.
		Per aumentarne la comprensibilit\`a, il metodo scientifico si avvale di strumenti detti \emph{modelli}, che rappresentano nel modo pi\`u preciso possibile il fenomeno, permettendo la formulazione di ipotesi e la loro verifica sul modello (tramite processi di induzione, deduzione e abduzione), anzich\'e direttamente sul fenomeno (tramite esperimenti).
		
		Un modello ``cognitivo'' soffre tuttavia di ambiguit\`a, inconsistenze e incompletezze che possono introdurre errori, spesso significativi, sui risultati ottenuti dal modello, rispetto ai risultati sperimentali; un \emph{modello formale} ovvia all'inconveniente tramite rappresentazioni matematiche ben formate, ovvero non ambigue, consistenti e complete.
		Poich\'e la complessit\`a dei fenomeni (e conseguentemente dei modelli che li rappresentano) \`e spesso elevata, un modello formale deve offrire la possibilit\`a di una risoluzione algoritmica (garantita dai formalismi matematici sottostanti) eseguita da un calcolatore, che diminuisca drasticamente il tempo di analisi ed elimini completamente il rischio di errori (di calcolo, avendo scongiurato quelli di fedelt\`a applicando le tecniche formali) che potrebbero passare inosservati a un esecutore umano.
		
		Tipicamente un singolo formalismo non \`e in  grado di catturare tutti gli aspetti (comportamentale, prestazionale, temporale e strutturale) di un fenomeno, di conseguenza vengono frequentemente integrati pi\`u modelli che descrivono le varie sfaccettature del fenomeno.
		
		Per un fenomeno \emph{reattivo} (ossia in grado di interagire con l'ambiente esterno), un modello potr\`a descrivere propriet\`a costitutive (che vengono definite \emph{stati}, ovvero ci\`o che il fenomeno ``\`e'' in un determinato istante) o propriet\`a interattive (dette \emph{azioni}, ovvero ci\`o che il sistema ``fa'' in un determinato istante).
		Grazie a questo dualismo, \`e possibile ricondurre ogni modello reattivo a un \emph{grafo} (spesso esteso con informazioni aggiuntive), ovvero una coppia $\mathcal{G} = (\mathbb{V}, \mathcal{E})$, dove $\mathbb{V}$ \`e un insieme di vertici (che corrispondono agli stati del modello) ed $\mathcal{E} \subseteq \mathbb{V} \times \mathbb{V}$ rappresenta gli archi (che corrispondono alle azioni del modello), definiti come relazione tra vertici.
		Ogni grafo possiede una rappresentazione grafica e, in ultima analisi, \`e possibile ricondurre qualsiasi tipologia di operazione sul grafo (e quindi sul modello che l'ha generato) a una ricerca di determinati percorsi, a partire da uno o pi\`u \emph{stati iniziali}.
		
		In figura \ref{fig:diag1} \`e esemplificato un modello per il dogma centrale della biologia molecolare, dove sia i vertici che gli archi sono arricchiti con etichette (la struttura cos\`i arricchita prende il nome di diagramma a stati e transizioni) e dove \`e stato evidenziato lo stato iniziale in grigio.
		Il modello cattura la ``posizione'' dell'informazione genetica negli stati e in che modo possa essere trasferita da una componente all'altra (ma non, ad esempio, informazioni sulle cinetiche o sulla regolazione) e pu\`o essere utilizzato per derivare informazioni come: ``non \`e possibile trasferire l'informazione dalla proteina al DNA'', ``\`e possibile, ma non certo, che dal DNA si ottenga, prima o poi, una proteina'' e via dicendo.
		Relativamente alla seconda proposizione si pu\`o notare che esistono i cicli infiniti $DNA \xrightarrow{trascrizione} RNA \xrightarrow{retrotrascrizione} DNA$ e $DNA \xrightarrow{replicazione} DNA$ che non raggiungono lo stato $Proteina$, motivo per cui viene affermata la possibilit\`a, anzich\'e la certezza.
		
		
			\begin{figure}
				\center
				\begin{tikzpicture}[->, auto, node distance=5cm and 5cm, on grid, semithick, state/.style={circle, draw, black, minimum width=2cm}]
				\node[state, fill=lightgray] (A) []{DNA};
				\node[state] (B) [right= of A]{RNA};
				\node[state] (C) [right= of B]{Proteina};
				
				\path (A) edge[loop above] node[]{replicazione} (A);
				\path (A) edge[bend left=15] node[]{trascrizione} (B);
				\path (B) edge[bend left=15] node[]{retrotrascrizione} (A);
				\path (B) edge[] node[]{traduzione} (C);
				
				\end{tikzpicture}
				\caption{Modello a stati e transizioni del dogma centrale della biologia molecolare}
				\label{fig:diag1}
			\end{figure}
		
		Un modello che codifica informazioni solo sugli stati \`e detto \emph{state-based}, se invece codifica solo azioni si dice \emph{action-based}.
		Le due famiglie di modelli sono del tutto equivalenti, in quanto \`e possibile applicare le semplici conversioni:
		\begin{labeling}{$stato \rightarrow azione$}
			\item [$stato \rightarrow azione$] per ogni stato $s$, l'azione che lo precede viene definita come ``diventa vero $s$'';
			\item [$azione \rightarrow stato$] per ogni azione $a$, lo stato che la precede viene definito come ``\`e possibile eseguire $a$''.
		\end{labeling}
		
		Un'altra classificazione dei formalismi pu\`o essere costituita dal meccanismo di risoluzione delle scelte, ovvero in quale stato evolvere tra due o pi\`u possibili successori (nei modelli state-based) o quale eseguire tra due o pi\`u azioni abilitate (in quelli action-based). I principali meccanismi sono:
		\begin{labeling}{non determinismo}
			\item [non determinismo] non viene fissato alcun criterio di scelta, quindi tutti i possibili percorsi sul grafo verranno analizzati separatamente;
			\item [priorit\`a] alle scelte \`e associato un valore numerico e, per ogni scelta, vengono eliminate quelle con la priorit\`a pi\`u bassa, nel caso in cui ci siano due priorit\`a uguali, si procede in maniera non deterministica;
			\item [race policy] le scelte alternative competono secondo un criterio (temporale o probabilistico) e viene eseguita solo l'azione pi\`u ``veloce''.
		\end{labeling}
	
		Da un modello formale \`e possibile eseguire:
		\begin{labeling}{simulazioni}
			\item [simulazioni] \emph{tracce}, intese come sequenze di stati o azioni che determinano un percorso sul modello, sono generate casualmente a partire dagli stati iniziali (o stabilite dal modellista in modo da raggiungere stati di suo interesse) e si osserva il comportamento del modello man mano che le tracce vengono percorse;
			\item [verifiche] vengono definite delle propriet\`a e viene calcolato se il modello le \emph{soddisfa}, tramite una ricerca esaustiva su tutti gli stati e le azioni.
		\end{labeling}
		
		I due approcci presentano pro e contro che rendono spesso auspicabile l'utilizzo integrato di entrambi.
		Agendo le simulazioni su tracce, sono efficienti in termini di memoria, al costo di una intrinseca non esaustivit\`a e di un tempo di analisi che cresce linearmente con la lunghezza delle tracce.
		Le verifiche invece sono corrette e complete rispetto alle propriet\`a e sono in grado di generare \emph{controesempi} nel caso di propriet\`a non verificate, al costo di un'estrema avidit\`a in memoria (salvo utilizzare tecniche simboliche)~\cite{simvsver}.
		
		Poich\'e la verifica formale risulta memory-intensive, sono spesso necessarie tecniche di modellazione che riducano la dimensione del modello, inficiando il meno possibile sulla fedelt\`a di quest'ultimo; le pi\`u comuni sono:
		\begin{labeling}{composizione}
			\item [astrazione] un grande spazio degli stati \`e ridotto definendo classi di equivalenza che eliminano informazioni sulle differenze di scarso interesse tra stati;
			\item [composizione] un modello complesso \`e rappresentato (ed eventualmente semplificato) come un insieme di pi\`u modelli semplici che interagiscono.
		\end{labeling}
		
		\subsection{Verifica formale}
		
		Le tecniche di verifica formale si dividono in:
		\begin{labeling}{theorem proving}
			\item [model checking] il modello \`e descritto con un formalismo riconducibile a un grafo di dimensione finita e le propriet\`a sono descritte in termini di raggiungibilit\`a di determinati vertici (o sequenze di questi);
			\item [theorem proving] sia il modello che le propriet\`a sono descritte da formule logiche e la verifica avviene per dimostrazione di teoremi basati su quelle formule.
		\end{labeling}
		
		Gli approcci al model checking tradizionale sono due~\cite{clarke1996formal}:
		\begin{labeling}{model checking basato su automi}
			\item [model checking temporale] il modello \`e descritto da un sistema di transizioni etichettate e le propriet\`a sono descritte da formule di logica temporale (che descrivono i percorsi in termini delle relazioni di ordinamento ``prima'' e ``dopo'' tra stati);
			\item [model checking basato su automi] sia modello che propriet\`a sono descritti da automi e la verifica pu\`o avvenire tramite: inclusione di linguaggi~\cite{har1990software}, raffinamento di ordinamenti ed equivalence checking~\cite{cleaveland1993concurrency}.
		\end{labeling}
		
		I due approcci sono equivalenti ed \`e possibile convertire un modello adatto a un approccio in un modello adatto all'altro~\cite{vardi1986automata}.
		
		Il \emph{model checking simbolico}~\cite{mcmillan1993symbolic} ovvia al problema dell'\emph{esplosione dello spazio degli stati}, ovvero il fenomeno per cui la dimensione del modello aumenta esponenzialmente all'aumentare del numero di componenti del sistema modellato, utilizzando strutture dette \emph{binary decision diagram}~\cite{bryant1986graph}, che ``compattano'' il grafo che si otterrebbe con le tecniche tradizionali.
		Altri approcci per la riduzione del problema dell'esplosione, tramite eliminazione di stati inutili (ovvero astrazione), sono:
		\begin{itemize}
			\item sfruttamento delle informazioni sugli ordinamenti parziali~\cite{peled1994combining};
			\item riduzione della localizzazione~\cite{kurshan1994complexity};
			\item minimizzazione semantica~\cite{elseaidy1997modeling}.
		\end{itemize}

		Una tecnica correlata al model checking (e spesso integrata negli stessi tool), in quanto utilizza le stesse strutture per i modelli, \`e costituita dall'\emph{e\-quiv\-a\-lence checking}. Essa consiste nel verificare che tra due modelli diversi esista una \emph{relazione di equivalenza} comportamentale.
		Se due modelli esibiscono lo stesso comportamento, allora soddisferanno le stesse propriet\`a (in un determinato sottoinsieme della logica temporale utilizzata) e ci\`o consente di coadiuvare il model checking, ad esempio:
		\begin{itemize}
			\item se un modello complesso \`e equivalente a uno talmente semplice da avere le propriet\`a banalmente verificate, \`e possibile bypassare completamente il model checking;
			\item se un modello grande \`e equivalente a uno piccolo, \`e possibile effettuare il model checking su quello piccolo, risparmiando tempo e memoria;
			\item modificando gli algoritmi di equivalence checking, \`e possibile \emph{minimizzare} in maniera automatica un modello, ottenendone uno pi\`u piccolo equivalente.
		\end{itemize}
		
		Per \emph{caratterizzazione logica} di un'equivalenza comportamentale si intende l'insieme di tutte e sole le formule la cui soddisfacibilit\`a \`e mantenuta tra i due modelli equivalenti e pu\`o essere usata come definizione alternativa dell'equivalenza stessa~\cite{bernardo2008survey}.
		 Inoltre se il modello \`e un'algebra di processi (sezione \ref{sez:processalgebra}), \`e possibile definire degli \emph{assiomi di equivalenza}, che permettono di correlare l'equivalenza comportamentale (semantica) all'equivalenza sintattica (ad esempio, se tra gli assiomi di equivalenza \`e presente la propriet\`a distributiva, varr\`a l'equivalenza $a.P + a.Q \sim a.(P + Q)$, senza dover effettuare controlli di natura semantica).
		
		Le equivalenze pi\`u comuni sono:
		\begin{labeling}{equivalenza basata su test}
			\item [equivalenza a tracce] due modelli sono equivalenti se, partendo dagli stati iniziali, tutte le tracce generate coincidono;
			\item [bisimulazione] due modelli sono equivalenti se, partendo dagli stati iniziali, ognuno dei due modelli pu\`o simulare passo dopo passo qualsiasi azione eseguita dall'altro;
			\item [equivalenza basata su test] un test \`e un modello (di struttura arbitraria) basato sulle stesse azioni dei modelli da verificare e arricchito dall'\emph{azione di successo}, un \emph{esperimento} \`e un modello costituito dalla \emph{composizione} di un test e del modello da verificare.
			Si dice che l'esperimento \`e superato se prima o poi sar\`a possibile eseguire l'azione di successo, altrimenti l'esperimento \`e detto fallito. Due modelli sono equivalenti se l'esito degli esperimenti sar\`a lo stesso, per tutti i test costruibili con le loro azioni.
		\end{labeling}
		
		Il \emph{theorem proving} si basa su sistemi deduttivi da cui ottenere la dimostrazione di un teorema nella forma ``modello $\implies$ propriet\`a''.
		Poich\'e sia il modello che le propriet\`a sono formule logiche, \`e possibile operare anche su modelli di dimensione infinita (utilizzando, ad esempio, il principio di induzione). A differenza del model checking, che utilizza modelli molto pi\`u complessi, il theorem proving pu\`o essere effettuato anche per via manuale e l'ampia variet\`a di strumenti a disposizione copre un ampio spettro di interattivit\`a (da quelli semimanuali a quelli completamente automatizzati).
		
	\section{Catene di Markov}\label{sez:markov}
	Un \emph{processo stocastico} \`e un modello che descrive l'andamento di un fenomeno aleatorio in funzione di una variabile indipendente detta \emph{tempo}. Se lo spazio degli stati (ovvero l'insieme di tutte le possibili condizioni in cui pu\`o trovarsi il processo) \`e discreto, il processo prende il nome di \emph{catena}. Se la distribuzione di probabilit\`a del fenomeno \`e caratterizzata dalla \emph{propriet\`a di assenza di memoria}, la catena \`e detta di Markov.
	
	Pi\`u formalmente, sia $X$ la distribuzione di probabilit\`a relativa al tempo di attesa in uno stato, allora la \emph{propriet\`a di Markov} indica che la probabilit\`a di transitare allo stato successivo dopo un tempo $t + s$, condizionata al tempo $s$ gi\`a trascorso, coincide con la probabilit\`a di transitare dopo il tempo $t$ residuo:
	\begin{equation}
		\mathcal{P}(X > t + s \mid X > s) = \mathcal{P}(X > t),\quad \forall t, s > 0.\label{eq:memorylessness}
	\end{equation}
	
	Le due sole distribuzioni di probabilit\`a che soddisfano la propriet\`a di Markov sono le distribuzioni geometrica ($\mathcal{G}_p(n) = p(1 - p)^n$) nel caso di tempo discreto ed esponenziale ($\mathcal{E}_\lambda(t) = \lambda e^{-\lambda t}$) nel caso di tempo continuo.
	Per definizione di probabilit\`a condizionata, si ha che l'equazione \ref{eq:memorylessness} pu\`o essere riscritta come:
	\begin{equation*}
		\mathcal{P}(X > t + s) = \mathcal{P} (X > s) \mathcal{P}(X > t);
	\end{equation*}
	per la distribuzione esponenziale vale:
	\begin{equation*}
		\mathcal{P}(X > t) = \int_{t}^{\infty} \lambda e^{-\lambda x} dx = e^{-\frac{t}{\lambda}},
	\end{equation*}
	ma allora, per sostituzione, si ha:
	\begin{align*}
		\mathcal{P}(X > t + s) &= e^{-\frac{t + s}{\lambda}}\\
		&= e^{-\frac{t}{\lambda}} e^{-\frac{s}{\lambda}}\\
		&= \mathcal{P} (X > t) \mathcal{P}(X > s),
	\end{align*}
	dimostrando il caso per la distribuzione esponenziale~\cite{expforgetfulness}; con un ragionamento analogo si dimostra il caso della distribuzione geometrica~\cite{geomforgetfulness}.

	Una catena di Markov \`e detta a tempo discreto se il tempo viene osservato in \emph{passi} (di conseguenza tutte le catene di Markov a tempo discreto saranno governate da distribuzioni geometriche), se invece il tempo viene osservato con continuit\`a, viene detta \emph{a tempo continuo} (e sar\`a conseguentemente governata da distribuzioni esponenziali).
	
	Di seguito ci si concentrer\`a sulle catene di Markov a tempo continuo (CTMC), pi\`u adatte alla modellazione di reti metaboliche: dalla teoria degli urti, accennata nella sezione \ref{sez:caratt}, conseguono infatti l'assenza di memoria di una reazione chimica (in quanto la possibilit\`a o meno che avvenga un urto efficace tra due molecole di substrato non dipende da quanti urti, efficaci o meno, hanno gi\`a subito le molecole) e la natura continua del processo (lo spostamento delle molecole, e conseguentemente gli urti, non avvengono ``a scatti'').
	
	Formalmente una CTMC (secondo il formalismo adattato al model checking) \`e una tupla $\mathcal{M} = (\mathbb{S}, \mathbf{R}, \mathcal{P}_{init}, \mathbb{AP}, \mathcal{L})$, dove:
	\begin{itemize}
		\item $\mathbb{S}$ \`e lo spazio degli stati;
		\item $\mathbf{R}$ \`e la matrice delle transizioni $\mathbf{R}: \mathbb{S}\times \mathbb{S} \mapsto \mathbb{R}_{\geq 0}$, una struttura che indica, per ogni coppia di stati $(s_1, s_2)$, con che rate (esponenziale) avviene la transizione $s_1 \rightarrow s_2$ (poich\'e il rate \`e l'inverso del tempo medio di attesa, una transizione che non pu\`o avvenire avr`a rate pari a 0);
		\item $\mathcal{P}_{init}$ \`e la distribuzione di probabilit\`a iniziale, che indica da quali stati (e con che probabilit\`a) dovranno originarsi i cammini sulla catena;
		\item $\mathbb{AP}$ \`e un insieme di proposizioni atomiche che rende pi\`u leggibile lo spazio degli stati;
		\item $\mathcal{L}$ \`e la funzione di etichettamento degli stati $\mathcal{L}: \mathbb{S} \mapsto 2^{\mathbb{AP}}$, che associa a ogni stato zero o pi\`u etichette (ad esempio, associando allo stato $s_1$ le etichette ``X = \SI{10}{\mu mol}'', ``Y non ha ancora reagito'', ``Z esaurito'' e ``soluzione di colore azzurro''), rendendo pi\`u facilmente interpretabile la catena.
	\end{itemize}
	
	Una catena di Markov si presta bene sia a tecniche simulative, basandosi sull'osservazione della matrice $\mathbf{R}$, che a tecniche di verifica, basandosi su formule di logica costruite su $\mathbb{AP}$.
		
	Ulteriore potere espressivo \`e dato dall'estensione della catena tramite strutture dette \emph{reward}, che rappresentano dei punteggi dal significato arbitrario (e.g.\ costi, tempi, quantit\`a).
	Esistono due tipi di reward:
	\begin{labeling}{reward istantaneo}
		\item [reward istantaneo] punteggio applicato una volta sola, al passaggio da uno stato al successivo;
		\item [reward cumulativo] punteggio moltiplicato al tempo di soggiorno nello stato a cui \`e associato.
	\end{labeling}
	
	La tupla $\mathcal{M}$ viene estesa con la matrice $\iota : \mathbb{S} \times \mathbb{S} \mapsto \mathbb{R}$ che mappa ogni coppia di stati in un reward istantaneo e/o la funzione $\rho : \mathbb{S} \mapsto \mathbb{R}$ che associa un reward stazionario ad ogni stato.
	
	La tupla estesa $\mathcal{M} = (\mathbb{S}, \mathbf{R}, \mathcal{P}_{init}, \iota, \rho, \mathbb{AP}, \mathcal{L})$ prende il nome di Markov reward model (MRM)~\cite{aldini2007mixing}.
	
	\`E possibile rappresentare una catena di Markov tramite un diagramma a stati e transizioni, dove gli stati sono etichettati da $\mathcal{L}$ e le transizioni da $\mathbf{R}$.
	Come \`e intuibile, questa rappresentazione permette di effettuare sia operazioni di simulazione che di model checking, in termini di cammini sulla catena.
	In figura \ref{fig:markov} \`e rappresentata una catena con funzione di etichettamento:
	\begin{equation*}
		\mathcal{L} : \{s_0 \mapsto A, s_1 \mapsto B, s_2 \mapsto C\}
	\end{equation*}
	 e matrice delle transizioni:
	 \begin{equation*}
	 	\mathbf{R} = \begin{matrix}
	 	0&0.5&0\\
	 	100&2&1\\
	 	3.14&0&0
	 	\end{matrix}.
	 \end{equation*}
	
	\begin{figure}
		\center
		\begin{tikzpicture}[->, auto, node distance=4cm and 4cm, on grid, semithick, state/.style={circle, draw, black, minimum width=2cm}]
		\node[state] (EpS) []{$A$};
		\node[state] (ES) [below right= of EpS]{$B$};
		\node[state] (EpP) [above right= of ES]{$C$};
		
		\path (EpS) edge[bend right=15] node[left=0.5cm]{$0.5$} (ES);
		\path (ES) edge[bend right=15] node[right=0.5cm]{$100$} (EpS);
		\path (ES) edge[bend right=15] node[below=0.5cm]{$1$} (EpP);
		\path (ES) edge[loop below] node{$2$} (ES);
		\path (EpP) edge[bend right=15] node[above=0.5cm]{$3.14$} (EpS);
		
		
		\end{tikzpicture}
		\caption{Rappresentazione grafica di una catena di Markov a tempo continuo}
		\label{fig:markov}
	\end{figure}

		\subsection{Linguaggio di modellazione di PRISM}\label{sez:prism}
		PRISM~\cite{kwiatkowska2011prism} \`e un model checker per modelli probabilistici basati su catene di Markov, che utilizza un formalismo state-based per semplificare la creazione dei modelli.
		I formalismi supportati sono:
		\begin{labeling}{catene di Markov a tempo continuo (CTMC)}
			\item [catene di Markov a tempo discreto (DTMC)] la risoluzione delle scelte \`e probabilistica;
			\item [processi decisionali di Markov (MDP)]  la risoluzione pu\`o avvenire sia non deterministicamente che probabilisticamente;
			\item [catene di Markov a tempo continuo (CTMC)] la risoluzione avviene tramite race policy su transizioni temporizzate esponenzialmente;
			\item [automi probabilistici temporizzati (PTA)] la risoluzione pu\`o avvenire sia non deterministicamente che con race policy.
		\end{labeling}
		
		Un modello PRISM \`e costituito da un insieme di \emph{moduli} -- ognuno dotato di un proprio \emph{stato interno} e di un comportamento -- che possono essere considerati delle piccole catene indipendenti.
		
		All'interno di un modulo, lo stato \`e rappresentato da \emph{variabili} che possono assumere valori appartenenti a un intervallo limitato e viene aggiornato ad ogni transizione.
		Le transizioni possono essere \emph{etichettate} da un'azione e \emph{abilitate} da un'espressione di guardia.
		Il comportamento di un modulo \`e descritto da una serie di \emph{comandi} nella forma:
		\begin{equation*}
			[azione] (guardia) \rightarrow prob_1 : agg_1 + prob_2 : agg_2 + \cdots + prob_n : agg_n,
		\end{equation*}
		dove $azione$ \`e l'etichetta da associare alla transizione, $guardia$ \`e un predicato che attiva la transizione solo negli stati in cui viene soddisfatto, $prob_n$ \`e una probabilit\`a (per DTMC e MDP) o un rate (per CTMC e PTA) e $agg_n$ \`e un'istruzione di aggiornamento.
		La scelta tra pi\`u aggiornamenti (separati dall'operatore $+$) avviene tramite probabilit\`a o race policy.
		Le istruzioni di aggiornamento sono nella forma:
		\begin{equation*}
			(x' = expr_x)\&(y' = expr_y)\&\cdots \& (z' = expr_z),
		\end{equation*}
		dove ogni variabile (a sinistra) \`e aggiornata valutando l'espressione a destra dell'uguale.
		Se delle variabili vengono omesse, si assume che il loro valore non cambi.
		
		La catena di Markov a tempo continuo di figura \ref{fig:markov} pu\`o essere rappresentata da un singolo modulo contenente una sola variabile e con il comportamento descritto dai seguenti comandi:
		\begin{align*}
			[] (stato = A) &\rightarrow 0.5 : stato' = B\\
			[] (stato = B) &\rightarrow 100 : stato' = A + 2 : stato' = B + 1 : stato' = C\\
			[] (stato = C) &\rightarrow 3.14 : stato' = A\\
		\end{align*}		
		
		Il modello del sistema \`e ottenuto per composizione dei singoli moduli, tramite \emph{sincronizzazione} su tutte le azioni aventi lo stesso nome.
		PRISM presenta quattro motori computazionali (applicati a tutti i modelli a eccezione degli automi probabilistici temporizzati) per il \emph{model checking esatto}, che utilizzano rappresentazioni diverse del modello per ottimizzare le prestazioni nella maggior parte dei casi d'uso:
		\begin{labeling}{MTBDD}
			\item [explicit] il modello \`e rappresentato con una matrice delle transizioni e il model checking avviene per via tradizionale, con prestazioni temporali elevate, ma occupazione in memoria altrettanto elevata: ci\`o lo rende inadeguato per modelli di grosse dimensioni;
			\item [sparse] il modello \`e rappresentato con una matrice sparsa, risparmiando enormi quantit\`a di memoria se il modello contiene molti stati non raggiungibili (in caso contrario l'occupazione in memoria \`e peggiore della matrice esplicita), anche in questo caso il model checking \`e di tipo tradizionale;
			\item [MTBDD] la catena viene rappresentata da un multi terminal binary decision diagram (sezione \ref{sez:mtbdd}), risparmiando molta memoria se il modello presenta un elevato livello di regolarit\`a (ad esempio molte transizioni che convergono in uno stesso stato con lo stesso rate) al costo di prestazioni temporali ridotte; per modelli altamente irregolari l'occupazione in memoria potrebbe essere di molto maggiore rispetto alla matrice esplicita;
			\item [hybrid] la catena viene rappresentata sia da un MTBDD che da una matrice sparsa e il model checking \`e di tipo ibrido~\cite{kwiatkowska2004probabilistic}, nella maggior parte dei casi offre il miglior compromesso tra occupazione in memoria e tempo di calcolo.
		\end{labeling}
		
		PRISM \`e inoltre in grado di effettuare \emph{model checking statistico}~\cite{kwiatkowska2018probabilistic}, che permette di stimare l'esito di una propriet\`a a partire da \emph{simulazioni} sul modello, nel caso in cui le tecniche esatte non siano attuabili.
		
		PRISM \`e stato impiegato con successo in casi di studio in ambito biologico come i seguenti:
		\begin{itemize}
			\item controllo del ciclo cellulare eucariota~\cite{cyclin};
			\item ciclo circadiano astratto~\cite{circadian};
			\item via FGF~\cite{heath2008probabilistic};
			\item DNA computing~\cite{dannenberg2013dna};
			\item fusione dei virus dell'influenza~\cite{dobay2011many};
			\item cinetica dei ribosomi~\cite{bovsnacki2009silico};
			\item segnalazione dei linfociti T~\cite{owens2008modelling};
			\item pacemaker cardiaci~\cite{chen2012quantitative}.
		\end{itemize}

	\section{Algebre di processi}\label{sez:processalgebra}
	Le algebre di processi sono un formalismo action-based che estremizza il concetto di composizionalit\`a definendo un modello come un espressione costituita da operazioni (che determinano interazioni) tra entit\`a dette \emph{processi}: cos\`i come l'algebra dei numeri reali permette la costruzione di espressioni quali $3 + 2 \cdot 1.4$, un'algebra di processi permette di costruire modelli del tipo $a.(b.\underline{0} + c.\underline{0})$.
	
	Un processo \`e un'entit\`a che pu\`o compiere \emph{azioni} (evolvendo in altri processi) o rimanere bloccata (e in tal caso si indica come \emph{processo nullo}, rappresentato da $\underline{0}$) e le operazioni definite dall'algebra permettono di descrivere tutti i possibili comportamenti (intesi come tracce o punti di scelta) tramite un'espressione che utilizzi $\underline{0}$ o costanti (ad esempio $P$) che identifichino gli altri processi. Le operazioni definite da un'algebra di processi generica sono:
	\begin{labeling}{composizione parallela}
		\item [prefisso d'azione] $a.P$: viene eseguita l'azione $a$ e si evolve nel processo $P$;
		\item [scelta] $P_1 + P_2$: si evolve in $P_1$ oppure $P_2$, secondo un meccanismo di risoluzione (come quelli esemplificati nella sezione \ref{sez:silico}) dipendente dall'algebra di processi. Nel caso di meccanismi diversi da quello non deterministico, \`e necessario arricchire le azioni con informazioni aggiuntive;
		\item [composizione parallela] $P_1 \underset{\mathbb{A}}{\bowtie} P_2$: $P_1$ e $P_2$ evolvono parallelamente, indipendentemente tra loro per ogni azione non contenuta nell'insieme $\mathbb{A}$, ma aspettando ognuno la controparte (eventualmente per un tempo infinito) per eseguire \emph{contemporaneamente} le azioni contenute in $\mathbb{A}$;
		\item [hiding] $P / \mathbb{A}$: tutte le azioni contenute in $\mathbb{A}$ diventano \emph{invisibili} (indicate con $\tau$). Questo operatore \`e particolarmente utile per l'equivalence checking basato su \emph{equivalenze deboli} (varianti delle equivalenze comportamentali che consentono di ignorare le azioni invisibili, nel caso della bisimulazione debole, ad esempio, ogni processo dovr\`a simulare passo dopo passo solo le azioni visibili della controparte, potendo ignorare tutte quelle invisibili);
		\item [restrizione] $P \backslash \mathbb{A}$: nessuna delle azioni contenute nell'insieme $\mathbb{A}$ pu\`o essere eseguita;
		\item [ricorsione] il processo evolve in s\'e stesso (tipicamente dopo aver eseguito alcune azioni), la notazione fa utilizzo delle costanti di processo e di equazioni di definizione ($P = comportamento$, dove $P$ compare all'interno del comportamento stesso).
	\end{labeling}
	
	Un'equazione di process algebra che pu\`o generare il diagramma a stati e transizioni della figura \ref{fig:diag1} \`e la seguente:
	\begin{equation*}
		P = replicazione.P + trascrizione.(retrotrascrizione.P + traduzione.\underline{0}),
	\end{equation*}
	tuttavia risulta pi\`u agevole definire un sistema di equazioni con un numero maggiore di costanti (e con identificatori pi\`u descrittivi) nel seguente modo:
	\begin{align*}
		DNA &= replicazione.DNA + trascrizione.RNA\\
		RNA &= retrotrascrizione.DNA + traduzione.PROTEINA.
	\end{align*}
	Dal sistema cos\`i definito risulta pi\`u semplice ricavare il diagramma a stati e transizioni, essendo sufficiente (in questo caso in cui manca l'operatore di sincronizzazione) creare uno stato per ogni processo ed etichettare le transizioni uscenti con le azioni abilitate in ognuno di essi.

	Il contenuto informativo delle azioni, che determina il meccanismo di risoluzione, varia per ogni algebra di processi.
	I modi pi\`u comuni per etichettare le transizioni sono:
	\begin{labeling}{$\langle azione, priorit\`a\rangle$}
		\item [$azione$] associare soltanto un'etichetta (eventualmente $\tau$) alla transizione, per algebre di processi \emph{non deterministiche};
		\item [$\langle azione, priorit\`a\rangle$] per algebre di processi \emph{prioritarie};
		\item [$\langle azione, tempo\rangle$] per algebre di processi \emph{temporizzate deterministicamente};
		\item [$\langle azione, rate\rangle$] per algebre di processi \emph{stocastiche} (dove il $rate$ indica il parametro da associare a una distribuzione di probabilit\`a, tipicamente dotata di propriet\`a di Markov, usata per risolvere la race policy).
	\end{labeling}
	
	Tra le algebre di processi stocastiche, spiccano quelle temporizzate esponenzialmente (tutte le azioni hanno distribuzione $\mathcal{E}(t) = \lambda e^{-\lambda t}$), dove la race policy \`e esprimibile (nel caso in cui i due rate non coincidano) tramite:
	\begin{align*}
		\langle a, \lambda \rangle.P_1 + \langle b, \mu \rangle.P_2 &\xrightarrow{a, \lambda} P_1 \text{ con probabilit\`a } \frac{\lambda}{\lambda + \mu}\\
		\langle a, \lambda \rangle.P_1 + \langle b, \mu \rangle.P_2 &\xrightarrow{b, \mu} P_2 \text{ con probabilit\`a } \frac{\mu}{\lambda + \mu}.
	\end{align*}
	
	Le algebre temporizzate esponenzialmente sono equivalenti a un diagramma a stati e transizioni esteso da informazioni sui rate di transizione, ovvero da catene di Markov a tempo continuo etichettate da azioni (ACTMC).
		
		\subsection{Bio-PEPA}\label{sez:biopepa}
		Bio-PEPA~\cite{ciocchetta2009bio} \`e un'algebra di processi basata su Performance Evaluation Process Algebra (PEPA)~\cite{pepa} (un'algebra temporizzata esponenzialmente), alla quale aggiunge informazioni di tipo \emph{quantitativo} sui processi, finalizzate a semplificare la modellazione di sistemi biologici.
		
		Da un unico modello Bio-PEPA sono ottenibili modelli in altri formalismi che descrivono pi\`u aspetti del sistema modellato:
		\begin{itemize}
			\item un modello PRISM, che rappresenta una CTMC e una serie di formule CSL preformate, permette di effettuare model checking e simulazioni;
			\item un modello basato su equazioni differenziali ordinarie permette di effettuare analisi sulla cinetica;
			\item un modello StockKit~\cite{sanft2011stochkit2} permette di compilare un simulatore efficiente in linguaggio C.
		\end{itemize}
		I risultati delle analisi sui vari modelli vengono integrati automaticamente in un unico rapporto, contenente il modello Bio-PEPA e vari grafici ottenuti dalle analisi.
		
		Alcuni casi studio in cui \`e stato applicato Bio-PEPA sono:
		\begin{itemize}
			\item via NF-$\kappa$ B~\cite{ciocchetta2010modelling};
			\item ciclo circadiano di \emph{Neurospora crassa}~\cite{akman2009modelling};
			\item via gp130 JAK STAT~\cite{guerriero2009qualitative};
			\item via cAMP/PKA/MAPK~\cite{ciocchetta2009compartmental};
			\item modelli epidemiologici per l'influenza aviaria~\cite{ciocchetta2010bio};
			\item ciclo circadiano di \emph{Ostreococcus tauri}~\cite{akman2010complementary}.
		\end{itemize}
		
		Un modello Bio-PEPA \`e composto da tre sezioni:
		\begin{itemize}
			\item quantit\`a iniziali e costanti, definite in una matrice memorizzata in un file .csv: ogni riga della matrice costituisce un'\emph{istanza} dello stesso modello con valori diversi, agevolando l'esecuzione di pi\`u \emph{esperimenti} sullo stesso modello;
			\item cinetiche e contatori, raggruppati nella sezione iniziale del file .biopepa;
			\item processi veri e propri, raggruppati nella sezione finale del file .biopepa.
		\end{itemize}
		
		Ogni azione \`e temporizzata esponenzialmente e il rate viene calcolato tramite espressioni (dette \emph{cinetiche}, anche se non vengono necessariamente utilizzate a tale scopo), che utilizzano la quantit\`a corrente ed eventuali costanti.
		Ci\`o permette, ad esempio, di modellare l'andamento di una reazione, variandone il rate in funzione della concentrazione di substrati.
		Un \emph{contatore} \`e un costrutto Bio-PEPA che permette di rappresentare grandezze derivate dal modello (ad esempio la carica energetica, come rapporto tra i nucleotidi trifosfati e i nucleotidi totali).
		Sia cinetiche che contatori utilizzano la sintassi:
		\begin{align*}
			azione &= [espressione];\\
			CONTATORE &= [espressione];
		\end{align*}
		dove $espressione$ pu\`o contenere qualsiasi identificatore definito nel file delle quantit\`a, costante numerica e operatore aritmetico.
		
		In Bio-PEPA, la  definizione dei processi avviene secondo un approccio \emph{bottom-up}, definendo prima il comportamento dei singoli elementi del sistema e poi componendoli (tipicamente tramite operatore di composizione parallela) in un unico processo rappresentante l'intero modello.
		Per tenere conto delle quantit\`a, le azioni sono modificate nella coppia $(azione, stechiometria)$ (essendo il rate definito implicitamente nella sezione relativa alle cinetiche) e l'operatore prefisso d'azione \`e modificato nelle seguenti varianti:
		\begin{labeling}{modulatore generico}
			\item [reagente] $(azione, stechiometria) \downarrow P$: se le quantit\`a sono superiori a $stechiometria$, il processo diminuisce di tale quantit\`a ed evolve in $P$;
			\item [prodotto] $(azione, stechiometria) \uparrow P$: il processo aumenta di $stechiometria$ unit\`a (eventualmente saturandosi fino a un massimo) ed evolve in $P$;
			\item [attivatore] $(azione, stechiometria) \oplus P$: il processo abilita $azione$ se \`e presente in quantit\`a superiori a $stechiometria$, poi evolve in $P$ senza alterare di quantit\`a;
			\item [inattivatore] $(azione, stechiometria) \ominus P$: il processo disabilita $azione$ se \`e presente in quantit\`a superiori a $stechiometria$, poi evolve in $P$ senza alterare di quantit\`a;
			\item [modulatore generico] $(azione, stechiometria) \odot P$: il processo evolve in $P$ senza alterare di quantit\`a, questo operatore coincide con il prefisso d'azione tradizionale.
		\end{labeling}
		
		Un modello esemplificativo, tratto da~\cite{ciocchetta2009bio} (a cui si rimanda per i risultati ottenuti dall'analisi sul modello), sulla regolazione genica a feedback negativo basata su dimerizzazione in \emph{E.\ coli}~\cite{bundschuh2003fluctuations} \`e il seguente:
				
		\begin{align*}
			\alpha_1 &= \left[\frac{\nu}{K_M + P2}\right];\\
			\alpha_2 &= [fMA(k_2)];\\
			\alpha_3 &= [fMA(k_3)];\\
			\alpha_4 &= [fMA(k_4)];\\
			\alpha_5 &= [fMA(k_5)];\\
			\alpha_{5i} &= [fMA(k_{5i})];
		\end{align*}
		\begin{align*}
			M &= (\alpha_1, 1) \uparrow M + (\alpha_2, 1) \oplus M + (\alpha_3, 1) \downarrow M;\\
			P &= (\alpha_2, 1) \uparrow P + (\alpha_4, 1) \downarrow P + (\alpha_5, 2) \downarrow P + (\alpha_{5i}, 2) \uparrow P;\\
			P2 &= (\alpha_1, 1) \ominus P2 + (\alpha_5, 1) \uparrow P2 + (\alpha_{5i}, 1) \downarrow P2;\\
		\end{align*}
		\begin{equation}\label{eq:biopepa}
			M \underset{\alpha_2}{\bowtie} P \underset{\{\alpha_5, \alpha_{5i}\}}{\bowtie} P2.
		\end{equation}
		
				\begin{table}[H]
					\centering
					\begin{tabular}{| c | c |}
						\hline
						M & \SI{0}{nM}\\
						\hline
						P & \SI{0}{nM}\\
						\hline
						P2 & \SI{0}{nM}\\
						\hline
						$\nu$ & \SI{2.19}{s^{-1}}\\
						\hline
						$K_M$ & \SI{356}{nM}\\
						\hline
						$k_2$ & \SI{0.043}{s^{-1}}\\
						\hline
						$k_3$ & \SI{0.0039}{s^{-1}}\\
						\hline
						$k_4$ & \SI{0.0007}{s^{-1}}\\
						\hline
						$k_5$ & \SI{0.025}{s^{-1} nM^{-1}}\\
						\hline
						$k_{5i}$ & \SI{0.5}{s^{-1}}\\
						\hline
					\end{tabular}
					\caption{Parametri del modello sulla regolazione genica basata su dimerizzazione di \emph{E.\ coli}}
				\end{table}
		Il modello \`e composto da sei reazioni:
		\begin{labeling}{$\alpha_{5i}$}
			\item [$\alpha_1$] trascrizione dell'mRNA ($M$), inibita dal dimero di proteina ($P2$);
			\item [$\alpha_2$] traduzione dei monomeri di proteina ($P$), promossa dall'mRNA;
			\item [$\alpha_3$] degradazione dell'mRNA;
			\item [$\alpha_4$] degradazione della proteina;
			\item [$\alpha_5$] dimerizzazione della proteina;
			\item [$\alpha_{5i}$] rottura del dimero di proteina.
		\end{labeling}
		Ad eccezione di $\alpha_1$, tutte seguono la legge di azione di massa ($fMA$). Osservando i coefficienti stechiometrici, \`e evidente che siano tutte del primo ordine ($fMA(k) = k \frac{[prodotti]}{[reagenti]}$), esclusa $\alpha_5$, del secondo ordine ($fMA(k) = k \frac{[prodotti]}{[reagenti]^2}$).
		
	
	\section{Logiche temporali}\label{sez:logiche}
	Le logiche temporali sono un'estensione della logica dei predicati, basate su una relazione d'ordine di tipo ``prima-dopo''; grazie a tale relazione risultano uno strumento particolarmente efficace per descrivere \emph{percorsi} su un modello e, di conseguenza, propriet\`a verificabili tramite model checking.
	
	Partendo da \emph{predicati} che possono essere veri o falsi (atomici), \`e possibile costruirne di nuovi combinandoli con \emph{connettivi logici}. I connettivi pi\`u comuni sono:
	\begin{labeling}{$\phi_1 \iff \phi_2$}
		\item [$\neg \phi$] vero solo se $\phi$ \`e falso;
		\item [$\phi_1 \wedge \phi_2$] vero se sia $\phi_1$ che $\phi_2$ sono veri;
		\item [$\phi_1 \vee \phi_2$] vero se almeno uno tra $\phi_1$ e $\phi_2$ \`e vero;
		\item [$\phi_1 \rightarrow \phi_2$] vero se $\phi_1$ e $\phi_2$ sono entrambi veri, oppure se $\phi_1$ \`e falso;
		\item [$\phi_1 \leftrightarrow \phi_2$] vero se $\phi_1$ e $\phi_2$ sono entrambi veri oppure entrambi falsi.
	\end{labeling}
	
	Aggiungendo delle variabili, \`e possibile \emph{quantificare} i predicati:
	\begin{labeling}{$\forall x (\phi)$}
		\item [$\forall x (\phi)$] vero se $\phi$ \`e vero per tutti i possibili valori di $x$;
		\item [$\exists x (\phi)$] vero se $\phi$ \`e vero per almeno un valore di $x$.
	\end{labeling}
	
	Gli \emph{operatori temporali} aggiungono la possibilit\`a di indicare quando dovr\`a essere vero un predicato, i pi\`u comuni (rivolti al futuro) sono:
	\begin{labeling}{$\phi_1 \mathcal{R} \phi_2$}
		\item [$\mathcal{X} \phi$] vero se $\phi$ sar\`a vero nell'immediato futuro;
		\item [$\mathcal{F} \phi$] vero se prima o poi $\phi$ sar\`a vero;
		\item [$\mathcal{G} \phi$] vero se $\phi$ \`e sempre vero;
		\item [$\phi_1 \mathcal{U} \phi_2$] vero se $\phi_1$ \`e vero finch\'e non inizier\`a a valere $\phi_2$;
		\item [$\phi_1 \mathcal{R} \phi_2$] vero se $\phi_2$ \`e vero finch\'e $\phi_1$ rimane vero.
	\end{labeling}
	
	\`E possibile applicare le formule di una logica temporale a un modello a stati e transizioni (ad esempio una catena di Markov a tempo continuo), facendo coincidere l'insieme di predicati con l'insieme di etichette e definendo un'unica variabile relativa a un generico percorso $\omega$.
	Sulla catena di figura \ref{fig:markov} \`e, ad esempio, possibile definire le propriet\`a $A$ (``lo stato corrente \`e $A$''), $A \wedge \mathcal{X} B$ (``lo stato corrente \`e $A$ e il prossimo sar\`a $B$'') o $\exists \omega (B \mathcal{U} C)$ (compattato in $\exists(B \mathcal{U} C$), ``esiste almeno un percorso in cui $B$ \`e vero finch\'e non inizier\`a a valere $C$'').
	
	La logica cos\`i costruita prende il nome di CTL* ed \`e molto espressiva, non imponendo vincoli su come combinare gli operatori: a tale espressivit\`a corrisponde il fatto che il problema della \emph{soddisfacibilit\`a} (data una formula e un modello, dire se \`e vera o falsa per quel modello) sia \emph{intrattabile} (classe di complessit\`a $2-EXPTIME$~\cite{visser2000practical}, sebbene in passato sia stato considerato addirittura \emph{indecidibile}~\cite{emerson1988complexity}).
	
	Imponendo dei vincoli su una logica temporale, \`e possibile creare delle logiche meno espressive, ma di cui \`e possibile calcolare la soddisfacibilit\`a, che si possono classificare in:
	\begin{labeling}{branching time}
		\item [linear time] \`e possibile ragionare solo su un percorso alla volta, quindi i quantificatori ($\forall$ e $\exists$) sono eliminati; la logica linear time ottenuta a partire da CTL* prende il nome di \emph{linear temporal logic} (LTL);
		\item [branching time] \`e possibile ragionare solo sulle scelte disponibili, quindi gli operatori temporali ($\mathcal{X}$, $\mathcal{F}$, $\mathcal{G}$, $\mathcal{U}$ e $\mathcal{R}$) devono essere sempre quantificati (``tra le scelte a disposizione, almeno una oppure tutte verificheranno una certa condizione''); la logica branching time ottenuta a partire da CTL* prende il nome di \emph{computation tree logic} (CTL, senza asterisco).
	\end{labeling}
	
	Probabilistic CTL (PCTL)~\cite{hansson1994logic} estende CTL con gli operatori:
	\begin{labeling}{$\mathbb{S}$}
		\item [$\mathbb{P}$] operatore di probabilit\`a, vero se la probabilit\`a che sia verificata la sottoformula a cui \`e applicato \`e limitata da un certo intervallo, considerando le possibili scelte;
		\item [$\mathbb{S}$] operatore di probabilit\`a allo steady state, come $\mathbb{P}$, ma allo stato stazionario.
	\end{labeling}
	
	I due operatori possono essere usati sia per ottenere valori numerici, nella forma $\mathbb{P}_{=?} [\phi]$ (``quale \`e la probabilit\`a che $\phi$ sia vera?''), che per ottenere valori di verit\`a, nella forma $\mathbb{P}_{[a,b]} [\phi]$ (``la probabilit\`a che $\phi$ sia vera \`e compresa tra $a$ e $b$?'').
	In PCTL i quantificatori risultano superflui, in quanto valgono le equivalenze $\forall(\phi) = \mathbb{P}_{=1}[\phi]$ e $\exists(\phi) = \mathbb{P}_{>0}[\phi]$.
	
		\subsection{Continuous Stochastic Logic}
		Passando da un modello probabilistico a uno temporizzato esponenzialmente, \`e necessario poter catturare anche aspetti relativi al tempo trascorso.
		CSL~\cite{baier2003model} modifica gli operatori temporali di PCTL rendendoli \emph{time-bounded}, ovvero imponendo un limite temporale entro cui devono essere verificati.
		
		Gli operatori time-bounded si indicano con $\mathcal{OP}^{[a,b]}$, dove $\mathcal{OP}$ \`e un'operatore temporale e $[a,b]$ \`e l'intervallo temporale entro quale deve valere la sottoformula.
		\`E possibile ricavare gli operatori temporali illimitati (tradizionali) osservando che $\mathcal{OP} = \mathcal{OP}^{< \infty}$, quindi CSL \`e pi\`u espressivo di PCTL.
		
		Un esempio di propriet\`a CSL da verificare sul modello \ref{eq:biopepa} (dove, a seguito della conversione, il processo $P$ corrisponde alla componente $\_P$) pu\`o essere:
		\begin{equation*}
			\mathbb{P}_{=?}[\mathcal{G}^{\leq 0.5} \_P2 > 10]
		\end{equation*}
		``Quale \`e la probabilit\`a che il dimero di proteina sia \emph{sempre} in quantit\`a maggiore di 10 unit\`a, nelle prime 0.5 unit\`a di tempo?''\\
		\`E importante notare che ci\`o che avviene fuori dall'intervallo in cui sono limitati gli operatori viene \emph{ignorato} e non deve necessariamente essere falso (nell'esempio il calcolo della probabilit\`a non \`e influenzato dal fatto che $\_P2 > 10$ sia vero o meno dal tempo 0.51 al tempo 10).
		
		Se il modello \`e arricchito da reward, \`e possibile estendere CSL con l'operatore $\mathbb{R}$, che permette di misurare il valore dei reward accumulati lungo un percorso: la logica cos\`i arricchita prende il nome di \emph{continuous stochastic reward logic} (CSRL)~\cite{baier2000logical}.		
		
		PRISM definisce cinque varianti per l'operatore $\mathbb{R}$~\cite{reward}:
		\begin{labeling}{$\mathbb{R}_{\{rew\}} {[\mathcal{C} \leq T]}$}
			\item [$\mathbb{R}_{\{rew\}} {[\mathcal{F} \phi]}$] accumula i reward, definiti secondo la struttura $rew$, finch\'e il predicato $\phi$ non \`e verificato;
			\item [$\mathbb{R}_{\{rew\}} {[\mathcal{C} \leq T]}$] accumula i reward, definiti secondo $rew$, fino al tempo $T$;
			\item [$\mathbb{R}_{\{rew\}} {[\mathcal{C}]}$] accumula i reward, definiti secondo $rew$, fino a un tempo infinito;
			\item [$\mathbb{R}_{\{rew\}} {[\mathcal{I} = T]}$] restituisce i reward, definiti secondo $rew$, all'istante $T$;
			\item [$\mathbb{R}_{\{rew\}} {[\mathcal{S}]}$] restituisce i reward, definiti secondo $rew$, allo stato stazionario.
		\end{labeling}
		Come $\mathbb{P}$ ed $\mathbb{S}$, anche $\mathbb{R}$ pu\`o essere usato nelle varianti $\mathbb{R}_{=?}$ e $\mathbb{R}_{[a, b]}$, permettendo il calcolo di un valore o la verifica che il valore sia entro un intervallo.
	
	\section{Simulazioni su catene di Markov a tempo continuo}\label{sez:simul}
	In una CTMC, la matrice delle transizioni $\mathbf{R}$ pu\`o anche essere vista come il prodotto tra la probabilit\`a di transizione e il vettore dei tempi medi di soggiorno ($\mathbf{R}_{i,j} = \mathbf{P}_{i,j} \cdot \mathbf{r}_{i}$): astraendo da $\mathbf{r}$ si ottiene una catena di Markov a tempo discreto, detta \emph{embedded Markov chain} (EMC)~\cite{simulationmarkov}.
	
	L'algoritmo di Gillespie~\cite{gillespie1977exact} (applicato inizialmente alla simulazione di un sistema di reazioni chimiche e successivamente generalizzato alle catene di Markov a tempo continuo) costituisce una procedura numerica esatta per la simulazione di eventi stocastici, basandosi su tecniche Monte Carlo.
	
	Partendo da uno stato iniziale, estratto casualmente dalla CTMC secondo la distribuzione $\mathcal{P}_{init}$, si genera iterativamente un percorso, finch\'e non si raggiunge un tempo di terminazione $T$; ad ogni istante, se lo stato corrente \`e $s_k$, il tempo di soggiorno viene estratto casualmente secondo una distribuzione esponenziale con rate $\mathbf{r}_k$ e lo stato successivo \`e scelto estraendo una variabile aleatoria con distribuzione pari alla riga della matrice delle probabilit\`a $\mathbf{P}_{k,-}$.
	Qualora sia necessario ottenere valori esatti, la media di pi\`u percorsi avr\`a un andamento convergente a quello del fenomeno modellato.
	
	\begin{algorithm}[H]
			\KwData{$\mathcal{M} = (\mathbb{S}, \mathbf{P}, \mathbf{r}, \mathcal{P}_{init}), T$}
			\KwResult{Percorso sulla catena fino al tempo $T$}
			$s_0 \leftarrow$ stato iniziale estratto da $\mathcal{P}_{init}$\;
			$k \leftarrow 0$\;
			$t \leftarrow 0$\;
			\While{$t < T$}{
				$P_{s_k} \leftarrow$ riga di $\mathbf{P}$ corrispondente allo stato $s_k$\;
				$s_{k+1}$ estratto casualmente secondo la distribuzione $P_{s_k}$\;
				$t_k$ estratto casualmente secondo una distribuzione esponenziale con rate $\mathbf{r}_{s_k}$\;
				$t \leftarrow t + t_k$\;
				$k \leftarrow k + 1$\;
			}
		\KwRet{$s_0 \xrightarrow{t_0} s_1 \xrightarrow{t_1} \dots \xrightarrow{t_{n-1}} s_n$.}
		\caption{Algoritmo di Gillespie per la simulazione di un percorso su una CTMC fino al tempo T}
		\label{alg:gillespie}
	\end{algorithm}
	
	\section{Model checking di catene di Markov a tempo continuo}\label{sez:modelcheck}
	CSL \`e formalmente definito da tutte le formule costruibili a partire dalla seguente grammatica:
	\begin{align}
		\phi ::=& \top \mid atomo \mid \phi \wedge \phi \mid \neg \phi \mid \mathbb{P}_{\bowtie p} [\psi] \mid \mathbb{S}_{\bowtie p} [\phi]\label{eq:state}\\
		\psi ::=& \mathcal{X} \phi \mid \phi \mathcal{U} \phi \mid \phi \mathcal{U}^{\leq t} \phi\label{eq:path},
	\end{align}
	dove la produzione \ref{eq:state} viene espansa in una \emph{formula di stato} e \ref{eq:path} in una formula di percorso.
	L'insieme di formule ottenuto \`e \emph{funzionalmente completo}, \`e quindi possibile ottenere tutti gli altri operatori a partire da quelli gi\`a definiti (ad esempio $\phi_1 \vee \phi_2 = \neg(\neg \phi_1 \wedge \neg \phi_2)$, oppure $\mathcal{F}(\phi) = \top \mathcal{U} \phi$).
	Sebbene valga l'uguaglianza $\phi_1 \mathcal{U} \phi_2 = \phi_1 \mathcal{U}^{< \infty} \phi_2$, vengono indicate entrambe le varianti dell'operatore, essendo, all'atto pratico, conveniente utilizzare due algoritmi di model checking diversi.
	
	La semantica associata alle formule di stato \`e la seguente:
	\begin{align}
		s \models \top &\quad\forall s \in \mathbb{S}\label{eq:true}\\
		s \models atomo &\iff atomo \in \mathcal{L}(s)\\
		s \models \phi_1 \wedge \phi_2 &\iff s \models \phi_1 \wedge s \models \phi_2\\
		s \models \neg \phi &\iff s \not \models \phi\label{eq:not}\\
		s \models \mathbb{P}_{\bowtie p} [\psi] &\iff Prob_s(\{\omega \in Path_s \mid \omega \models \psi\}) \bowtie p\label{eq:prob}\\
		s \models \mathbb{S}_{\bowtie p} [\phi] &\iff \underset{s' \models \phi}{\sum} \pi_s(s') \bowtie p\label{eq:stead},
	\end{align}
	e quella associata alle formule di percorso risulta:
	\begin{align}
		\omega \models \mathcal{X} \phi &\iff \exists \omega(1) \wedge \omega(1) \models \phi\label{eq:next}\\
		\omega \models \phi_1 \mathcal{U} \phi_2 &\iff \exists k \geq 0, \omega(k) \models \phi_2 \wedge \forall j < k, \omega(j) \models \phi_1\label{eq:unbounduntil}\\
		\omega \models \phi_1 \mathcal{U}^{\leq t} \phi_2 &\iff \exists x \in [0; t], \omega @ x \models \phi_2 \wedge \forall y \in [0; x), \omega @ y \models \phi_1\label{eq:bounduntil}.
	\end{align}
	
	Le regole \ref{eq:true}--\ref{eq:not} definiscono predicati e connettivi logici con il loro significato tradizionale.
	La regola \ref{eq:prob} asserisce che $\mathbb{P}_{\bowtie p} [\psi]$ \`e soddisfatto dallo stato $s$ se e solo se la probabilit\`a che tutti i percorsi $\omega$ che si originano da $s$ soddisfino $\psi$ \`e in relazione $\bowtie$ con $p$ (ad esempio $\leq 0.5$).
	La regola \ref{eq:stead} afferma, invece, che $\mathbb{S}_{\bowtie p} [\phi]$ \`e soddisfatto dallo stato $s$ se e solo se la somma delle probabilit\`a a regime delle transizioni da $s$ a qualsiasi stato $s'$ che soddisfi $\phi$ \`e $\bowtie p$.
	
	La regola \ref{eq:next} asserisce che $\mathcal{X} \phi$ \`e soddisfatto dal percorso $\omega$ se e solo se esiste uno stato successivo ($\omega(1)$) a quello corrente e tale stato soddisfa $\phi$.
	Le regole \ref{eq:unbounduntil} e \ref{eq:bounduntil} descrivono la semantica dell'operatore $\mathcal{U}$ nei casi illimitato e limitato.
	Nel primo, $\phi_1 \mathcal{U} \phi_2$ \`e soddisfatto da $\omega$ se e solo se esiste uno stato futuro $\omega(k)$ che soddisfa $\phi_2$ e tutti gli stati precedenti a $\omega(k)$ soddisfano $\phi_1$.
	Nel secondo caso, $\phi_1 \mathcal{U}^{\leq t} \phi_2$ \`e soddisfatto da $\omega$ se e solo se esiste un tempo $x$ minore o uguale a $t$ in cui lo stato occupato al tempo $x$ soddisfa $\phi_2$ e qualsiasi stato occupato in un tempo minore di $x$ soddisfa $\phi_1$.
	
	Il tempo richiesto per la terminazione degli algoritmi di model checking \`e lineare rispetto alla lunghezza della formula e polinomiale rispetto al numero di stati del modello. Pi\`u in dettaglio, la verifica della formula $\mathbb{P}_{\bowtie p} [\phi_1 \mathcal{U} \phi_2]$ ha complessit\`a asintotica $O(|\mathbb{S}|^3 + |\phi|)$, mentre la verifica di $\mathbb{P}_{\bowtie p} [\phi_1 \mathcal{U}^{\leq t} \phi_2]$ ha complessit\`a $O(|\mathbb{S}|^2 + q \cdot t + |\phi|)$ (con $q$ fattore di uniformizzazione utilizzato per il metodo di Jensen)~\cite{ctmc}.

	In PRISM, la verifica degli operatori $\mathbb{P}$, $\mathbb{R}[\mathcal{I} = t]$ e $\mathbb{R}[\mathcal{C} \leq t]$ avviene grazie al \emph{metodo di Jensen} (o algoritmo di uniformizzazione), mentre per gli operatori $\mathbb{S}$, $\mathbb{R}[\mathcal{F}  \phi]$ e $\mathbb{R}[\mathcal{S}]$ si utilizzano metodi basati sulla risoluzione di sistemi di equazioni lineari.
	
	L'algoritmo di uniformizzazione fissa un fattore $q$ pari al rate della transizione pi\`u veloce della CTMC e, scalando tutte le transizioni del fattore $q$, ottiene una DTMC che ne approssima il comportamento e su cui \`e possibile calcolare il vettore delle probabilit\`a transitorie dello stato $s$ dopo un tempo $t$ tramite:
	\begin{equation*}
		\underline{\pi}_{s,t} = \underline{\pi}_{s, 0} \cdot \sum\limits_{i = 0}^{\infty} \mathcal{P}_{q\cdot t}(i) \cdot \mathbf{P}^i,
	\end{equation*}
	dove $\underline{\pi}_{s, 0}$ \`e il vettore delle probabilit\`a transitorie iniziali, $\mathcal{P}_\lambda(n) = \frac{\lambda^n e^{-\lambda}}{n!}$ \`e la distribuzione di probabilit\`a di Poisson e $\mathbf{P}$ \`e la matrice delle probabilit\`a di transizione ($\mathbf{R}$ scalato del fattore $q$).
	Metodi numerici, come quello di \emph{Fox-Glynn}, approssimano il calcolo fissando un errore $\varepsilon$ e troncando la sommatoria quando l'errore sulla probabilit\`a stimata \`e inferiore a $\varepsilon$.
	
	A partire dalla DTMC uniformizzata, sono utilizzati i tre algoritmi di model checking:
	\begin{labeling}{$\mathbb{P}{[\phi_1 \mathcal{U}^{\leq t} \phi_2]}$}
		\item [$\mathbb{P}{[\mathcal{X}\phi]}$] si verifica banalmente calcolando la probabilit\`a che $\phi$ sia vero nei successori dello stato iniziale;
		\item [$\mathbb{P}{[\phi_1 \mathcal{U} \phi_2]}$] eliminati gli stati che soddisfano la propriet\`a CTL $\forall[\phi_1 \mathcal{U} \phi_2]$ e quelli che non soddisfano $\exists[\phi_1 \mathcal{U} \phi_2]$, si risolve il \emph{problema del percorso stocastico pi\`u breve} sugli stati rimanenti~\cite{probabilistic};
		\item [$\mathbb{P}{[\phi_1 \mathcal{U}^{\leq t} \phi_2]}$] eliminando dalla DTMC uniformizzata tutte le transizioni che escono dagli stati che soddisfano $\phi_2$ e da tutti gli stati che non soddisfano la formula CTL $\exists [\phi_1 \mathcal{U} \phi_2]$, si calcola la probabilit\`a di trovarsi in uno stato in cui valga $\phi_2$ al tempo $t$~\cite{parker2003implementation}.
	\end{labeling}

	\subsection{Model checking simbolico}\label{sez:mtbdd}
	Poich\'e la classe di complessit\`a degli algoritmi di model checking \`e \emph{polinomiale} rispetto allo spazio degli stati (ma questi crescono esponenzialmente al crescere del numero di componenti del sistema), \`e spesso opportuno effettuare il model checking utilizzando rappresentazioni simboliche della CTMC note come \emph{multi terminal binary decision diagram} (MTBDD).
	
	Una funzione logica \`e definita come $f(x_1, x_2, \dots, x_n): \mathbb{B}^n \mapsto \mathbb{B}$ e ad essa pu\`o essere associato un \emph{albero di decisione} che la descrive graficamente.
	Ogni nodo dell'albero rappresenta una variabile e possiede due figli: uno raggiunto se la variabile \`e vera, l'altro se la variabile \`e falsa; le foglie costituiscono il valore della funzione.
	
	In tabella \ref{tab:logica} \`e esemplificata la tabella di verit\`a della funzione logica $(x \vee y) \wedge z$ e in figura \ref{fig:albero} il corrispondente albero di decisione (con il ramo vero rappresentato da una linea continua e quello falso da una linea tratteggiata).
	Come si pu\`o notare, per una funzione di $n$ variabili, l'albero di decisione ha $2^{n+1} - 1$ nodi.
	
	Un \emph{binary decision diagram} (BDD)~\cite{bryant1986graph} \`e un grafo diretto aciclico che collassa insieme pi\`u nodi dell'albero di decisione, ottenendo una rappresentazione compatta e canonica (due BDD della stessa funzione booleana sono sempre isomorfi).
	In figura \ref{fig:bdd} \`e rappresentato il BDD relativo all'albero di decisione di figura \ref{fig:albero}.
	
	\begin{table}
		\centering
		\begin{tabular}{| c | c | c | c |}
			\hline
			$x$ & $y$ & $z$ & $(x \vee y) \wedge z$ \\
			\hline
			0 & 0 & 0 & 0 \\
			\hline
			0 & 0 & 1 & 0 \\
			\hline
			0 & 1 & 0 & 0 \\
			\hline
			0 & 1 & 1 & 1 \\
			\hline
			1 & 0 & 0 & 0 \\
			\hline
			1 & 0 & 1 & 1 \\
			\hline
			1 & 1 & 0 & 0 \\
			\hline
			1 & 1 & 1 & 1 \\
			\hline
		\end{tabular}
		\caption{Tabella di verit\`a della funzione logica $(x \vee y) \wedge z$}
		\label{tab:logica}
	\end{table}
	
	\begin{figure}
		\center
		\begin{tikzpicture}[->, auto, on grid, semithick, state/.style={circle, solid, draw, black, minimum width=0.75cm}, leaf/.style={rectangle, solid, draw, black, minimum size=0.75cm},level/.style={sibling distance = 6cm/#1,
			level distance = 1.5cm}]
		\node [state] {$x$}
			child[dashed]{ node [state] {$y$}
				child[dashed]{ node [state] {$z$} 
					child[dashed]{ node [leaf] {0}}
					child[solid]{ node [leaf] {0}}
				}
				child[solid]{ node [state] {$z$}
					child[dashed]{ node [leaf] {0}}
					child[solid]{ node [leaf] {1}}
				}
			}
		child{ node [state] {$y$}
			child[dashed]{ node [state] {$z$} 
				child[dashed]{ node [leaf] {0}}
				child[solid]{ node [leaf] {1}}
			}
			child[solid]{ node [state] {$z$}
				child[dashed]{ node [leaf] {0}}
				child[solid]{ node [leaf] {1}}
			}
		}
		; 
		
		\end{tikzpicture}
		\caption{Albero di decisione relativo alla funzione logica $(x \vee y) \wedge z$}
		\label{fig:albero}
	\end{figure}
	
	\begin{figure}
		\center
		\begin{tikzpicture}[->, auto, node distance=1.5cm and 1.5cm, on grid, semithick, state/.style={circle, solid, draw, black, minimum width=0.75cm}, leaf/.style={rectangle, solid, draw, black, minimum size=0.75cm}]
		\node[state] (x1) []{$x$};
		\node[state] (y1) [below left= of x1]{$y$};
		\node[state] (z1) [below right= of y1]{$z$};
		\node[leaf] (true1) [below= of z1]{1};
		\node[leaf] (false1) [left= of true1]{0};
		
		\path (x1) edge[dashed] (y1);
		\path (x1) edge (z1);
		\path (y1) edge (z1);
		\path (z1) edge (true1);
		\path (z1) edge[dashed] (false1);
		\path (y1) edge[dashed] (false1);
		\end{tikzpicture}
		\caption{Binary decision diagram relativo alla funzione logica $(x \vee y) \wedge z$}
		\label{fig:bdd}
	\end{figure}
	
	La costruzione di un BDD a partire dall'albero di decisione avviene attraverso le seguenti operazioni:
	\begin{enumerate}
		\item le foglie con lo stesso valore sono collassate;
		\item i nodi isomorfi sono collassati;
		\item i nodi ridondanti (dove sia il figlio vero che il figlio falso coincidono) sono eliminati.
	\end{enumerate}
	
	Nei BDD vale la \emph{legge di Shannon}, che permette di espandere ricorsivamente la funzione logica in due sotto-BDD:
	\begin{equation*}
		f_A = (\neg x \wedge f_{A|x=0}) \vee (x \wedge f_{A|x=1}),
	\end{equation*}
	dove $f_A$ \`e la funzione logica di partenza, $A|x=1$ \`e il sotto-BDD di $A$ costruito a partire dal figlio vero della variabile $x$ e, analogamente, $A|x=0$ rappresenta il sotto-BDD costruito a partire dal figlio falso.
	Tale espansione permette di combinare pi\`u BDD tramite \emph{operatori logici}, ottenendo una funzione logica (e quindi un BDD) pari alla combinazione delle funzioni di partenza.
	Per la logica dei predicati valgono le seguenti espansioni:
	\begin{align*}
		\neg f_A &= \neg x \wedge f_{A|x=1} \vee x \wedge f_{A|x=0}\\
		f_A \odot f_B &= \neg x \wedge (f_{A|x=0} \odot f_{B|x=0}) \vee x \wedge (f_{A|x=1} \odot f_{B|x=1})\\
		\exists x f_A &= f_{A|x=0} \vee f_{A|x=1}.
	\end{align*}
	
	\`E possibile espandere un BDD per renderlo in grado di rappresentare funzioni con codominio reale ($f(x_1, x_2, \dots, x_n): \mathbb{B}^n \mapsto \mathbb{R}$), semplicemente inserendo una foglia per ogni valore mappato del codominio. La struttura cos\`i estesa prende il nome di \emph{multi terminal binary decision diagram} (MTBDD).
	Le operazioni applicabili sugli MTBDD sono estese a tutti gli operatori \emph{algebrici}.
	
	Associando un identificatore intero ad ogni stato di una CTMC e utilizzando una funzione che ne rappresenta una codifica binaria di qualche tipo (ad esempio naturale, $101 \mapsto 5$, di Gray, $111 \mapsto 5$, o di altro tipo), \`e possibile rappresentare lo spazio degli stati con $n$ variabili booleane.
	Utilizzando tale codifica, la matrice delle transizioni $\mathbf{R} : \mathbb{S} \times \mathbb{S} \mapsto \mathbb{R}$ diventa una funzione booleana $\mathbf{R} : \mathbb{B}^n \times \mathbb{B}^n \mapsto \mathbb{R} = \mathbb{B}^{2n} \mapsto \mathbb{R}$, codificabile da un MTBDD~\cite{mtbdd}.

	I motori simbolici di PRISM, tuttavia, risparmiano memoria generando l'MTBDD direttamente a partire dal modello, senza passare dalla CTMC intermedia.
	L'MTBDD relativo all'intero modello \`e costruito come somma degli MTBDD dei singoli moduli e ognuno di essi codifica una funzione logica con dominio costituito da variabili di riga (stato di partenza), variabili di colonna (stato di destinazione) e variabili logiche che codificano le variabili PRISM locali al modulo.
	
	Dopo la costruzione dell'MTBDD, vengono eliminati tutti gli stati non raggiungibili a partire dallo stato iniziale.
	Tale operazione riduce la regolarit\`a dell'MTBDD, aumentandone purtroppo il numero di nodi, tuttavia ci\`o si riflette, generalmente, in una maggiore efficienza del model checking (che fa uso degli stessi algoritmi descritti nel caso tradizionale, con il vantaggio di un numero di stati inferiore)~\cite{parker2003implementation}.
	\chapter{Caso di studio: sviluppo di eritrociti ingegnerizzati per il trattamento della deficienza GAMT}\label{cap:casostudio}
In questo capitolo viene presentata come caso di studio una terapia in fase di sviluppo per la deficienza della guanidinoacetato metiltransferasi (sezione \ref{sez:defgamt}), basata su eritrociti ingegnerizzati per funzionare come bioreattore circolante nel flusso ematico.
Dopo aver introdotto nella sezione \ref{sez:defgamt} le caratteristiche del metabolismo della creatina e gli effetti della patologia su quest'ultimo e sull'organismo, viene descritto nella sezione \ref{sez:rcl} l'apparato in grado di incapsulare S-adenosilmetionina sintasi e guanidinoacetato metiltransferasi all'interno degli eritrociti.
Infine vengono descritti gli strumenti di analisi utilizzati (sezioni \ref{sez:hplc} e \ref{sez:spray}) e lo stato attuale della sperimentazione \emph{in vitro} del trattamento della deficienza GAMT (sezione \ref{sez:arte}).

	\section{Deficienza GAMT}\label{sez:defgamt}
	
		\subsection{Metabolismo della creatina}
		La creatina (Cr) \`e un composto azotato presente nei tessuti ad elevato dispendio energetico (muscoli e neuroni) con funzioni di tampone e intermedio energetico, in grado di accumulare e rilasciare energia tramite la fosforilazione in fosfocreatina (PCr), con contemporanea defosforilazione di adenosintrifosfato (ATP) in adenosindifosfato (ADP):
		\begin{equation*}
			Cr + ATP \rightleftharpoons PCr + ADP.
		\end{equation*}
		In condizioni fisiologiche, il pool di creatina dell'organismo \`e mantenuto a 6-\SI{50}{\micro mol/l} nel siero e 11-\SI{244}{mmol/l} nelle urine~\cite{o2009guanidinoacetate} da sintesi endogena, assunzione con la dieta e degradazione in creatinina (successivamente espulsa con le urine, insieme a una parte di creatina).
		
		La via biosintetica della creatina avviene in due fasi.
		La prima, nei reni, produce \emph{guanidinoacetato} e ornitina a partire da arginina e glicina tramite arginina-glicina amidinotransferasi.
		La seconda, nel fegato, trasferisce un gruppo metilico (usando \emph{S-adenosil metionina} come donatore) sul guanidinoacetato per produrre creatina tramite \emph{guanidinoacetato metiltransferasi}:
		\begin{equation}
			Arg + Gly \xrightleftharpoons{AGAT} Orn + GAA\text{, nei reni;}\label{eq:agat}
		\end{equation}
		\begin{equation}
			GAA + SAM \xrightarrow{GAMT} Cr + SAH\text{, nel fegato.}\label{eq:gamt}
		\end{equation}
		
		La S-adenosil metionina \`e prodotta dalle cellule a partire da metionina e ATP tramite \emph{SAM sintasi}:
		\begin{equation}
			Met + ATP \xrightarrow{SAMS} SAM + P_i + PP_i.
		\end{equation}
		
		Altre reazioni possono interferire con la sintesi della creatina solo tramite sequestro o produzione dei precursori aminoacidici (arginina, glicina od ornitina) o di S-adenosil metionina, in quanto non esiste, nei mammiferi, nessun'altra reazione che utilizzi guanidinoacetato.
		
		La produzione di creatinina avviene tramite idratazione della creatina o disidratazione della fosfocreatina:
		\begin{align*}
			Cr + H_2O &\rightarrow Creatinina\\
			PCr &\rightarrow P_i + H_2O + Creatinina.
		\end{align*}
		
		I metaboliti della via della creatina sono trasportati a livello sistemico dal sangue e dal liquido cefalorachidiano.
		Gli scambi tra i due fluidi sono mediati dalle barriere \emph{emato-encefalica} ed \emph{emato-liquorale}.
		
		A livello cellulare, gli scambi sono mediati dai seguenti trasportatori~\cite{tachikawa2011transport}:
		\begin{labeling}{Famiglia SLC6}
			\item [Famiglia SLC6] Solute Carrier 6, trasportatori attivi di guanidinocomposti (tra cui guanidinoacetato e creatina) $Na^+$ e $Cl^-$ dipendenti:
			\begin{itemize}
				\item Creatine Transporter 1 (CRT o SLC6A8) media l'ingresso di creatina e guanidinoacetato, sebbene il trasporto di quest'ultimo sia inibito dalla creatina (e quindi apprezzabile solo in pazienti GAMT deficienti),
				\item Taurine Transporter (TauT o SLC6A6) media il trasporto di taurina e l'efflusso di guanidinoacetato (con affinit\`a un ordine di grandezza pi\`u basso rispetto alla taurina e inibizione ad opera di quest'ultima);
			\end{itemize}
			\item [OCT3] Organic Cationic Transporter 3, media l'efflusso di creatinina.
		\end{labeling}
		
		Gli eritrociti (che non sintetizzano nativamente creatina, ma che possono essere infusi con GAMT e SAMS, come descritto in sezione \ref{sez:arte}) costituiscono un ambiente con un numero molto limitato di interazioni, in quanto:
		\begin{itemize}
			\item non andando incontro a traduzione, non alterano il proprio pool aminoacidico in maniera significativa;
			\item pur possedendo SAMS endogena, non influiscono sensibilmente sull'aumento di SAM (l'attivit\`a dell'enzima risulta cinque ordini di grandezza pi\`u bassa rispetto alla variante espressa nel fegato);
			\item possiedono un solo enzima (protein glutammato O-metil transferasi) che consuma S-adenosil metionina, secondo la reazione:
			\begin{equation}
				\text{protein }L-Glu + SAM \xrightleftharpoons{PGOMT} \text{metil estere protein }L-Glu + SAH;
			\end{equation}
			\item il substrato della protein glutammato O-metil transferasi \`e costituito dall'L-glutammato di proteine di membrana, modificato solo in maturazione, quindi l'enzima non opera negli eritrociti maturi.
		\end{itemize}
		
		Il trasporto di creatina mostra una cinetica bifasica che rallenta con l'et\`a dei globuli rossi e risulta inibita da acido guanidinopropionico, indicando la presenza di SLC6A8 e di una diffusione passiva a bassa affinit\`a, mentre quello di creatinina presenta una cinetica tradizionale~\cite{ku1980creatine}.
		Anche il trasporto di S-adenosil metionina \`e caratterizzato da una cinetica bifasica (la cinetica lenta sembra associata al metabolismo interno, pi\`u che a un trasporto vero e proprio, quella veloce suggerisce la presenza di due siti ad affinit\`a diversa)~\cite{pezzoli1978uptake,stramentinoli1978uptake}.
		
		\subsection{Deficienza dell'enzima guanidinoacetato metiltransferasi}
		Un malfunzionamento della guanidinoacetato metiltransferasi pu\`o causare rallentamento o blocco completo della reazione \ref{eq:gamt}: tale malfunzionamento causa livelli bassi di creatina e accumulo di guanidinoacetato, non essendo utilizzato da nessun'altra reazione nell'intero organismo.
		
		I bassi livelli di creatina causano deficit di tipo cognitivo e di sviluppo, mentre il guanidinoacetato, ad alte dosi, causa formazione di ossido nitrico e alterazione della membrana sinaptica (interferendo con le attivit\`a di ATPasi e ioni $Na^+$ e $K^+$) per antagonismo con il recettore GABA$_A$, risultando dunque neurotossico~\cite{gordon2010guanidinoacetate}.
		
		A causa della funzione di tampone energetico della creatina, sono presenti adattamenti metabolici (ma non istologici) secondari, atti a compensarne (tramite sovrapproduzione di ATP) la mancanza in pazienti GAMT deficienti.
		In particolare, nei tessuti ad alto dispendio energetico, si osserva una sovraespressione degli enzimi coinvolti nella catena respiratoria (le quantit\`a dei Complessi I, III e V, espressi nei mitocondri di fibroblasti incubati in assenza di creatina, si sono rivelate doppie rispetto alle colture di controllo), con conseguente aumento dell'attivit\`a mitocondriale~\cite{das2000upregulation}.
		
		Un elettroencefalogramma rileva convulsioni e iperattivit\`a di globo pallido, campo H di Forel, sostanza nera e corteccia, dopo iniezione intracisternale di guanidinocomposti in animali da laboratorio; ad eccezione del guanidinoacetato, gli altri guanidinocomposti causano inoltre formazione di radicali~\cite{mori1987biochemistry}.
		
		\`E stata dimostrata la natura genetica (autosomica recessiva) della deficienza GAMT, transfettando fibroblasti GAMT deficienti con vettori codificanti GAMT nativa sovraespressa e osservando un ripristino del normale metabolismo~\cite{almeida2006overexpression}.
		Topi knockout per il gene gamt presentano maggiore mortalit\`a neonatale, ipotonia muscolare, fertilit\`a maschile ridotta, perdita di peso non mediata da leptina e adattamenti metabolici analoghi a quelli riscontrati nei pazienti umani, rendendoli un modello biologico adeguato per lo studio della malattia~\cite{schmidt2004severely}.
		
		\subsection{Sintomi}
		La deficienza GAMT \`e caratterizzata da ritardo comportamentale e disabilit\`a intellettuale, principalmente relativa ai domini del linguaggio e del comportamento, da attribuirsi alla carenza di creatina, e da epilessia e disturbi extrapiramidali (alterazione dei movimenti ``espressivi'' e di quelli involontari, come tremori, rallentamenti o spasmi), causati dall'accumulo di guanidinoacetato nel tessuto nervoso.
		
		Nonostante il supporto scolastico, pazienti gravemente deficienti presentano funzionalit\`a intellettuali estremamente limitate e mancano dell'indipendenza in et\`a adulta; pazienti con sintomatologia lieve presentano invece funzionalit\`a intellettuali sufficienti a raggiungere un certo livello d'indipendenza da adulti.
		La severit\`a degli episodi epilettici \`e correlata alla gravit\`a della deficienza e i disturbi di movimento non sono osservati in pazienti che non presentano anche fenomeni epilettici~\cite{stockler2014guanidinoacetate}.
		
		In~\cite{araujo2005guanidinoacetate,ganesan1997guanidinoacetate,mikati2013epileptic,o2009guanidinoacetate,vodopiutz2007severe} sono presentati alcuni casi clinici.
		
		\subsection{Diagnosi}
		Poich\'e i sintomi risultano aspecifici e gli effetti secondari sono analoghi all'encefalopatia mitocondriale~\cite{gordon2010guanidinoacetate}, risulta necessario procedere ad esami diagnostici specifici (e ancora poco diffusi):
		\begin{itemize}
			\item sequenziamento del gene gamt;
			\item tomografia a risonanza magnetica, che rileva un aumento del segnale relativo al globo pallido;
			\item spettroscopia di risonanza magnetica $^1H$ su urine e liquor, che rileva basse concentrazioni di creatina (assenza del picco singoletto a \SI{3.05}{ppm}) e creatinina (assenza del picco singoletto a \SI{3.13}{ppm}), e alte concentrazioni di guanidinoacetato (picco doppietto a \SI{3.98}{ppm} nelle urine, non distinguibile nel liquor per la presenza di molecole simili)~\cite{engelke2009guanidinoacetate};
			\item spettroscopia di risonanza magnetica $^{31}P$, che rileva basse concentrazioni di fosfocreatina in cervello e muscoli, e alte concentrazioni di fosfoguanidinoacetato~\cite{renema2003mr};
			\item reazione di Sakaguchi su campioni di urina: dopo una cromatografia a strato sottile, le piastre cromatografiche vengono prima immerse in una soluzione allo \SI{0.1}{\percent} di idrossichinolina in acetone e poi spruzzate con bromo allo \SI{0.3}{\percent} in soluzione \SI{0.5}{M} di $NaOH$, il test \`e positivo (ma la diagnosi va confermata con altri metodi) se si formano macchie rosso-arancio, a causa della reazione tra bromo e guanidine monosostituite: tra queste l'unica che pu\`o essere presente nelle urine \`e il guanidinoacetato~\cite{schulze1996sakaguchi}.
		\end{itemize}
		
		\subsection{Terapie}
		Essendo la deficienza GAMT un difetto genetico, il trattamento pu\`o avvenire soltanto per integrazione di creatina ed eliminazione di guanidinoacetato.
		
		Poich\'e alte dosi di creatina (\SI{1}{g/kg/die} nei ratti) riducono l'espressione di SLC6A8~\cite{tarnopolsky2003acute}, non \`e possibile eccedere con dosaggio o durata del trattamento.
		L'integrazione di creatina monoidrato segue i dosaggi previsti per i difetti del ciclo dell'urea (300-\SI{800}{mg/kg/die}) e pu\`o essere associata o meno a una dieta che riduca la formazione di guanidinoacetato (riducendo la disponibilit\`a di arginina, per inibire la reazione \ref{eq:agat}) secondo i seguenti regimi~\cite{stockler2014guanidinoacetate}:
		\begin{itemize}
			\item dieta ipoproteica;
			\item dieta ipoproteica (0.6-\SI{1.8}{mg/kg/die}) e formula di aminoacidi non contenente arginina;
			\item dieta fortemente ipoproteica (0.2-\SI{0.5}{mg/kg/die}) con restrizione di arginina a \SI{250}{mg/kg/die} e formula di aminoacidi non contenente arginina.
		\end{itemize}
		
		Ognuno dei regimi pu\`o essere ulteriormente integrato dall'aggiunta di ornitina (\SI{100}{mg/kg/die}), atta a spostare l'equilibrio della reazione \ref{eq:agat} verso i reagenti.
		La restrizione dell'arginina non interferisce con le vie di detossificazione dei composti azotati, ma causa una riduzione del guanidinoacetato, associata a una scomparsa quasi completa di convulsioni e attivit\`a epileptogeniche~\cite{schulze2001improving}.
		Effettuando una risonanza magnetica $^1H$ durante la terapia, si osserva un recupero delle concentrazioni di creatina pi\`u rapido nel tessuto muscolare rispetto a quello nervoso~\cite{ensenauer2004guanidinoacetate}.
		
		Oltre a una dieta carente di arginina, \`e possibile inibire la reazione \ref{eq:agat} somministrando benzoato di sodio (\SI{100}{mg/kg/die}), che sequestra la glicina formando acido ippurico (escreto rapidamente dai reni).
		
		Un trattamento precoce (prenatale o nei primi mesi di vita) \`e in grado di prevenire o ridurre notevolmente i difetti di sviluppo, pertanto \`e importante uno screening tempestivo, prima dell'insorgenza dei sintomi~\cite{stockler2014guanidinoacetate}.
	
	\section{Red Cell Loader}\label{sez:rcl}
		Red Cell Loader (RCL)~\cite{magnani1998erythrocyte} \`e un apparato, sviluppato dall'Universit\`a di Urbino, in grado di incapsulare sostanze non diffusibili all'interno dei globuli rossi tramite processi osmotici reversibili.
		Il sangue cos\`i arricchito costituisce una nuova via di somministrazione a rilascio prolungato di farmaci o mezzi di contrasto, e la possibilit\`a di creare un bioreattore in grado di circolare \emph{in vivo}, sequestrando dal circolo ematico substrati (ad esempio sostanze tossiche o farmaci inattivi) e rilasciando prodotti (ad esempio farmaci attivati)~\cite{biagiotti2011drug,magnani2012erythrocytes,rossi2005erythrocyte}.
		
		RCL presenta vantaggi rispetto a tecniche di incapsulamento preesistenti basate su dialisi ad alto ematocrito~\cite{hamarat2000encapsulation}, quali apirogenicit\`a, sterilit\`a ed emocompatibilit\`a (che lo rendono adatto per utilizzi \emph{in vivo}), la ridotta quantit\`a di sangue necessario (\SI{50}{ml} contro un'unit\`a di sangue, \SI{450}{ml}), la velocit\`a della procedura (circa due ore) e l'alta resa (sopravvivenza del 35-\SI{50}{\percent} dei globuli rossi e incapsulamento del \SI{30}{\percent} di sostanza).
		Lo stesso apparato permette inoltre di ingegnerizzare l'``invecchiamento'' degli eritrociti, modulando quindi il rate di fagocitosi (e conseguentemente il rate di rilascio della sostanza) ad opera dei macrofagi.
		
		\begin{figure}
			\centerline{\resizebox{.9\linewidth}{!}{\includegraphics{figure/RBC_Loader.png}}}
			\caption{Red Cell Loader}
			\label{fig:rcl}
		\end{figure}
		
		Gli eritrociti caricati dal RCL sono gi\`a stati sperimentati nei seguenti ambiti clinici:
		\begin{itemize}
			\item rilascio prolungato di azidotimidina ed etambutolo per il trattamento di infezioni disseminate di \emph{Mycobacterium avium} in pazienti con AIDS allo stadio avanzato~\cite{magnani1996synthesis,rossi1999heterodimer};
			\item rilascio di competitori di nucleosidi fosforilati all'interno di monociti e macrofagi infetti da HIV~\cite{magnani1992targeting};
			\item rilascio prolungato e localizzato di dexametasone in pazienti con broncopneumopatia cronica ostruttiva~\cite{rossi2001erythrocyte};
			\item incapsulamento di mezzi di contrasto per risonanza magnetica per aumentarne l'emivita (in forma libera sono eliminati dai reni in poche ore)~\cite{antonelli2011encapsulation}.
		\end{itemize}
		
	\section{Cromatografia liquida ad alte prestazioni}\label{sez:hplc}
	La cromatografia liquida ad alte prestazioni (HPLC) \`e una tecnica di separazione e analisi delle componenti di un liquido.
	Come per la cromatografia tradizionale, la separazione avviene tramite il passaggio della \emph{fase mobile} (da analizzare) attraverso una colonna analitica al cui interno \`e contenuta la \emph{fase stazionaria}, costituita da particelle sferiche in grado di interagire con la fase mobile trattenendola (ad esempio perch\'e porose o cariche elettrostaticamente).
	Componenti diverse della fase mobile avranno differente affinit\`a per la fase stazionaria e conseguentemente la attraverseranno a velocit\`a diverse.
	Ponendo uno spettrometro all'uscita della colonna, \`e possibile analizzare le singole componenti man mano che ne escono.
	
	A differenza della tecnica tradizionale, l'HPLC utilizza alte pressioni (anzich\'e la semplice forza di gravit\`a) per spingere la fase mobile attraverso la colonna analitica, diminuendo drasticamente il tempo di analisi e aumentandone la precisione.
	L'utilizzo di una pompa garantisce un flusso costante e le alte pressioni permettono l'utilizzo di colonne pi\`u sottili che impediscono alla fase mobile di diffondersi trasversalmente, alterando il tempo di uscita dalla colonna.
		
	\section{Spettrometria di massa a ionizzazione per elettrospray}\label{sez:spray}
	L'elettrospray \`e una tecnica di ionizzazione di macromolecole che permette la desolvatazione (la separazione del soluto dal solvente) senza il rischio di frammentazione delle molecole.
	L'applicazione di alta tensione causa il passaggio dalla fase liquida alla dispersione in aerosol per repulsione elettrostatica.
	Il ridotto volume delle microgocce causa un brusco aumento della densit\`a di carica, che si riflette su un ulteriore aumento delle forze repulsive, le quali a loro volta innescano una separazione in microgocce pi\`u piccole.
	Il fenomeno continua finch\'e le singole molecole ionizzate non sono espulse dalla goccia (venendo effettivamente separate dal solvente) e disperse in fase gassosa in una camera a vuoto.

	La procedura specifica per la misura di guanidinoacetato e creatina da gocce di sangue essiccate su cartone parte dall'estrazione dal cartone con una soluzione di acqua, metanolo e N-metil-D$_3$-creatina (creatina marcata con deuterio) e viene effettuata risospendendo in acqua, metanolo e acido acetico, dopo l'evaporazione del primo solvente.
	L'analisi impiega circa un minuto e presenta range di misura lineari sia per guanidinoacetato che per creatina, con una precisione tale da rendere l'elettrospray una tecnica diagnostica promettente per le deficienze GAMT e AGAT~\cite{carducci2006quantitative}.
	
	\section{Stato dell'arte del trattamento della deficienza GAMT}\label{sez:arte}
	\`E in fase di sperimentazione l'utilizzo del red cell loader per incapsulare negli eritrociti GAMT nativa dei pazienti, che sopperisca alla mancanza della GAMT difettosa.
	Gli esperimenti programmati appartengono ai seguenti gruppi:
	\begin{enumerate}
		\item misura dell'uptake di guanidinoacetato ad opera delle membrane degli eritrociti;
		\item incapsulamento di GAMT negli eritrociti e misura della sintesi di creatina;
		\item incapsulamento di GAMT e SAMS negli eritrociti e misura della sintesi di creatina.
	\end{enumerate}
	
	Allo stato attuale, la sperimentazione \emph{in vitro} sta procedendo con la seconda fase.
	Le simulazioni \emph{in silico} (capitolo \ref{cap:simulazione}) seguono lo stesso ordine e sono state invece portate a termine.

	Il workflow generale per ogni esperimento procede nel seguente modo:
	\begin{labeling}{analisi elettrospray}
		\item [incapsulamento] gli eritrociti sono separati dal sangue di un donatore e incapsulati tramite RCL con enzimi ed eventualmente substrati;
		\item [analisi HPLC] un'aliquota di eritrociti ingegnerizzati viene lisata e analizzata tramite HPLC per verificare le quantit\`a effettivamente incapsulate;
		\item [incubazione] vengono aggiunti i metaboliti esterni (tra cui guanidinoacetato marcato con $^{13}C$, HEPES come tampone a pH fisiologico e PIGPA per il ringiovanimento degli eritrociti~\cite{usry1975morphology}) e la soluzione viene incubata per un tempo prestabilito;
		\item [analisi elettrospray] ogni reazione \`e fermata per essiccazione su cartone e le gocce di sangue vengono analizzate per elettrospray, quantificando il guanidinoacetato e la creatina marcati con $^{13}C$.
	\end{labeling}
	
	\subsection{Esperimento sull'uptake del guanidinoacetato}\label{sez:uptake}
	Poich\'e gli eritrociti ingegnerizzati tramite RCL sostituiscono il fegato relativamente alla reazione \ref{eq:gamt}, \`e necessario caratterizzare la cinetica di trasporto del guanidinoacetato tra plasma e interno degli eritrociti:
	\begin{equation*}
	GAA_{fuori} \xrightleftharpoons{trasporto} GAA_{dentro}.
	\end{equation*}
	
	Per verificare che la membrana degli eritrociti sia effettivamente permeabile al guanidinoacetato e per osservarne la cinetica, sono state incubate a \SI{37}{\celsius} quantit\`a costanti di eritrociti (al 15\% di ematocrito) con quantit\`a variabili di guanidinoacetato $^{13} C$.
	
	A intervalli di tempo prefissati (0, 5, 15, 30 e 60 minuti) sono state prelevate aliquote da cui sono stati separati gli eritrociti per centrifugazione a 9600 g per 3 minuti in presenza di 1-bromo-dodecano.
	Al termine della centrifugazione sono stati ottenuti pellet al 50\% circa di ematocrito, che sono stati essiccati su cartone e analizzati tramite elettrospray.
	
	La figura \ref{fig:gaauptake} riassume i dati ottenuti in un grafico della concentrazione del guanidinoacetato intracellulare nel tempo, partendo da diverse concentrazioni di guanidinoacetato extracellulare.
	L'andamento dei grafici \`e coerente con una diffusione passiva del guanidinoacetato attraverso la membrana.
	
			\begin{figure}
				\center
				\begin{tikzpicture}
				\begin{axis}[axis lines=middle, xmin=0, xmax=100, ymin=0, ymax=100,samples=100, xtick={5, 15, 30, 60}, xlabel={$t$}, ylabel={$[GAA^*]$}]
				\addplot[
				scatter,
				point meta=explicit symbolic,
				scatter/classes={
					a={mark=o,draw=black},
					b={mark=x, draw=black},
					c={mark=square,draw=black},
					d={mark=triangle,draw=black},
					e={mark=asterisk,draw=black}}
				]
				table [meta=label]{
					x y label err
					0 0 a 1
					5 0 a 2
					15 0 a 3
					30 0 a 4
					60 0 a 5
				};
				\addplot[
				scatter,
				point meta=explicit symbolic,
				scatter/classes={
					a={mark=o,draw=black},
					b={mark=x, draw=black},
					c={mark=square,draw=black},
					d={mark=triangle,draw=black},
					e={mark=asterisk,draw=black}}
				]
				table [meta=label]{
					x y label
					0 0 b
					5 1 b
					15 3 b
					30 5 b
					60 7 b
				};
				\addplot[
				scatter,
				point meta=explicit symbolic,
				scatter/classes={
					a={mark=o,draw=black},
					b={mark=x, draw=black},
					c={mark=square,draw=black},
					d={mark=triangle,draw=black},
					e={mark=asterisk,draw=black}}
				]
				table [meta=label]{
					x y label
					0 2 c
					5 3 c
					15 7 c
					30 12 c
					60 17 c
				};
				\addplot[
				scatter,
				point meta=explicit symbolic,
				scatter/classes={
					a={mark=o,draw=black},
					b={mark=x, draw=black},
					c={mark=square,draw=black},
					d={mark=triangle,draw=black},
					e={mark=asterisk,draw=black}}
				]
				table [meta=label]{
					x y label
					0 3 d
					5 7 d
					15 14 d
					30 25 d
					60 38 d
				};
				\addplot[
				scatter,
				point meta=explicit symbolic,
				scatter/classes={
					a={mark=o,draw=black},
					b={mark=x, draw=black},
					c={mark=square,draw=black},
					d={mark=triangle,draw=black},
					e={mark=asterisk,draw=black}}
				]
				table [meta=label]{
					x y label
					0 5 e
					5 11 e
					15 24 e
					30 41 e
					60 67 e
				};
				
				
				
				\legend{$0\mu M$,$10\mu M$,$25\mu M$,$50\mu M$,$100\mu M$};
				
				\end{axis}
				\end{tikzpicture}
				\caption{Rate di ingresso del guanidinoacetato negli eritrociti a diverse concentrazioni}
				\label{fig:gaauptake}
			\end{figure}
	
	Per meglio caratterizzare la cinetica, \`e stato realizzato il diagramma dei doppi reciproci in figura \ref{fig:gaalineweaver}, ponendo in ascissa l'inverso delle concentrazioni e in ordinata l'inverso delle velocit\`a, calcolate come pendenza della retta di regressione lineare ottenuta sui punti di ognuna delle curve, ad eccezione della curva corrispondente a $0\mu M$ di guanidinoacetato esterno.
	\`E stata effettuata un'ulteriore regressione lineare sui punti per ottenere la retta di regressione:
	\begin{equation*}
		y = 48.9333 x + 0.0853333, \qquad(R^2 = 0.998832),
	\end{equation*}
	da cui \`e stata calcolata una costante $v\_uptake = 0.02 \mu M/min$.
		
		\begin{figure}
			\center
			\begin{tikzpicture}
			\begin{axis}[axis lines=middle, xmin=-0.01, xmax=0.1, ymin=-0.5, ymax=5,domain=-0.1:0.5,samples=100, xlabel={$\frac{1}{[S]}$}, ylabel={$\frac{1}{V}$}]
			\addplot[black]{48.9333*x + 0.0853333};
			\addplot[
			scatter,
			only marks,
			point meta=explicit symbolic,
			scatter/classes={
				a={mark=o,draw=black}}
			]
			table [meta=label]{
				x y label
				0.1 5 a
				0.04 2 a
				0.02 1 a
				0.01 0.66 a
			};
			
			\end{axis}
			\end{tikzpicture}
			\caption{Diagramma dei doppi reciproci dell'ingresso di guanidinoacetato negli eritrociti}
			\label{fig:gaalineweaver}
		\end{figure}
	
	\subsection{Esperimenti sul dosaggio della creatina prodotta}\label{sez:dosaggio}
		In un paziente GAMT deficiente, l'elevata disponibilit\`a di guanidinoacetato nel plasma si riflette in una cinetica potenzialmente elevata della GAMT incapsulata negli eritrociti.
		Oltre all'uptake di guanidinoacetato, l'altro elemento limitante pu\`o essere costituito dall'S-adenosil metionina, che rischia di non essere prodotta a velocit\`a sufficientemente elevata dalla SAMS eritrocitaria.
		
		Per verificare se sia necessario incapsulare, oltre alla GAMT, SAMS clonata da \emph{E.\ coli}, si sono effettuati esperimenti di dosaggio della creatina prodotta, a partire da quantit\`a preformate di S-adenosil metionina.
		
		Il sangue \`e stato centrifugato fino a ottenere tre aliquote di 780 $\mu l$ al \SI{90}{\percent} di ematocrito.
		Una delle aliquote \`e stata addizionata a 220 $\mu l$ di HEPES e funge da controllo, le altre due sono state addizionate ciascuna a 220 $\mu l$ di soluzione contenente 660 $\mu g$ di GAMT.
		Da ognuna sono stati prelevati \SI{50}{\micro l} e alle aliquote contenenti GAMT sono state aggiunte rispettivamente 2.5 $\mu l$ e 10 $\mu l$ di soluzione 5 $nM$ di S-adenosil metionina.
		Le due aliquote sono state processate tramite RCL per incapsulare GAMT e S-adenosil metionina, e, insieme all'aliquota controllo, sono state diluite fino a ottenere un ematocrito al \SI{50}{\percent}.
		
		Dalle tre soluzioni sono stati prelevati 250 $\mu l$ ed \`e stata verificata tramite HPLC la quantit\`a di S-adenosil metionina effettivamente incapsulata, ottenendo una concentrazione di circa 25 $\mu M$ per l'aliquota trattata con 2.5 $\mu l$ e 125 $\mu M$ per l'aliquota trattata con 10 $\mu l$.
		
		Sono stati aggiunti 7 $\mu l$ di soluzione contenente guanidinoacetato $^{13}C$ a concentrazione 5 $nM$ e si sono incubate le soluzioni a \SI{37}{\celsius}.
		Ai tempi 0, 1, 2, 3 e 20 ore sono stati prelevati 50 $\mu l$ di soluzione ed essiccati su cartoncino per bloccare le reazioni.
		
		L'analisi elettrospray ha prodotto le seguenti tabelle:
	\begin{table}[H]
		\centering
		\begin{tabular}{| c | c | c |}
			\hline
			Tempo (ore) & $[GAA]\ (\mu M)$ & $[Cr]\ (\mu M)$ \\
			\hline
			0 & 0.13 & 59.80\\
			\hline
			1 & 0.15 & 72.20\\
			\hline
			2 & 0.16 & 75.25\\
			\hline
			3 & 0.20 & 75.86\\
			\hline
			20 & 0.27 & 76.31\\
			\hline
		\end{tabular}
		\caption{Concentrazioni di guanidinoacetato e creatina in eritrociti non ingegnerizzati.}
		\label{tab:unloaded}
	\end{table}
	
	\begin{table}[H]
		\centering
		\begin{tabular}{| c | c | c |}
			\hline
			Tempo (ore) & $[GAA]\ (\mu M)$ & $[Cr]\ (\mu M)$ \\
			\hline
			0 & 2.61 & 66.01\\
			\hline
			1 & 3.10 & 61.90\\
			\hline
			2 & 4.62 & 63.37\\
			\hline
			3 & 5.11 & 67.50\\
			\hline
			20 & 5.26 & 59.31\\
			\hline
		\end{tabular}
		\caption{Concentrazioni di guanidinoacetato e creatina in eritrociti ingegnerizzati con GAMT e 25 $\mu M$ di SAM.}
		\label{tab:25um}
	\end{table}
	
	\begin{table}[H]
		\centering
		\begin{tabular}{| c | c | c |}
			\hline
			Tempo (ore) & $[GAA]\ (\mu M)$ & $[Cr]\ (\mu M)$ \\
			\hline
			0 & 7.32 & 66.32\\
			\hline
			1 & 10.52 & 64.55\\
			\hline
			2 & 13.87 & 65.87\\
			\hline
			3 & 12.18 & 61.58\\
			\hline
			20 & 15.13 & 59.88\\
			\hline
		\end{tabular}
		\caption{Concentrazioni di guanidinoacetato e creatina in eritrociti ingegnerizzati con GAMT e 125 $\mu M$ di SAM.}
		\label{tab:125um}
	\end{table}

	\chapter{Costruzione del modello per il trattamento della deficienza GAMT}\label{cap:costruzione}
In questo capitolo viene presentata la fase di descrizione formale del caso di studio, riassumendo brevemente le caratteristiche specifiche del metabolismo (sezione \ref{sez:via}) e mostrando il modello Bio-PEPA (sezione \ref{sez:modpepa}) che lo descrive e i modelli PRISM generati a partire da quest'ultimo (sezione \ref{sez:modprism}).

\section{Via metabolica da modellare}\label{sez:via}
Riassumendo i dati raccolti nel capitolo \ref{cap:casostudio}, si parte dalla seguente via:
\begin{align}
		Arg + Gly &\xrightleftharpoons{AGAT} Orn + GAA\label{eq:agat2}\\
		Met + ATP &\xrightarrow{SAMS} SAM + P_i + PP_i\label{eq:sams2}\\
		GAA_{fuori} &\xrightleftharpoons{trasporto} GAA_{dentro}\label{eq:trasporto2}\\
		GAA_{dentro} + SAM &\xrightarrow{GAMT} Cr + SAH,\label{eq:gamt2}
\end{align}

Assumendo che la produzione di guanidinoacetato avvenga molto pi\`u lentamente della sua degradazione, si pu\`o astrarre dalla reazione \ref{eq:agat2}, considerando una quantit\`a iniziale di guanidinoacetato libera nel plasma che non viene rifornita.
Vista l'elevata efficacia dei meccanismi di regolazione, si pu\`o assumere che i metabolismi energetici e di biosintesi riescano a compensare ampiamente l'aumentato fabbisogno di metionina e ATP degli eritrociti, di conseguenza si possono assumere livelli costanti di questi ultimi nella reazione \ref{eq:sams2}.
L'equilibrio della reazione \ref{eq:trasporto2} risulta essere spostato verso i prodotti, a causa del sequestro immediato del guanidinoacetato intracellulare ad opera della reazione \ref{eq:gamt2}, di conseguenza \`e possibile semplificare la reazione rendendola unidirezionale (uptake).

Le cinetiche coinvolte nelle tre reazioni rimaste sono:
\begin{labeling}{trasporto}
	\item [SAMS] catalisi con formazione di complessi ternari ad ordine di associazione obbligata, inibita da S-adenosil metionina;
	\item [GAMT] catalisi con formazione di complessi ternari ad ordine di associazione obbligata, inibita da S-adenosil omocisteina;
	\item [trasporto] diffusione passiva attraverso la membrana, dal plasma, verso l'interno degli eritrociti (equazione \ref{eq:diffusione}).
\end{labeling}

Per limitare l'esplosione dello spazio degli stati del modello, \`e necessario eliminare tutto ci\`o che non \`e rilevante ai fini delle cinetiche (ad eccezione della creatina, di cui \`e richiesta la quantificazione), di conseguenza le reazioni sono semplificate come segue:
\begin{align}
	Met_{costante} + ATP_{costante} &\xrightarrow{SAMS} SAM\label{eq:sams3}\\
	GAA_{fuori} &\xrightarrow{uptake} GAA_{dentro}\label{eq:uptake3}\\
	GAA_{dentro} + SAM &\xrightarrow{GAMT} Cr + SAH\label{eq:gamt3}.
\end{align}

\subsection{Cinetiche di reazione}
Per modellare fedelmente il caso di studio, \`e necessario ricavare le velocit\`a a cui dovranno avvenire le azioni che modellano le reazioni precedentemente descritte.
Di seguito vengono ricavate le cinetiche utili alla modellazione del caso di studio.

\subsubsection{Catalisi enzimatica con formazione di complessi ternari}
Gli enzimi che catalizzano le reazioni \ref{eq:sams3} e \ref{eq:gamt3} sono nella forma:
\begin{equation*}
	A + B \xrightarrow{E} P + Q,
\end{equation*}
	che costituisce in realt\`a una rappresentazione astratta (che considera solo gli istanti iniziale e finale, ignorando i passaggi intermedi) della sequenza di reazioni pi\`u complessa:
\begin{align*}
	E + A &\rightleftharpoons EA\\
	EA + B &\rightleftharpoons EAB\\
	EAB &\rightarrow EPQ\\
	EPQ &\rightleftharpoons EQ + P\\
	EQ &\rightleftharpoons E + Q,
\end{align*}
	dove i due substrati $A$ e $B$ sono convertiti nei due prodotti $P$ e $Q$ ed \`e possibile un solo ordine di associazione (prima si forma il complesso binario $EA$ e poi quello ternario $EAB$, analogamente, prima si dissocia il prodotto $P$ e poi quello $Q$).
	
	Ai fini del calcolo della cinetica, risulta essere pi\`u conveniente ragionare sui singoli passi della reazione, anzich\'e sulla rappresentazione astratta. Una volta ricavata l'equazione della cinetica, si pu\`o applicare quest'ultima direttamente alla reazione astratta, per semplificare il modello.

Utilizzando metodi come quello di King-Altman~\cite{king1956schematic}, che semplificano il calcolo della cinetica nonostante l'elevato numero di reazioni coinvolte, si ottiene l'equazione:
\begin{equation}\label{eq:michaelistern}
v =  \frac{\frac{V_{max}}{km_B + [B]} \cdot [A]}{\frac{ki \cdot km_B + km_A \cdot [B]}{km_B + [B]} + [A]}.
\end{equation}
Risulta evidente che:
\begin{align*}
	V^{app}_{max} &= \frac{V_{max}}{km_B + [B]}\\
	K^{app}_M &= \frac{ki \cdot km_B + km_A \cdot [B]}{km_B + [B]},
\end{align*}
dove $k_i$ indica un'eventuale costante di inibizione, $km_A$ e $km_B$ indicano le affinit\`a dell'enzima rispetto ad $A$ e $B$, rispettivamente, e $V_{max}$ indica la velocit\`a massima dell'enzima.

Come precedentemente affermato, la funzione \ref{eq:michaelistern} definisce l'andamento della velocit\`a delle reazioni catalizzate da SAMS e GAMT.

\begin{figure}
	\center
	\begin{tikzpicture}
	\begin{axis}[xmin=0, xmax=20, ymin=0, ymax=1.1,domain=0:20,samples=100, xtick={1}, xticklabels={$K_m$}, ytick={0.5,1}, yticklabels={$\frac{V_{max}}{2}$, $V_{max}$}, xlabel={$[S]$}, ylabel={$v$}]
	\addplot[black]{x/(1+x)};
	\addplot[black, dashed]{1};
	\path[draw=black, dashed] (10, 50) -- (10, 0);
	\end{axis}
	\end{tikzpicture}
	\caption{Andamento della velocit\`a di reazione in funzione della concentrazione di substrato in un enzima di Michaelis-Menten}
	\label{fig:michaelis}
\end{figure}

\section{Modello Bio-PEPA}\label{sez:modpepa}
Il modello Bio-PEPA per il trattamento della deficienza GAMT a base di eritrociti ingegnerizzati, prodotto a partire dalle reazioni \ref{eq:sams3}, \ref{eq:uptake3} e \ref{eq:gamt3}, \`e il seguente:
\begin{align}
	sams &= \left [10 \cdot \frac{\frac{v\_sams}{km\_sams\_atp + atp} \cdot met}{\frac{ki\_sams\_sam \cdot km\_sams\_atp + km\_sams\_met \cdot atp}{km\_sams\_atp + atp} + met} \right ];\nonumber\\
	gamt &= \left [10 \cdot \frac{\frac{v\_gamt}{km\_gamt\_sam + \frac{SAM}{10}} \cdot \frac{GAA\_INT}{10}}{\frac{ki\_gamt\_sah \cdot km\_gamt\_sam + km\_gamt\_gaa\_int \cdot \frac{SAM}{10}}{km\_gamt\_sam + \frac{SAM}{10}} + \frac{GAA\_INT}{10}}\right ];\nonumber\\
	uptake &= \left [v\_uptake \cdot \frac{GAA\_EXT - GAA\_INT}{10} \right ];\nonumber\\
	UM\_SAM &= \left [\frac{SAM}{10}\right ];\nonumber\\
	UM\_SAH &= \left [\frac{SAH}{10}\right ];\nonumber\\
	UM\_GAA\_INT &= \left [\frac{GAA\_INT}{10}\right ];\nonumber\\
	UM\_GAA\_EXT &= \left [\frac{GAA\_EXT}{10}\right ];\nonumber\\
	UM\_CR &= \left [\frac{CR}{10}\right ];\label{mod:rate}
\end{align}
\begin{align}
	SAM &= sams \uparrow SAM + gamt \downarrow SAM;\nonumber\\
	SAH &= gamt \uparrow SAH;\nonumber\\
	GAA\_INT &= uptake \uparrow GAA\_INT + gamt \downarrow GAA\_INT;\nonumber\\
	GAA\_EXT &= uptake \downarrow GAA\_EXT;\nonumber\\
	CR &= gamt \uparrow CR;\label{mod:cost}
\end{align}
\begin{equation}
(SAM \underset{gamt}{\bowtie} SAH \underset{gamt}{\bowtie} GAA\_INT \underset{gamt}{\bowtie} CR)\underset{uptake}{\bowtie} GAA\_EXT.\label{mod:sincr}
\end{equation}

La porzione \ref{mod:rate} descrive rate e contatori definiti sul modello.
I rate sono associati alle tre azioni definite, che coincidono con le reazioni da modellare, ognuno \`e definito in base alla cinetica coinvolta ed \`e espresso in $10^{-7}$ $M \cdot min^{-1}$ (mentre le costanti sono definite in $\mu M$ e $\mu M \cdot min^{-1}$, motivo per cui \`e necessario includere fattori di conversione).
I contatori sono utilizzati per convertire le concentrazioni delle specie chimiche del modello in $\mu M$.

Le costanti di processo sono definite nella porzione \ref{mod:cost} e ognuna rappresenta il comportamento di ogni specie chimica, osservata in aliquote da $10^{-7}$ $M$.
Il comportamento dei processi \`e definito in termini di aumenti e diminuzioni a seguito di ogni reazione.
Il processo $SAM$, ad esempio, pu\`o aumentare a seguito della partecipazione alla reazione \ref{eq:sams3} (catalizzata da SAMS), oppure diminuire partecipando alla reazione \ref{eq:gamt3} (catalizzata da GAMT).

Il modello vero e proprio \`e costituito dalla porzione \ref{mod:sincr}, che sincronizza tutti i processi. La semantica dell'operatore di composizione parallela modella fedelmente il comportamento reale, poich\'e ogni reazione \`e costituita dalla trasformazione simultanea di tutte le specie coinvolte, ma, allo stesso tempo, due specie non coinvolte nella stessa reazione possono trasformarsi indipendentemente l'una dall'altra.
Essendo il risultato della composizione di due processi un nuovo processo, \`e possibile costruire il modello incrementalmente, sincronizzando man mano le singole specie, si pu\`o dunque immaginare la porzione \ref{mod:sincr} costruita come:
\begin{enumerate}
	\item $SAM$ e $SAH$ sincronizzati in un unico processo $reagenti\_gamt$;
	\item $GAA\_INT$ e $CR$ sincronizzati nel processo $prodotti\_gamt$;
	\item $reagenti\_gamt$ e $prodotti_gamt$ sincronizzati nel processo $reazione\_gamt$ (dove l'azione $gamt$ non richiede ulteriori sincronizzazioni, dato che tutti i processi che la utilizzano sono stati accorpati);
	\item $reazione\_gamt$ e $GAA\_EXT$ sincronizzati nel modello completo (dato che l'azione $uptake$ va sincronizzata solo tra $GAA\_INT$, che \`e incluso in $reazioni\_gamt$ e $GAA\_EXT$ e dato che l'azione $sams$ non ha bisogno di sincronizzazioni, essendo eseguita dal solo processo $SAM$).
	
\end{enumerate}

Dallo stesso modello \`e possibile analizzare diversi aspetti della terapia semplicemente modificando il valore delle quantit\`a iniziali e delle costanti. Le tabelle seguenti indicano i parametri utilizzati per ogni analisi.

\subsection{Ingresso di guanidinoacetato negli eritrociti}
In questa serie di istanze del modello si replica l'esperimento descritto nella sezione \ref{sez:uptake}, per validare la fedelt\`a del modello rispetto ai dati sperimentali.
Di queste istanze verr\`a fatto un utilizzo esclusivamente simulativo.

\begin{table}[H]
	\centering
	\begin{tabular}{| c | c | c | c |}
	\hline
	Parametro & Valore & Unit\`a & Note \\
	\hline
	$SAM$ & $0$ & \si{10^{-7} M} & \\
	\hline
	$SAH$ & $0$ & \si{10^{-7} M} & \\
	\hline
	$GAA\_INT$ & $5$ & \si{10^{-7} M} & valore sperimentale \\
	\hline
	$GAA\_EXT$ & $100$ & \si{10^{-7} M} & valore sperimentale \\
	\hline
	$CR$ & $0$ & \si{10^{-7} M} & \\
	\hline
	$met$ & $0$ & \si{\mu M} & \\
	\hline
	$atp$ & $0$ & \si{\mu M} & \\
	\hline
	$v\_sams$ & $0$ & \si{\mu M / min} & \\
	\hline
	$v\_gamt$ & $0$ & \si{\mu M / min} & \\
	\hline
	$km\_sams\_atp$ & $0$ & \si{\mu M} & \\
	\hline
	$km\_sams\_met$ & $0$ & \si{\mu M} & \\
	\hline
	$ki\_sams\_sam$ & $0$ & \si{\mu M} & \\
	\hline
	$km\_gamt\_sam$ & $0$ & \si{\mu M} & \\
	\hline
	$km\_gamt\_gaa\_int$ & $0$ & \si{\mu M} & \\
	\hline
	$ki\_gamt\_sah$ & $0$ & \si{\mu M} & \\
	\hline
	$v\_uptake$ & $0.02$ & \si{\mu M/min} & valore sperimentale \\
	\hline
\end{tabular}
\caption{Esperimento di uptake con \SI{10}{\mu M} di guanidinoacetato extracellulare}
\label{mod:1}
\end{table}

\begin{table}[H]
	\centering
	\begin{tabular}{| c | c | c | c |}
	\hline
	Parametro & Valore & Unit\`a & Note \\
		\hline
		$SAM$ & $0$ & \si{10^{-7} M} & \\
		\hline
		$SAH$ & $0$ & \si{10^{-7} M} & \\
		\hline
		$GAA\_INT$ & $12$ & \si{10^{-7} M} & valore sperimentale \\
		\hline
		$GAA\_EXT$ & $250$ & \si{10^{-7} M} & valore sperimentale \\
		\hline
		$CR$ & $0$ & \si{10^{-7} M} & \\
		\hline
		$met$ & $0$ & \si{\mu M} & \\
		\hline
		$atp$ & $0$ & \si{\mu M} & \\
		\hline
		$v\_sams$ & $0$ & \si{\mu M / min} & \\
		\hline
		$v\_gamt$ & $0$ & \si{\mu M / min} & \\
		\hline
		$km\_sams\_atp$ & $0$ & \si{\mu M} & \\
		\hline
		$km\_sams\_met$ & $0$ & \si{\mu M} & \\
		\hline
		$ki\_sams\_sam$ & $0$ & \si{\mu M} & \\
		\hline
		$km\_gamt\_sam$ & $0$ & \si{\mu M} & \\
		\hline
		$km\_gamt\_gaa\_int$ & $0$ & \si{\mu M} & \\
		\hline
		$ki\_gamt\_sah$ & $0$ & \si{\mu M} & \\
		\hline
		$v\_uptake$ & $0.02$ & \si{\mu M/min} & \\
		\hline
	\end{tabular}
	\caption{Esperimento di uptake con \SI{25}{\mu M} di guanidinoacetato extracellulare}
	\label{mod:2}
\end{table}

\begin{table}[H]
	\centering
	\begin{tabular}{| c | c | c | c |}
	\hline
	Parametro & Valore & Unit\`a & Note \\
		\hline
		$SAM$ & $0$ & \si{10^{-7} M} & \\
		\hline
		$SAH$ & $0$ & \si{10^{-7} M} & \\
		\hline
		$GAA\_INT$ & $25$ & \si{10^{-7} M} & valore sperimentale \\
		\hline
		$GAA\_EXT$ & $500$ & \si{10^{-7} M} & valore sperimentale \\
		\hline
		$CR$ & $0$ & \si{10^{-7} M} & \\
		\hline
		$met$ & $0$ & \si{\mu M} & \\
		\hline
		$atp$ & $0$ & \si{\mu M} & \\
		\hline
		$v\_sams$ & $0$ & \si{\mu M / min} & \\
		\hline
		$v\_gamt$ & $0$ & \si{\mu M / min} & \\
		\hline
		$km\_sams\_atp$ & $0$ & \si{\mu M} & \\
		\hline
		$km\_sams\_met$ & $0$ & \si{\mu M} & \\
		\hline
		$ki\_sams\_sam$ & $0$ & \si{\mu M} & \\
		\hline
		$km\_gamt\_sam$ & $0$ & \si{\mu M} & \\
		\hline
		$km\_gamt\_gaa\_int$ & $0$ & \si{\mu M} & \\
		\hline
		$ki\_gamt\_sah$ & $0$ & \si{\mu M} & \\
		\hline
		$v\_uptake$ & $0.02$ & \si{\mu M/min} & \\
		\hline
	\end{tabular}
	\caption{Esperimento di uptake con \SI{50}{\mu M} di guanidinoacetato extracellulare}
	\label{mod:3}
\end{table}

\begin{table}[H]
	\centering
	\begin{tabular}{| c | c | c | c |}
	\hline
	Parametro & Valore & Unit\`a & Note \\
		\hline
		$SAM$ & $0$ & \si{10^{-7} M} & \\
		\hline
		$SAH$ & $0$ & \si{10^{-7} M} & \\
		\hline
		$GAA\_INT$ & $52$ & \si{10^{-7} M} & valore sperimentale \\
		\hline
		$GAA\_EXT$ & $1000$ & \si{10^{-7} M} & valore sperimentale \\
		\hline
		$CR$ & $0$ & \si{10^{-7} M} & \\
		\hline
		$met$ & $0$ & \si{\mu M} & \\
		\hline
		$atp$ & $0$ & \si{\mu M} & \\
		\hline
		$v\_sams$ & $0$ & \si{\mu M / min} & \\
		\hline
		$v\_gamt$ & $0$ & \si{\mu M / min} & \\
		\hline
		$km\_sams\_atp$ & $0$ & \si{\mu M} & \\
		\hline
		$km\_sams\_met$ & $0$ & \si{\mu M} & \\
		\hline
		$ki\_sams\_sam$ & $0$ & \si{\mu M} & \\
		\hline
		$km\_gamt\_sam$ & $0$ & \si{\mu M} & \\
		\hline
		$km\_gamt\_gaa\_int$ & $0$ & \si{\mu M} & \\
		\hline
		$ki\_gamt\_sah$ & $0$ & \si{\mu M} & \\
		\hline
		$v\_uptake$ & $0.02$ & \si{\mu M/min} & \\
		\hline
	\end{tabular}
	\caption{Esperimento di uptake con \SI{100}{\mu M} di guanidinoacetato extracellulare}
	\label{mod:4}
\end{table}

\subsection{Sintesi di creatina in eritrociti incapsulati con SAM esogena}
In questa serie di istanze del modello si replica l'esperimento descritto nella sezione \ref{sez:dosaggio}, per validare la fedelt\`a del modello rispetto ai dati sperimentali.
Anche queste istanze verranno sottoposte esclusivamente ad analisi simulativa.

\begin{table}[H]
	\centering
	\begin{tabular}{| c | c | c | c |}
	\hline
	Parametro & Valore & Unit\`a & Note \\
		\hline
		$SAM$ & $100$ & \si{10^{-7} M} & valore sperimentale \\
		\hline
		$SAH$ & $13$ & \si{10^{-7} M} & \cite{oden1983s} \\
		\hline
		$GAA\_INT$ & $660$ & \si{10^{-7} M} & valore sperimentale \\
		\hline
		$GAA\_EXT$ & $1000$ & \si{10^{-7} M} & valore sperimentale \\
		\hline
		$CR$ & $26$ & \si{10^{-7} M} & valore sperimentale \\
		\hline
		$met$ & $12$ & \si{\mu M} &\cite{oden1983s} \\
		\hline
		$atp$ & $1100$ & \si{\mu M} & \cite{oden1983s} \\
		\hline
		$v\_sams$ & $0.033$ & \si{\mu M / min} & \cite{kim1974s} \\
		\hline
		$v\_gamt$ & $33$ & \si{\mu M / min} & valore sperimentale \\
		\hline
		$km\_sams\_atp$ & $80$ & \si{\mu M} & \cite{oden1983s} \\
		\hline
		$km\_sams\_met$ & $2$ & \si{\mu M} & \cite{oden1983s} \\
		\hline
		$ki\_sams\_sam$ & $2$ & \si{\mu M} & \cite{oden1983s} \\
		\hline
		$km\_gamt\_sam$ & $14.8$ & \si{\mu M} & \cite{ilas2000guanidinoacetate} \\
		\hline
		$km\_gamt\_gaa\_int$ & $78$ & \si{\mu M} & \cite{ilas2000guanidinoacetate} \\
		\hline
		$ki\_gamt\_sah$ & $0.4$ & \si{\mu M} &\cite{brendagamt} \\
		\hline
		$v\_uptake$ & $0.02$ & \si{\mu M/min} & \\
		\hline
	\end{tabular}
	\caption{Esperimento di sintesi della creatina con \SI{10}{\mu M} di S-adenosil metionina incapsulata}
	\label{mod:5}
\end{table}

\begin{table}[H]
	\centering
	\begin{tabular}{| c | c | c | c |}
	\hline
	Parametro & Valore & Unit\`a & Note \\
		\hline
		$SAM$ & $500$ & \si{10^{-7} M} & valore sperimentale \\
		\hline
		$SAH$ & $13$ & \si{10^{-7} M} & \\
		\hline
		$GAA\_INT$ & $663$ & \si{10^{-7} M} & valore sperimentale \\
		\hline
		$GAA\_EXT$ & $1000$ & \si{10^{-7} M} & \\
		\hline
		$CR$ & $73$ & \si{10^{-7} M} & valore sperimentale \\
		\hline
		$met$ & $12$ & \si{\mu M} & \\
		\hline
		$atp$ & $1100$ & \si{\mu M} & \\
		\hline
		$v\_sams$ & $0.033$ & \si{\mu M / min} & \\
		\hline
		$v\_gamt$ & $33$ & \si{\mu M / min} & \\
		\hline
		$km\_sams\_atp$ & $80$ & \si{\mu M} & \\
		\hline
		$km\_sams\_met$ & $2$ & \si{\mu M} & \\
		\hline
		$ki\_sams\_sam$ & $2$ & \si{\mu M} & \\
		\hline
		$km\_gamt\_sam$ & $14.8$ & \si{\mu M} & \\
		\hline
		$km\_gamt\_gaa\_int$ & $78$ & \si{\mu M} & \\
		\hline
		$ki\_gamt\_sah$ & $0.4$ & \si{\mu M} & \\
		\hline
		$v\_uptake$ & $0.02$ & \si{\mu M/min} & \\
		\hline
	\end{tabular}
	\caption{Esperimento di sintesi della creatina con \SI{50}{\mu M} di S-adenosil metionina incapsulata}
	\label{mod:6}
\end{table}

\begin{table}[H]
	\centering
	\begin{tabular}{| c | c | c | c |}
	\hline
	Parametro & Valore & Unit\`a & Note \\
		\hline
		$SAM$ & $35$ & \si{10^{-7} M} & \cite{oden1983s} \\
		\hline
		$SAH$ & $13$ & \si{10^{-7} M} & \\
		\hline
		$GAA\_INT$ & $598$ & \si{10^{-7} M} &  valore sperimentale \\
		\hline
		$GAA\_EXT$ & $1000$ & \si{10^{-7} M} & \\
		\hline
		$CR$ & $1$ & \si{10^{-7} M} &  valore sperimentale \\
		\hline
		$met$ & $12$ & \si{\mu M} & \\
		\hline
		$atp$ & $1100$ & \si{\mu M} & \\
		\hline
		$v\_sams$ & $0.033$ & \si{\mu M / min} & \\
		\hline
		$v\_gamt$ & $0$ & \si{\mu M / min} & non incapsulata \\
		\hline
		$km\_sams\_atp$ & $80$ & \si{\mu M} & \\
		\hline
		$km\_sams\_met$ & $2$ & \si{\mu M} & \\
		\hline
		$ki\_sams\_sam$ & $2$ & \si{\mu M} & \\
		\hline
		$km\_gamt\_sam$ & $14.8$ & \si{\mu M} & \\
		\hline
		$km\_gamt\_gaa\_int$ & $78$ & \si{\mu M} & \\
		\hline
		$ki\_gamt\_sah$ & $0.4$ & \si{\mu M} & \\
		\hline
		$v\_uptake$ & $0.02$ & \si{\mu M/min} & \\
		\hline
	\end{tabular}
	\caption{Esperimento di controllo della sintesi della creatina con S-adenosil metionina endogena}
	\label{mod:7}
\end{table}

\subsection{Predizione della sintesi di creatina in eritrociti incapsulati con SAM sintasi di \emph{E.\ coli}}
In questa serie di istanze del modello si prevede l'andamento di esperimenti non ancora effettuati, dove si valuta la quantit\`a di SAM sintasi di \emph{E.\ coli} (con affinit\`a e costanti di equilibrio diverse rispetto a quella eritrocitaria, usata come controllo) da incapsulare per fare in modo che l'S-adenosil metionina non sia il reagente limitante del metabolismo.
Queste istanze verranno utilizzate sia per l'applicazione delle tecniche simulative che di verifica.

\begin{table}[H]
	\centering
	\begin{tabular}{| c | c | c | c |}
	\hline
	Parametro & Valore & Unit\`a & Note \\
		\hline
		$SAM$ & $35$ & \si{10^{-7} M} & \\
		\hline
		$SAH$ & $13$ & \si{10^{-7} M} & \\
		\hline
		$GAA\_INT$ & $37$ & \si{10^{-7} M} & livelli patologici~\cite{kikuchi1981liquid} \\
		\hline
		$GAA\_EXT$ & $500$ & \si{10^{-7} M} & concentrazione massima ipotizzata \\
		\hline
		$CR$ & $0$ & \si{10^{-7} M} & \\
		\hline
		$met$ & $12$ & \si{\mu M} & \\
		\hline
		$atp$ & $1100$ & \si{\mu M} & \\
		\hline
		$v\_sams$ & $0.033$ & \si{\mu M / min} & \\
		\hline
		$v\_gamt$ & $15$ & \si{\mu M / min} & assumendo un incapsulamento del \\&&&30\% circa di \SI{1}{mg/ml} \\
		\hline
		$km\_sams\_atp$ & $2$ & \si{\mu M} & \\
		\hline
		$km\_sams\_met$ & $80$ & \si{\mu M} & \\
		\hline
		$ki\_sams\_sam$ & $2$ & \si{\mu M} & \\
		\hline
		$km\_gamt\_sam$ & $14.8$ & \si{\mu M} & \\
		\hline
		$km\_gamt\_gaa\_int$ & $78$ & \si{\mu M} & \\
		\hline
		$ki\_gamt\_sah$ & $0.4$ & \si{\mu M} & \\
		\hline
		$v\_uptake$ & $0.02$ & \si{\mu M/min} & \\
		\hline
	\end{tabular}
	\caption{Esperimento di controllo della sintesi della creatina con SAM sintasi endogena}
	\label{mod:8}
\end{table}

\begin{table}[H]
	\centering
	\begin{tabular}{| c | c | c | c |}
	\hline
	Parametro & Valore & Unit\`a & Note \\
		\hline
		$SAM$ & $35$ & \si{10^{-7} M} & \\
		\hline
		$SAH$ & $13$ & \si{10^{-7} M} & \\
		\hline
		$GAA\_INT$ & $37$ & \si{10^{-7} M} & \\
		\hline
		$GAA\_EXT$ & $500$ & \si{10^{-7} M} & \\
		\hline
		$CR$ & $0$ & \si{10^{-7} M} & \\
		\hline
		$met$ & $20$ & \si{\mu M} & \\
		\hline
		$atp$ & $1100$ & \si{\mu M} & \\
		\hline
		$v\_sams$ & $20$ & \si{\mu M / min} & \cite{brendasams}, \SI{0.05}{mg} incapsulati \\
		\hline
		$v\_gamt$ & $15$ & \si{\mu M / min} & \\
		\hline
		$km\_sams\_atp$ & $73$ & \si{\mu M} & \cite{brendasams} \\
		\hline
		$km\_sams\_met$ & $75$ & \si{\mu M} & \cite{brendasams} \\
		\hline
		$ki\_sams\_sam$ & $0$ & \si{\mu M} & \cite{brendasams} \\
		\hline
		$km\_gamt\_sam$ & $14.8$ & \si{\mu M} & \\
		\hline
		$km\_gamt\_gaa\_int$ & $78$ & \si{\mu M} & \\
		\hline
		$ki\_gamt\_sah$ & $0.4$ & \si{\mu M} & \\
		\hline
		$v\_uptake$ & $0.02$ & \si{\mu M/min} & \\
		\hline
	\end{tabular}
	\caption{Esperimento di sintesi della creatina con \SI{0.05}{mg} di SAM sintasi di \emph{E.\ coli} incapsulati}
	\label{mod:9}
\end{table}

\begin{table}[H]
	\centering
	\begin{tabular}{| c | c | c | c |}
	\hline
	Parametro & Valore & Unit\`a & Note \\
		\hline
		$SAM$ & $35$ & \si{10^{-7} M} & \\
		\hline
		$SAH$ & $13$ & \si{10^{-7} M} & \\
		\hline
		$GAA\_INT$ & $37$ & \si{10^{-7} M} & \\
		\hline
		$GAA\_EXT$ & $500$ & \si{10^{-7} M} & \\
		\hline
		$CR$ & $0$ & \si{10^{-7} M} & \\
		\hline
		$met$ & $20$ & \si{\mu M} & \\
		\hline
		$atp$ & $1100$ & \si{\mu M} & \\
		\hline
		$v\_sams$ & $40$ & \si{\mu M / min} & \SI{0.1}{mg} incapsulati \\
		\hline
		$v\_gamt$ & $15$ & \si{\mu M / min} & \\
		\hline
		$km\_sams\_atp$ & $73$ & \si{\mu M} & \\
		\hline
		$km\_sams\_met$ & $75$ & \si{\mu M} & \\
		\hline
		$ki\_sams\_sam$ & $0$ & \si{\mu M} & \\
		\hline
		$km\_gamt\_sam$ & $14.8$ & \si{\mu M} & \\
		\hline
		$km\_gamt\_gaa\_int$ & $78$ & \si{\mu M} & \\
		\hline
		$ki\_gamt\_sah$ & $0.4$ & \si{\mu M} & \\
		\hline
		$v\_uptake$ & $0.02$ & \si{\mu M/min} & \\
		\hline
	\end{tabular}
	\caption{Esperimento di sintesi della creatina con \SI{0.1}{mg} di SAM sintasi di \emph{E.\ coli} incapsulati}
	\label{mod:10}
\end{table}

\begin{table}[H]
	\centering
	\begin{tabular}{| c | c | c | c |}
	\hline
	Parametro & Valore & Unit\`a & Note \\
		\hline
		$SAM$ & $35$ & \si{10^{-7} M} & \\
		\hline
		$SAH$ & $13$ & \si{10^{-7} M} & \\
		\hline
		$GAA\_INT$ & $37$ & \si{10^{-7} M} & \\
		\hline
		$GAA\_EXT$ & $500$ & \si{10^{-7} M} & \\
		\hline
		$CR$ & $0$ & \si{10^{-7} M} & \\
		\hline
		$met$ & $20$ & \si{\mu M} & \\
		\hline
		$atp$ & $1100$ & \si{\mu M} & \\
		\hline
		$v\_sams$ & $200$ & \si{\mu M / min} & \SI{0.5}{mg} incapsulati \\
		\hline
		$v\_gamt$ & $15$ & \si{\mu M / min} & \\
		\hline
		$km\_sams\_atp$ & $73$ & \si{\mu M} & \\
		\hline
		$km\_sams\_met$ & $75$ & \si{\mu M} & \\
		\hline
		$ki\_sams\_sam$ & $0$ & \si{\mu M} & \\
		\hline
		$km\_gamt\_sam$ & $14.8$ & \si{\mu M} & \\
		\hline
		$km\_gamt\_gaa\_int$ & $78$ & \si{\mu M} & \\
		\hline
		$ki\_gamt\_sah$ & $0.4$ & \si{\mu M} & \\
		\hline
		$v\_uptake$ & $0.02$ & \si{\mu M/min} & \\
		\hline
	\end{tabular}
	\caption{Esperimento di sintesi della creatina con \SI{0.5}{mg} di SAM sintasi di \emph{E.\ coli} incapsulati}
	\label{mod:11}
\end{table}

\begin{table}[H]
	\centering
	\begin{tabular}{| c | c | c | c |}
	\hline
	Parametro & Valore & Unit\`a & Note \\
		\hline
		$SAM$ & $35$ & \si{10^{-7} M} & \\
		\hline
		$SAH$ & $13$ & \si{10^{-7} M} & \\
		\hline
		$GAA\_INT$ & $37$ & \si{10^{-7} M} & \\
		\hline
		$GAA\_EXT$ & $500$ & \si{10^{-7} M} & \\
		\hline
		$CR$ & $0$ & \si{10^{-7} M} & \\
		\hline
		$met$ & $20$ & \si{\mu M} & \\
		\hline
		$atp$ & $1100$ & \si{\mu M} & \\
		\hline
		$v\_sams$ & $400$ & \si{\mu M / min} & \SI{1}{mg} incapsulati \\
		\hline
		$v\_gamt$ & $15$ & \si{\mu M / min} & \\
		\hline
		$km\_sams\_atp$ & $73$ & \si{\mu M} & \\
		\hline
		$km\_sams\_met$ & $75$ & \si{\mu M} & \\
		\hline
		$ki\_sams\_sam$ & $0$ & \si{\mu M} & \\
		\hline
		$km\_gamt\_sam$ & $14.8$ & \si{\mu M} & \\
		\hline
		$km\_gamt\_gaa\_int$ & $78$ & \si{\mu M} & \\
		\hline
		$ki\_gamt\_sah$ & $0.4$ & \si{\mu M} & \\
		\hline
		$v\_uptake$ & $0.02$ & \si{\mu M/min} & \\
		\hline
	\end{tabular}
	\caption{Esperimento di sintesi della creatina con \SI{1}{mg} di SAM sintasi di \emph{E.\ coli} incapsulati}
	\label{mod:12}
\end{table}

\begin{table}[H]
	\centering
	\begin{tabular}{| c | c | c | c |}
	\hline
	Parametro & Valore & Unit\`a & Note \\
		\hline
		$SAM$ & $35$ & \si{10^{-7} M} & \\
		\hline
		$SAH$ & $13$ & \si{10^{-7} M} & \\
		\hline
		$GAA\_INT$ & $37$ & \si{10^{-7} M} & \\
		\hline
		$GAA\_EXT$ & $500$ & \si{10^{-7} M} & \\
		\hline
		$CR$ & $0$ & \si{10^{-7} M} & \\
		\hline
		$met$ & $20$ & \si{\mu M} & \\
		\hline
		$atp$ & $1100$ & \si{\mu M} & \\
		\hline
		$v\_sams$ & $1000$ & \si{\mu M / min} & \SI{2.5}{mg} incapsulati \\
		\hline
		$v\_gamt$ & $15$ & \si{\mu M / min} & \\
		\hline
		$km\_sams\_atp$ & $73$ & \si{\mu M} & \\
		\hline
		$km\_sams\_met$ & $75$ & \si{\mu M} & \\
		\hline
		$ki\_sams\_sam$ & $0$ & \si{\mu M} & \\
		\hline
		$km\_gamt\_sam$ & $14.8$ & \si{\mu M} & \\
		\hline
		$km\_gamt\_gaa\_int$ & $78$ & \si{\mu M} & \\
		\hline
		$ki\_gamt\_sah$ & $0.4$ & \si{\mu M} & \\
		\hline
		$v\_uptake$ & $0.02$ & \si{\mu M/min} & \\
		\hline
	\end{tabular}
	\caption{Esperimento di sintesi della creatina con \SI{2.5}{mg} di SAM sintasi di \emph{E.\ coli} incapsulati}
	\label{mod:13}
\end{table}

\begin{table}[H]
	\centering
	\begin{tabular}{| c | c | c | c |}
	\hline
	Parametro & Valore & Unit\`a & Note \\
		\hline
		$SAM$ & $35$ & \si{10^{-7} M} & \\
		\hline
		$SAH$ & $13$ & \si{10^{-7} M} & \\
		\hline
		$GAA\_INT$ & $37$ & \si{10^{-7} M} & \\
		\hline
		$GAA\_EXT$ & $500$ & \si{10^{-7} M} & \\
		\hline
		$CR$ & $0$ & \si{10^{-7} M} & \\
		\hline
		$met$ & $20$ & \si{\mu M} & \\
		\hline
		$atp$ & $1100$ & \si{\mu M} & \\
		\hline
		$v\_sams$ & $2000$ & \si{\mu M / min} & \SI{5}{mg} incapsulati \\
		\hline
		$v\_gamt$ & $15$ & \si{\mu M / min} & \\
		\hline
		$km\_sams\_atp$ & $73$ & \si{\mu M} & \\
		\hline
		$km\_sams\_met$ & $75$ & \si{\mu M} & \\
		\hline
		$ki\_sams\_sam$ & $0$ & \si{\mu M} & \\
		\hline
		$km\_gamt\_sam$ & $14.8$ & \si{\mu M} & \\
		\hline
		$km\_gamt\_gaa\_int$ & $78$ & \si{\mu M} & \\
		\hline
		$ki\_gamt\_sah$ & $0.4$ & \si{\mu M} & \\
		\hline
		$v\_uptake$ & $0.02$ & \si{\mu M/min} & \\
		\hline
	\end{tabular}
	\caption{Esperimento di sintesi della creatina con \SI{5}{mg} di SAM sintasi di \emph{E.\ coli} incapsulati}
	\label{mod:14}
\end{table}

\section{Modelli PRISM}\label{sez:modprism}
I modelli PRISM compilati a partire dalle istanze del modello Bio-PEPA per effettuare l'analisi sono tutti della stessa forma, cambiando soltanto le costanti definite a inizio file e le quantit\`a iniziali dei moduli.

Per semplicit\`a viene mostrato un unico modello parametrizzato, indicando in corsivo gli identificatori Bio-PEPA con cui sostituire i parametri (ad esempio, nel modello \ref{mod:1} \`e necessario sostituire $met$ con $0$, nel modello \ref{mod:5} con $12$ e cos\`i via).
Per ragioni di spazio, le cinetiche sono indicate con $sams$, $gamt$ e $uptake$ (anzich\'e con le formule gi\`a indicate nel modello Bio-PEPA), sebbene siano le stesse per ogni modello.

In questo modello parametrico vengono definiti un modulo per ogni specie (dove nel caso di un aumento viene controllato che non sia raggiunta la saturazione e nel caso di una diminuzione viene controllato che la specie non sia esaurita), e un modulo complessivo per tutti i rate.
Vengono inoltre definite tutte le strutture di reward che verranno utilizzate per l'analisi e che corrispondono al numero di reazioni effettuate, alla quantit\`a di ogni substrato, al quadrato della quantit\`a e ai contatori Bio-PEPA definiti.

\begin{lstlisting}[mathescape=true,language=prism]
ctmc
	const double _met = $met$;
	const double _atp = $atp$;
	const double _v_sams = $v\_sams$;
	const double _v_gamt = $v\_gamt$;
	const double _km_sams_atp = $km\_sams\_atp$;
	const double _km_sams_met = $km\_sams\_met$;
	const double _ki_sams_sam = $ki\_sams\_sam$;
	const double _km_gamt_sam = $km\_gamt\_sam$;
	const double _km_gamt_gaa_int = $km\_gaa\_int$;
	const double _ki_gamt_sah = $km\_gamt\_sah$;
	const double _v_uptake = $v\_uptake$;
	
	module Rates
		[_sams] $sams$ : true;
		[_gamt] $gamt$ : true;
		[_uptake] $uptake$ : true;
	endmodule
	
	const int MAX = $SAM$ + $SAH$ + $GAA\_INT$ + $GAA\_EXT$ + $CR$;
	
	module _SAM
		_SAM : [0..MAX] init $SAM$;
		[_sams] (_SAM + 1 $\leq$ MAX) $\rightarrow$ 1 : (_SAM' = _SAM + 1);
		[_gamt] (_SAM $\geq$ 1) $\rightarrow$ 1 : (_SAM' = _SAM - 1);
	endmodule
	
	module _SAH
		_SAH : [0..MAX] init $SAH$;
		[_gamt] (_SAH + 1 $\leq$ MAX) $\rightarrow$ 1 : (_SAH' = _SAH + 1);
	endmodule
	
	module _GAA_INT
	_GAA_INT : [0..MAX] init $GAA\_INT$;
		[_uptake] (_GAA_INT + 1 $\leq$ MAX) $\rightarrow$ 1 : (_GAA_INT' = _GAA_INT + 1);
		[_gamt] (_GAA_INT $\geq$ 1) $\rightarrow$ 1 : (_GAA_INT' = _GAA_INT - 1);
	endmodule
	
	module _GAA_EXT
		_GAA_EXT : [0..MAX] init $GAA\_EXT$;
		[_uptake] (_GAA_EXT $\geq$ 1) $\rightarrow$ 1 : (_GAA_EXT' = _GAA_EXT - 1);
	endmodule
	
	module _CR
		_CR : [0..MAX] init $CR$;
		[_gamt] (_CR + 1 $\leq$ MAX) $\rightarrow$ 1 : (_CR' = _CR + 1);
	endmodule
	
	
	rewards "_sams"
		[_sams] true : 1;
	endrewards
	
	rewards "_gamt"
		[_gamt] true : 1;
	endrewards
	
	rewards "_uptake"
		[_uptake] true : 1;
	endrewards
	
	rewards "_SAM"
		true : _SAM;
	endrewards
	
	rewards "_SAM_squared"
		true : _SAM * _SAM;
	endrewards
	
	rewards "_SAH"
		true : _SAH;
	endrewards
	
	rewards "_SAH_squared"
		true : _SAH * _SAH;
	endrewards
	
	rewards "_GAA_INT"
		true : _GAA_INT;
	endrewards
	
	rewards "_GAA_INT_squared"
		true : _GAA_INT * _GAA_INT;
	endrewards
	
	rewards "_GAA_EXT"
		true : _GAA_EXT;
	endrewards
	
	rewards "_GAA_EXT_squared"
		true : _GAA_EXT * _GAA_EXT;
	endrewards
	
	rewards "_CR"
		true : _CR;
	endrewards
	
	rewards "_CR_squared"
		true : _CR * _CR;
	endrewards
	
	rewards "_UM_SAM"
		true : (_SAM / 10.0);
	endrewards
	
	rewards "_UM_SAH"
		true : (_SAH / 10.0);
	endrewards
	
	rewards "_UM_GAA_INT"
		true : (_GAA_INT / 10.0);
	endrewards
	
	rewards "_UM_GAA_EXT"
		true : (_GAA_EXT / 10.0);
	endrewards
	
	rewards "_UM_CR"
		true : (_CR / 10.0);
	endrewards\end{lstlisting}
	\chapter{Analisi simulativa del trattamento della deficienza GAMT}\label{cap:simulazione}
In questo capitolo viene presentato l'esito delle simulazioni effettuate sui modelli sotto forma di grafici relativi all'andamento delle varie specie in funzione del tempo.
Pi\`u in dettaglio, sono stati ripetuti gli esperimenti gi\`a effettuati \emph{in vitro}, relativamente all'uptake di guanidinoacetato (sez. \ref{sez:simup}) e al dosaggio della creatina prodotta (sez. \ref{sez:simsint}), e gli esperimenti non ancora effettuati relativi all'incapsulamento di SAMS e GAMT (sez. \ref{sez:simpred}).


\section{Simulazione dell'esperimento sull'uptake del guanidinoacetato}\label{sez:simup}
I modelli \ref{mod:1}-\ref{mod:4} sono stati utilizzati per simulare l'esperimento di uptake di guanidinoacetato (sezione \ref{sez:uptake}).
Ogni modello \`e stato simulato per 60 minuti, con un campionamento degli eventi ogni 10 minuti.
La figura \ref{fig:uptake} raggruppa le quattro simulazioni e le sovrappone ai dati sperimentali.
\begin{figure}[H]
	\center
	\resizebox{\textwidth}{!}{
		\begin{tikzpicture}[spy using outlines={rectangle, magnification=3.0}, connect spies]
		\begin{axis}[axis lines=middle, xmin=0, xmax=60, ymin=0, ymax=70,samples=1000, xtick={0,10,...,60}, xlabel={$t (min)$}, ylabel={$[GAA*] (\mu M)$}, legend style={font=\tiny},
		x label style={at={(axis description cs:0.5,-0.1)},anchor=north},
		y label style={at={(axis description cs:-0.1,.5)},rotate=90,anchor=south},
		tick label style={font=\tiny},
		label style={font=\tiny},
		legend pos=outer north east,
		legend entries={$10 \mu M$ GAA*, $25 \mu M$ GAA*, $50 \mu M$ GAA*, $100 \mu M$ GAA*, Sperimentali}]
		
		\addplot[
		ultra thin,
		scatter,forget plot,
		point meta=explicit symbolic,
		scatter/classes={
			gaa10={mark=asterisk,draw=black,mark size=2},
			gaa25={mark=pentagon, draw=black,mark size=2},
			gaa50={mark=diamond, draw=black,mark size=2},
			gaa100={mark=o,draw=black,mark size=2},
			uptakeexp={mark=+,draw=black,mark size=2}}
		]
		file{../Dati/001-004/001.txt};
		
		\addplot[
		ultra thin,
		scatter,forget plot,
		point meta=explicit symbolic,
		scatter/classes={
			gaa10={mark=asterisk,draw=black,mark size=2},
			gaa25={mark=pentagon, draw=black,mark size=2},
			gaa50={mark=diamond, draw=black,mark size=2},
			gaa100={mark=o,draw=black,mark size=2},
			uptakeexp={mark=+,draw=black,mark size=2}}
		]
		file{../Dati/001-004/002.txt};
		
		
		
		\addplot[
		ultra thin,
		scatter,forget plot,
		point meta=explicit symbolic,
		scatter/classes={
			gaa10={mark=asterisk,draw=black,mark size=2},
			gaa25={mark=pentagon, draw=black,mark size=2},
			gaa50={mark=diamond, draw=black,mark size=2},
			gaa100={mark=o,draw=black,mark size=2},
			uptakeexp={mark=+,draw=black,mark size=2}}
		]
		file{../Dati/001-004/003.txt};
		
		\addplot[
		ultra thin,
		scatter,forget plot,
		point meta=explicit symbolic,
		scatter/classes={
			gaa10={mark=asterisk,draw=black,mark size=2},
			gaa25={mark=pentagon, draw=black,mark size=2},
			gaa50={mark=diamond, draw=black,mark size=2},
			gaa100={mark=o,draw=black,mark size=2},
			uptakeexp={mark=+,draw=black,mark size=2}}
		]
		file{../Dati/001-004/004.txt};
		
		\addplot[
		ultra thin,
		scatter,
		point meta=explicit symbolic,
		only marks,
		forget plot,
		scatter/classes={
			gaa10={mark=asterisk,draw=black,mark size=2},
			gaa25={mark=pentagon, draw=black,mark size=2},
			gaa50={mark=diamond, draw=black,mark size=2},
			gaa100={mark=o,draw=black,mark size=2},
			uptakeexp={mark=+,draw=black,mark size=2}}
		]
		file{../Dati/uptake.txt};
		
		\addlegendimage{mark=asterisk}
		\addlegendimage{mark=pentagon}
		\addlegendimage{mark=diamond}
		\addlegendimage{mark=o}
		\addlegendimage{only marks, mark=+}
		
		
		
		\end{axis}
		\end{tikzpicture}
	}
	\caption{Validazione dell'esperimento di uptake di guanidinoacetato}
	\label{fig:uptake}
\end{figure}
Si pu\`o notare come la fedelt\`a delle simulazioni rispetto ai dati sperimentali sia molto alta per tutte le concentrazioni iniziali considerate, ad eccezione del caso dell'incubazione con $100 \mu M$ di guanidinoacetato.
Tale curva presenta una leggera sovrastima rispetto ai dati sperimentali da attribuire, probabilmente, a una sottospecifica del modello.
Il volume degli eritrociti \`e infatti finito ed \`e ragionevole pensare che si presentino fenomeni di saturazione non dipendenti dal meccanismo di trasporto (che essendo una diffusione passiva non pu\`o essere, per sua natura, saturabile), bens\`i dal poco spazio a disposizione all'interno degli eritrociti per contenere grandi quantit\`a di guanidinoacetato.

Questa sovrastima \`e comunque accettabile, considerando il fatto che le aliquote di eritrociti ingegnerizzati utilizzate per questa serie di esperimenti risulta molto bassa rispetto a quelle utilizzate per gli esperimenti successivi.

\section{Simulazione dell'esperimento sulla sintesi di creatina}\label{sez:simsint}
	I modelli \ref{mod:5}-\ref{mod:7} simulano l'esperimento di dosaggio della creatina prodotta da eritrociti ingegnerizzati con GAMT (sezione \ref{sez:dosaggio}).
	Ogni modello \`e stato simulato per 10 ore, con un campionamento ogni minuto.
	Nei grafici \ref{fig:syntgaa} e \ref{fig:syntcr} sono mostrati gli andamenti delle concentrazioni del guanidinoacetato intracellulare e della creatina nel tempo.
	Il numero di marker \`e stato ridotto per aumentare la leggibilit\`a dei grafici.
	
	Si pu\`o osservare dalle curve ``Unloaded'', relative al modello \ref{mod:5}, che gli eritrociti nativi non sono dotati di alcuna attivit\`a di sintesi della creatina, quindi si limitano ad accumulare guanidinoacetato al loro interno.
	
	Le altre curve presentano invece due regioni separate da un brusco cambiamento di tendenza che coincide con l'esaurimento di S-adenosil metionina incapsulata.
	Nella prima si osservano una decrescita del guanidinoacetato e una crescita della creatina prodotta, indicando l'utilizzo dell'S-adenosil metionina incapsulata.
	La seconda mostra invece un andamento costante della creatina e un accumulo di guanidinoacetato, indicando che una volta esaurita l'S-adenosil metionina, la SAMS nativa non \`e in grado di produrne di nuova a una velocit\`a sufficientemente elevata da far riprendere la reazione di sintesi della creatina. 
	
\clearpage
\begin{sidewaysfigure}
	\center
	\resizebox{\textwidth}{!}{
		\begin{tikzpicture}[spy using outlines={rectangle, magnification=3.0}, connect spies]
		\begin{axis}[axis lines=middle, xmin=0, xmax=300, ymin=0, ymax=85,samples=1000, xtick={0,60,...,300}, xlabel={$t (min)$}, ylabel={$[Gaa_{int}] (\mu M)$}, legend style={font=\tiny},
		x label style={at={(axis description cs:0.5,-0.1)},anchor=north},
		y label style={at={(axis description cs:-0.1,.5)},rotate=90,anchor=south},
		tick label style={font=\tiny},
		label style={font=\tiny},
		legend pos=outer north east, mark repeat={10}]
		
		\addplot[
		ultra thin,
		scatter,
		point meta=explicit symbolic,
		scatter/classes={
			sam10={mark=asterisk,draw=black,mark size=2},
			sam50={mark=pentagon, draw=black,mark size=2},
			unload={mark=diamond, draw=black,mark size=2}}
		]
		file{../Dati/005-007/005-gaa.txt};
		
		
		
		\addplot[
		ultra thin,
		scatter,
		point meta=explicit symbolic,
		scatter/classes={
			sam10={mark=asterisk,draw=black,mark size=2},
			sam50={mark=pentagon, draw=black,mark size=2},
			unload={mark=diamond, draw=black,mark size=2}}
		]
		file{../Dati/005-007/006-gaa.txt};
		
		\addplot[
		ultra thin,
		scatter,
		point meta=explicit symbolic,
		scatter/classes={
			sam10={mark=asterisk,draw=black,mark size=2},
			sam50={mark=pentagon, draw=black,mark size=2},
			unload={mark=diamond, draw=black,mark size=2}}
		]
		file{../Dati/005-007/007-gaa.txt};
		
		
		\legend{$10 \mu M$ SAM, $50 \mu M$ SAM, Unloaded};
		
		\coordinate (spypoint) at (axis cs:10,65);
		\coordinate (spyviewer) at (axis cs:100,20);
		\end{axis}
		\spy [height=2.5cm, width=3cm,spy connection path={
			\begin{scope}[on background layer]
			\draw (tikzspyonnode.north east) -- (tikzspyinnode.north east);
			\draw (tikzspyonnode.north west) -- (tikzspyinnode.north west);
			\draw (tikzspyonnode.south west) -- (tikzspyinnode.south west);
			\draw (tikzspyonnode.south east) -- (tikzspyinnode.south east);
			\end{scope}
		}] on (spypoint)
		in node [fill=white] at (spyviewer);
		\end{tikzpicture}
	}
	\caption{Validazione dell'esperimento di sintesi di creatina: andamento del guanidinoacetato intracellulare}
	\label{fig:syntgaa}
\end{sidewaysfigure}

\begin{sidewaysfigure}
	\center
	\resizebox{\textwidth}{!}{
		\begin{tikzpicture}
		\begin{axis}[axis lines=middle, xmin=0, xmax=120, ymin=0, ymax=60,samples=1000, xtick={0,20,...,120}, xlabel={$t (min)$}, ylabel={$[Cr] (\mu M)$}, legend style={font=\tiny},
		x label style={at={(axis description cs:0.5,-0.1)},anchor=north},
		y label style={at={(axis description cs:-0.1,.5)},rotate=90,anchor=south},
		tick label style={font=\tiny},
		label style={font=\tiny},
		legend pos=outer north east, mark repeat={3}]
		
		\addplot[
		ultra thin,
		scatter,
		point meta=explicit symbolic,
		scatter/classes={
			sam10={mark=asterisk,draw=black,mark size=2},
			sam50={mark=pentagon, draw=black,mark size=2},
			unload={mark=diamond, draw=black,mark size=2}}
		]
		file{../Dati/005-007/005-cr.txt};
		
		
		
		\addplot[
		ultra thin,
		scatter,
		point meta=explicit symbolic,
		scatter/classes={
			sam10={mark=asterisk,draw=black,mark size=2},
			sam50={mark=pentagon, draw=black,mark size=2},
			unload={mark=diamond, draw=black,mark size=2}}
		]
		file{../Dati/005-007/006-cr.txt};
		
		\addplot[
		ultra thin,
		scatter,
		point meta=explicit symbolic,
		scatter/classes={
			sam10={mark=asterisk,draw=black,mark size=2},
			sam50={mark=pentagon, draw=black,mark size=2},
			unload={mark=diamond, draw=black,mark size=2}}
		]
		file{../Dati/005-007/007-cr.txt};
		
		
		\legend{$10 \mu M$ SAM, $50 \mu M$ SAM, Unloaded};
		
		\end{axis}
		\end{tikzpicture}
	}
	\caption{Validazione dell'esperimento di sintesi di creatina: andamento della creatina}
	\label{fig:syntcr}
\end{sidewaysfigure}
\clearpage

\section{Simulazione di eritrociti ingegnerizzati con GAMT e SAMS}\label{sez:simpred}
	I modelli \ref{mod:8}-\ref{mod:14} simulano esperimenti non ancora effettuati che prevedono l'incapsulamento di una quantit\`a fissa di GAMT e di una quantit\`a variabile di SAMS clonata da \emph{E.\ coli}.
	Ogni modello \`e stato simulato per 10 ore, con un campionamento ogni 10 minuti.
	Le figure \ref{fig:008}-\ref{fig:014} mostrano l'andamento di tutte le specie per ogni simulazione, mentre le figure \ref{fig:crsynt} e \ref{fig:gaacons} confrontano l'andamento, rispettivamente, della sintesi di creatina e del consumo di guanidinoacetato tra tutte le simulazioni, ad eccezione del modello \ref{mod:14} che, andando in saturazione, presenta dati non realistici.
	
	Si pu\`o osservare che il cambiamento di tendenza sulle curve del guanidinoacetato, appena visibile in figura \ref{fig:008} (e coincidente con l'esaurimento dell'S-adenosil metionina), viene mascherato nelle altre figure dall'attivit\`a della SAMS.
	L'S-adenosil metionina prodotta viene immediatamente sequestrata dalla GAMT, quindi la sintesi di creatina e il consumo di guanidinoacetato sono in grado di proseguire, nonostante i livelli di S-adenosil metionina siano mantenuti apparentemente a zero.
	
	L'attivit\`a della GAMT risulta essere proporzionale all'attivit\`a della SAMS incapsulata in un intervallo molto ampio.
	Nelle figure \ref{fig:013} e \ref{fig:014} si osserva invece una cinetica della SAMS che supera quella della GAMT, causando un accumulo di S-adenosil metionina, fino a raggiungere la saturazione.
	
	Le figure \ref{fig:crsynt} e \ref{fig:gaacons} evidenziano l'andamento proporzionale della cinetica della GAMT rispetto alla quantit\`a di SAMS incapsulata (e quindi della sua cinetica).

\begin{figure}[H]
	\center
	\resizebox{\textwidth}{!}{
		\begin{tikzpicture}
		\begin{axis}[axis lines=middle, xmin=0, xmax=240, ymin=0, ymax=55,samples=1000, xtick={0,60,...,240}, xlabel={$t (min)$}, ylabel={$[Specie] (\mu M)$}, legend style={font=\tiny},
		x label style={at={(axis description cs:0.5,-0.1)},anchor=north},
		y label style={at={(axis description cs:-0.1,.5)},rotate=90,anchor=south},
		tick label style={font=\tiny},
		label style={font=\tiny},
		legend pos=outer north east]

		
		\addplot[
		ultra thin,
		scatter,
		mark=asterisk,mark size=2,black,scatter/use mapped color={draw=black,fill=none}
			]
			file{../Dati/008-014/008-sam.txt};
			\addplot[
			ultra thin,
			scatter,
				mark=pentagon,mark size=2, black,scatter/use mapped color=
				{draw=black,fill=red}
				]
				file{../Dati/008-014/008-sah.txt};

				
				\addplot[
				ultra thin,
				scatter,
					mark=diamond,mark size=2,black,scatter/use mapped color={draw=black,fill=none}
					]
					file{../Dati/008-014/008-gaa_int.txt};
		
				\addplot[
				ultra thin,
				scatter, mark=o,mark size=2,black,scatter/use mapped color={draw=black,fill=none}
					]
					file{../Dati/008-014/008-gaa_ext.txt};
		
				\addplot[
				ultra thin,
				scatter, mark=x,mark size=2,black,scatter/use mapped color={draw=black,fill=none}
					]
					file{../Dati/008-014/008-cr.txt};
		
		\legend{$SAM$, $SAH$, $GAA_{int}$, $GAA_{ext}$, $CR$};
		
		\end{axis}
		\end{tikzpicture}
	}
	\caption{Simulazione di controllo con SAMS nativa (modello \ref{mod:8})}
	\label{fig:008}
\end{figure}

\begin{figure}[H]
	\center
	\resizebox{\textwidth}{!}{
		\begin{tikzpicture}
		\begin{axis}[axis lines=middle, xmin=0, xmax=240, ymin=0, ymax=55,samples=1000, xtick={0,60,...,240}, xlabel={$t (min)$}, ylabel={$[Specie] (\mu M)$}, legend style={font=\tiny},
		x label style={at={(axis description cs:0.5,-0.1)},anchor=north},
		y label style={at={(axis description cs:-0.1,.5)},rotate=90,anchor=south},
		tick label style={font=\tiny},
		label style={font=\tiny},
		legend pos=outer north east]
		
		
		\addplot[
		ultra thin,
		scatter,
		mark=asterisk,mark size=2,black,scatter/use mapped color={draw=black,fill=none}
		]
		file{../Dati/008-014/009-sam.txt};
		\addplot[
		ultra thin,
		scatter,
		mark=pentagon,mark size=2, black,scatter/use mapped color={draw=black,fill=none}
		]
		file{../Dati/008-014/009-sah.txt};
		
		\addplot[
		ultra thin,
		scatter,
		mark=diamond,mark size=2,black,scatter/use mapped color={draw=black,fill=none}
		]
		file{../Dati/008-014/009-gaa_int.txt};
		
		\addplot[
		ultra thin,
		scatter, mark=o,mark size=2,black,scatter/use mapped color={draw=black,fill=none}
		]
		file{../Dati/008-014/009-gaa_ext.txt};
		
		\addplot[
		ultra thin,
		scatter, mark=x,mark size=2,black,scatter/use mapped color={draw=black,fill=none}
		]
		file{../Dati/008-014/009-cr.txt};
		
		\legend{$SAM$, $SAH$, $GAA_{int}$, $GAA_{ext}$, $CR$};
		
		\end{axis}
		\end{tikzpicture}
	}
	\caption{Simulazione dell'incapsulamento di 0.05 mg di SAMS (modello \ref{mod:9})}
	\label{fig:009}
\end{figure}

\begin{figure}[H]
	\center
	\resizebox{\textwidth}{!}{
		\begin{tikzpicture}
		\begin{axis}[axis lines=middle, xmin=0, xmax=240, ymin=0, ymax=55,samples=1000, xtick={0,60,...,240}, xlabel={$t (min)$}, ylabel={$[Specie] (\mu M)$}, legend style={font=\tiny},
		x label style={at={(axis description cs:0.5,-0.1)},anchor=north},
		y label style={at={(axis description cs:-0.1,.5)},rotate=90,anchor=south},
		tick label style={font=\tiny},
		label style={font=\tiny},
		legend pos=outer north east]
		
		
		\addplot[
		ultra thin,
		scatter,
		mark=asterisk,mark size=2,black,scatter/use mapped color={draw=black,fill=none}
		]
		file{../Dati/008-014/010-sam.txt};
		\addplot[
		ultra thin,
		scatter,
		mark=pentagon,mark size=2, black,scatter/use mapped color={draw=black,fill=none}
		]
		file{../Dati/008-014/010-sah.txt};
		
		\addplot[
		ultra thin,
		scatter,
		mark=diamond,mark size=2,black,scatter/use mapped color={draw=black,fill=none}
		]
		file{../Dati/008-014/010-gaa_int.txt};
		
		\addplot[
		ultra thin,
		scatter, mark=o,mark size=2,black,scatter/use mapped color={draw=black,fill=none}
		]
		file{../Dati/008-014/010-gaa_ext.txt};
		
		\addplot[
		ultra thin,
		scatter, mark=x,mark size=2,black,scatter/use mapped color={draw=black,fill=none}
		]
		file{../Dati/008-014/010-cr.txt};
		
		\legend{$SAM$, $SAH$, $GAA_{int}$, $GAA_{ext}$, $CR$};
		
		\end{axis}
		\end{tikzpicture}
	}
	\caption{Simulazione dell'incapsulamento di 0.1 mg di SAMS (modello \ref{mod:10})}
	\label{fig:010}
\end{figure}

\begin{figure}[H]
	\center
	\resizebox{\textwidth}{!}{
		\begin{tikzpicture}[spy using outlines={rectangle, magnification=3.0}, connect spies]
		\begin{axis}[axis lines=middle, xmin=0, xmax=600, ymin=0, ymax=55,samples=1000, xtick={0,60,...,600}, xlabel={$t (min)$}, ylabel={$[Specie] (\mu M)$}, legend style={font=\tiny},
		x label style={at={(axis description cs:0.5,-0.1)},anchor=north},
		y label style={at={(axis description cs:-0.1,.5)},rotate=90,anchor=south},
		tick label style={font=\tiny},
		label style={font=\tiny},
		legend pos=outer north east,mark repeat={3}]
		
		
		\addplot[
		ultra thin,
		scatter,
		mark=asterisk,mark size=2,black,scatter/use mapped color={draw=black,fill=none}
		]
		file{../Dati/008-014/011-sam.txt};
		\addplot[
		ultra thin,
		scatter,
		mark=pentagon,mark size=2, black,scatter/use mapped color={draw=black,fill=none}
		]
		file{../Dati/008-014/011-sah.txt};
		
		\addplot[
		ultra thin,
		scatter,
		mark=diamond,mark size=2,black,scatter/use mapped color={draw=black,fill=none}
		]
		file{../Dati/008-014/011-gaa_int.txt};
		
		\addplot[
		ultra thin,
		scatter, mark=o,mark size=2,black,scatter/use mapped color={draw=black,fill=none}
		]
		file{../Dati/008-014/011-gaa_ext.txt};
		
		\addplot[
		ultra thin,
		scatter, mark=x,mark size=2,black,scatter/use mapped color={draw=black,fill=none}
		]
		file{../Dati/008-014/011-cr.txt};
		
		\legend{$SAM$, $SAH$, $GAA_{int}$, $GAA_{ext}$, $CR$};
		
		\end{axis}
		\end{tikzpicture}
	}
	\caption{Simulazione dell'incapsulamento di 0.5 mg di SAMS (modello \ref{mod:11})}
	\label{fig:011}
\end{figure}

\begin{figure}[H]
	\center
	\resizebox{\textwidth}{!}{
		\begin{tikzpicture}[spy using outlines={rectangle, magnification=3.0}, connect spies]
		\begin{axis}[axis lines=middle, xmin=0, xmax=600, ymin=0, ymax=55,samples=1000, xtick={0,60,...,600}, xlabel={$t (min)$}, ylabel={$[Specie] (\mu M)$}, legend style={font=\tiny},
		x label style={at={(axis description cs:0.5,-0.1)},anchor=north},
		y label style={at={(axis description cs:-0.1,.5)},rotate=90,anchor=south},
		tick label style={font=\tiny},
		label style={font=\tiny},
		legend pos=outer north east,mark repeat={3}]
		
		
		\addplot[
		ultra thin,
		scatter,
		mark=asterisk,mark size=2,black,scatter/use mapped color={draw=black,fill=none}
		]
		file{../Dati/008-014/012-sam.txt};
		\addplot[
		ultra thin,
		scatter,
		mark=pentagon,mark size=2, black,scatter/use mapped color={draw=black,fill=none}
		]
		file{../Dati/008-014/012-sah.txt};
		
		\addplot[
		ultra thin,
		scatter,
		mark=diamond,mark size=2,black,scatter/use mapped color={draw=black,fill=none}
		]
		file{../Dati/008-014/012-gaa_int.txt};
		
		\addplot[
		ultra thin,
		scatter, mark=o,mark size=2,black,scatter/use mapped color={draw=black,fill=none}
		]
		file{../Dati/008-014/012-gaa_ext.txt};
		
		\addplot[
		ultra thin,
		scatter, mark=x,mark size=2,black,scatter/use mapped color={draw=black,fill=none}
		]
		file{../Dati/008-014/012-cr.txt};
		
		\legend{$SAM$, $SAH$, $GAA_{int}$, $GAA_{ext}$, $CR$};
		
		\end{axis}
		\end{tikzpicture}
	}
	\caption{Simulazione dell'incapsulamento di 1 mg di SAMS (modello \ref{mod:12})}
	\label{fig:012}
\end{figure}

\begin{figure}[H]
	\center
	\resizebox{\textwidth}{!}{
		\begin{tikzpicture}[spy using outlines={rectangle, magnification=3.0}, connect spies]
		\begin{axis}[axis lines=middle, xmin=0, xmax=600, ymin=0, ymax=65,samples=1000, xtick={0,60,...,600}, xlabel={$t (min)$}, ylabel={$[Specie] (\mu M)$}, legend style={font=\tiny},
		x label style={at={(axis description cs:0.5,-0.1)},anchor=north},
		y label style={at={(axis description cs:-0.1,.5)},rotate=90,anchor=south},
		tick label style={font=\tiny},
		label style={font=\tiny},
		legend pos=outer north east,mark repeat={3}]
		
		
		\addplot[
		ultra thin,
		scatter,
		mark=asterisk,mark size=2,black,scatter/use mapped color={draw=black,fill=none}
		]
		file{../Dati/008-014/013-sam.txt};
		\addplot[
		ultra thin,
		scatter,
		mark=pentagon,mark size=2, black,scatter/use mapped color={draw=black,fill=none}
		]
		file{../Dati/008-014/013-sah.txt};
		
		\addplot[
		ultra thin,
		scatter,
		mark=diamond,mark size=2,black,scatter/use mapped color={draw=black,fill=none}
		]
		file{../Dati/008-014/013-gaa_int.txt};
		
		\addplot[
		ultra thin,
		scatter, mark=o,mark size=2,black,scatter/use mapped color={draw=black,fill=none}
		]
		file{../Dati/008-014/013-gaa_ext.txt};
		
		\addplot[
		ultra thin,
		scatter, mark=x,mark size=2,black,scatter/use mapped color={draw=black,fill=none}
		]
		file{../Dati/008-014/013-cr.txt};
		
		\legend{$SAM$, $SAH$, $GAA_{int}$, $GAA_{ext}$, $CR$};
		
				\coordinate (spypoint) at (axis cs:240,3);
				\coordinate (spyviewer) at (axis cs:650,20);
				\end{axis}
				\spy [height=2.5cm, width=4cm,spy connection path={
					\begin{scope}[on background layer]
					\draw (tikzspyonnode.north east) -- (tikzspyinnode.north east);
					\draw (tikzspyonnode.north west) -- (tikzspyinnode.north west);
					\draw (tikzspyonnode.south west) -- (tikzspyinnode.south west);
					\draw (tikzspyonnode.south east) -- (tikzspyinnode.south east);
					\end{scope}
				}] on (spypoint)
				in node [fill=white] at (spyviewer);
				\end{tikzpicture}
	}
	\caption{Simulazione dell'incapsulamento di 2.5 mg di SAMS (modello \ref{mod:13})}
	\label{fig:013}
\end{figure}

\begin{figure}[H]
	\center
	\resizebox{\textwidth}{!}{
		\begin{tikzpicture}
		\begin{axis}[axis lines=middle, xmin=0, xmax=600, ymin=0, ymax=65,samples=1000, xtick={0,60,...,600}, xlabel={$t (min)$}, ylabel={$[Specie] (\mu M)$}, legend style={font=\tiny},
		x label style={at={(axis description cs:0.5,-0.1)},anchor=north},
		y label style={at={(axis description cs:-0.1,.5)},rotate=90,anchor=south},
		tick label style={font=\tiny},
		label style={font=\tiny},
		legend pos=outer north east,mark repeat={3}]
		
		
		\addplot[
		ultra thin,
		scatter,
		mark=asterisk,mark size=2,black,scatter/use mapped color={draw=black,fill=none}
		]
		file{../Dati/008-014/014-sam.txt};
		\addplot[
		ultra thin,
		scatter,
		mark=pentagon,mark size=2, black,scatter/use mapped color={draw=black,fill=none}
		]
		file{../Dati/008-014/014-sah.txt};
		
		\addplot[
		ultra thin,
		scatter,
		mark=diamond,mark size=2,black,scatter/use mapped color={draw=black,fill=none}
		]
		file{../Dati/008-014/014-gaa_int.txt};
		
		\addplot[
		ultra thin,
		scatter, mark=o,mark size=2,black,scatter/use mapped color={draw=black,fill=none}
		]
		file{../Dati/008-014/014-gaa_ext.txt};
		
		\addplot[
		ultra thin,
		scatter, mark=x,mark size=2,black,scatter/use mapped color={draw=black,fill=none}
		]
		file{../Dati/008-014/014-cr.txt};
		
		\legend{$SAM$, $SAH$, $GAA_{int}$, $GAA_{ext}$, $CR$};
		
		\end{axis}

		\end{tikzpicture}
	}
	\caption{Simulazione dell'incapsulamento di 5 mg di SAMS (modello \ref{mod:14})}
	\label{fig:014}
\end{figure}

\begin{sidewaysfigure}
	\center
	\resizebox{\textwidth}{!}{
		\begin{tikzpicture}[spy using outlines={rectangle, magnification=3.0}, connect spies]
		\begin{axis}[axis lines=middle, xmin=0, xmax=600, ymin=0, ymax=55,samples=1000, xtick={0,60,...,600}, xlabel={$t (min)$}, ylabel={$[Cr] (\mu M)$}, legend style={font=\tiny},
		x label style={at={(axis description cs:0.5,-0.1)},anchor=north},
		y label style={at={(axis description cs:-0.1,.5)},rotate=90,anchor=south},
		tick label style={font=\tiny},
		label style={font=\tiny},
		legend pos=outer north east, mark repeat={3}]
		
		\addplot[
		ultra thin,
		scatter,
		point meta=explicit symbolic,
		scatter/classes={
			emat={mark=asterisk,draw=black,mark size=2},
			eco005={mark=pentagon, draw=black,mark size=2},
			eco01={mark=diamond, draw=black,mark size=2},
			eco05={mark=o,draw=black,mark size=2},
			eco1={mark=x, draw=black,mark size=2},
			eco25={mark=square,draw=black,mark size=2},
			eco5={mark=triangle,draw=black,mark size=2}}
		]
		file{../Dati/008-014/008-cr.txt};
		
		\addplot[
		ultra thin,
		scatter,
		point meta=explicit symbolic,
		scatter/classes={
			emat={mark=asterisk,draw=black,mark size=2},
			eco005={mark=pentagon, draw=black,mark size=2},
			eco01={mark=diamond, draw=black,mark size=2},
			eco05={mark=o,draw=black,mark size=2},
			eco1={mark=x, draw=black,mark size=2},
			eco25={mark=square,draw=black,mark size=2},
			eco5={mark=triangle,draw=black,mark size=2}}
		]
		file{../Dati/008-014/009-cr.txt};
		
		
		
		\addplot[
		ultra thin,
		scatter,
		point meta=explicit symbolic,
		scatter/classes={
			emat={mark=asterisk,draw=black,mark size=2},
			eco005={mark=pentagon, draw=black,mark size=2},
			eco01={mark=diamond, draw=black,mark size=2},
			eco05={mark=o,draw=black,mark size=2},
			eco1={mark=x, draw=black,mark size=2},
			eco25={mark=square,draw=black,mark size=2},
			eco5={mark=triangle,draw=black,mark size=2}}
		]
		file{../Dati/008-014/010-cr.txt};
		
		\addplot[
		ultra thin,
		scatter,
		point meta=explicit symbolic,
		scatter/classes={
			emat={mark=asterisk,draw=black,mark size=2},
			eco005={mark=pentagon, draw=black,mark size=2},
			eco01={mark=diamond, draw=black,mark size=2},
			eco05={mark=o,draw=black,mark size=2},
			eco1={mark=x, draw=black,mark size=2},
			eco25={mark=square,draw=black,mark size=2},
			eco5={mark=tria del modellongle,draw=black,mark size=2}}
		]
		file{../Dati/008-014/011-cr.txt};
		
		\addplot[
		ultra thin,
		scatter,
		point meta=explicit symbolic,
		scatter/classes={
			emat={mark=asterisk,draw=black,mark size=2},
			eco005={mark=pentagon, draw=black,mark size=2},
			eco01={mark=diamond, draw=black,mark size=2},
			eco05={mark=o,draw=black,mark size=2},
			eco1={mark=x, draw=black,mark size=2},
			eco25={mark=square,draw=black,mark size=2},
			eco5={mark=triangle,draw=black,mark size=2}}
		]
		file{../Dati/008-014/012-cr.txt};
		
		\addplot[
		ultra thin,
		scatter,
		point meta=explicit symbolic,
		scatter/classes={
			emat={mark=asterisk,draw=black,mark size=2},
			eco005={mark=pentagon, draw=black,mark size=2},
			eco01={mark=diamond, draw=black,mark size=2},
			eco05={mark=o,draw=black,mark size=2},
			eco1={mark=x, draw=black,mark size=2},
			eco25={mark=square,draw=black,mark size=2},
			eco5={mark=triangle,draw=black,mark size=2}}
		]
		file{../Dati/008-014/013-cr.txt};
		
		
		\legend{H. sapiens, $0.05mg$, $0.1mg$, $0.5mg$, $1.0mg$, $2.5mg$};
		
		\coordinate (spypoint) at (axis cs:30,5);
		\coordinate (spyviewer) at (axis cs:120,50);
		\end{axis}
		\spy [height=1.5cm, width=2cm,spy connection path={
			\begin{scope}[on background layer]
			\draw (tikzspyonnode.north east) -- (tikzspyinnode.north east);
			\draw (tikzspyonnode.north west) -- (tikzspyinnode.north west);
			\draw (tikzspyonnode.south west) -- (tikzspyinnode.south west);
			\draw (tikzspyonnode.south east) -- (tikzspyinnode.south east);
			\end{scope}
		}] on (spypoint)
		in node [fill=white] at (spyviewer);
		\end{tikzpicture}
	}
	\caption{Sintesi di creatina negli eritrociti ingegnerizzati con quantit\`a variabili di SAMS di \emph{E.\ coli}}
	\label{fig:crsynt}
\end{sidewaysfigure}

\begin{sidewaysfigure}
	\center
	\resizebox{\textwidth}{!}{
		\begin{tikzpicture}
		\begin{axis}[axis lines=middle, xmin=0, xmax=600, ymin=0, ymax=27
		,samples=1000, xtick={0,60,...,600}, xlabel={$t (min)$}, ylabel={$[GAA_{int}] (\mu M)$}, legend style={font=\tiny},
		x label style={at={(axis description cs:0.5,-0.1)},anchor=north},
		y label style={at={(axis description cs:-0.1,.5)},rotate=90,anchor=south},
		tick label style={font=\tiny},
		label style={font=\tiny},
		legend pos=outer north east,mark repeat={3}]
		
		\addplot[
		ultra thin,
		scatter,
		point meta=explicit symbolic,
		scatter/classes={
			emat={mark=asterisk,draw=black,mark size=2},
			eco005={mark=pentagon, draw=black,mark size=2},
			eco01={mark=diamond, draw=black,mark size=2},
			eco05={mark=o,draw=black,mark size=2},
			eco1={mark=x, draw=black,mark size=2},
			eco25={mark=square,draw=black,mark size=2},
			eco5={mark=triangle,draw=black,mark size=2}}
		]
		file{../Dati/008-014/008-gaa_int.txt};
		
		\addplot[
		ultra thin,
		scatter,
		point meta=explicit symbolic,
		scatter/classes={
			emat={mark=asterisk,draw=black,mark size=2},
			eco005={mark=pentagon, draw=black,mark size=2},
			eco01={mark=diamond, draw=black,mark size=2},
			eco05={mark=o,draw=black,mark size=2},
			eco1={mark=x, draw=black,mark size=2},
			eco25={mark=square,draw=black,mark size=2},
			eco5={mark=triangle,draw=black,mark size=2}}
		]
		file{../Dati/008-014/009-gaa_int.txt};
		
		\addplot[
		ultra thin,
		scatter,
		point meta=explicit symbolic,
		scatter/classes={
			emat={mark=asterisk,draw=black,mark size=2},
			eco005={mark=pentagon, draw=black,mark size=2},
			eco01={mark=diamond, draw=black,mark size=2},
			eco05={mark=o,draw=black,mark size=2},
			eco1={mark=x, draw=black,mark size=2},
			eco25={mark=square,draw=black,mark size=2},
			eco5={mark=triangle,draw=black,mark size=2}}
		]
		file{../Dati/008-014/010-gaa_int.txt};
		
		\addplot[
		ultra thin,
		scatter,
		point meta=explicit symbolic,
		scatter/classes={
			emat={mark=asterisk,draw=black,mark size=2},
			eco005={mark=pentagon, draw=black,mark size=2},
			eco01={mark=diamond, draw=black,mark size=2},
			eco05={mark=o,draw=black,mark size=2},
			eco1={mark=x, draw=black,mark size=2},
			eco25={mark=square,draw=black,mark size=2},
			eco5={mark=triangle,draw=black,mark size=2}}
		]
		file{../Dati/008-014/011-gaa_int.txt};
		
		\addplot[
		ultra thin,
		scatter,
		point meta=explicit symbolic,
		scatter/classes={
			emat={mark=asterisk,draw=black,mark size=2},
			eco005={mark=pentagon, draw=black,mark size=2},
			eco01={mark=diamond, draw=black,mark size=2},
			eco05={mark=o,draw=black,mark size=2},
			eco1={mark=x, draw=black,mark size=2},
			eco25={mark=square,draw=black,mark size=2},
			eco5={mark=triangle,draw=black,mark size=2}}
		]
		file{../Dati/008-014/012-gaa_int.txt};
		
		\addplot[
		ultra thin,
		scatter,
		point meta=explicit symbolic,
		scatter/classes={
			emat={mark=asterisk,draw=black,mark size=2},
			eco005={mark=pentagon, draw=black,mark size=2},
			eco01={mark=diamond, draw=black,mark size=2},
			eco05={mark=o,draw=black,mark size=2},
			eco1={mark=x, draw=black,mark size=2},
			eco25={mark=square,draw=black,mark size=2},
			eco5={mark=triangle,draw=black,mark size=2}}
		]
		file{../Dati/008-014/013-gaa_int.txt};

		\legend{H. sapiens, $0.05mg$, $0.1mg$, $0.5mg$, $1.0mg$, $2.5mg$};
		
		\end{axis}
		\end{tikzpicture}
	}
	\caption{Consumo di guanidinoacetato negli eritrociti ingegnerizzati con quantit\`a variabili di SAMS di \emph{E.\ coli}}
	\label{fig:gaacons}
\end{sidewaysfigure}
	\chapter{Model checking del trattamento della deficienza GAMT}\label{cap:modelchecking}
In questo capitolo vengono descritte le difficolt\`a tecniche riscontrate durante il model checking (sez. \ref{sez:limiti}), le propriet\`a CSL verificate sui modelli PRISM e l'esito del model checking per ognuna di esse (sez. \ref{sez:propepa} per le propriet\`a generate da Bio-PEPA e sez. \ref{sez:propcustom} per le propriet\`a definite manualmente).

\section{Limiti e requisiti del model checking}\label{sez:limiti}
Poich\'e la sperimentazione del trattamento per la deficienza GAMT \`e ancora in fase iniziale, le simulazioni si sono rivelate sufficienti a riprodurre gli esperimenti effettuati.
Sebbene possa essere interessante confrontare gli esiti simulativi con quelli di verifica formale, l'elevata dimensione del modello rende impraticabile tale confronto.

In termini predittivi, risulta di particolare interesse un'analisi di tipo prestazionale, che identifichi le reazioni limitanti della via metabolica, e di tipo temporale, per valutare l'efficacia della terapia.
Oltre a tali propriet\`a, pu\`o essere interessante, a scopo didattico, verificare le propriet\`a predefinite da Bio-PEPA, le quali pur essendo di tipo quantitativo, tuttavia non forniscono informazioni aggiuntive rispetto alle simulazioni.

I modelli PRISM, generati a partire dal modello Bio-PEPA, vengono convertiti in CTMC di dimensioni intrattabili per un personal computer.
Si \`e pertanto tentato di costruire le CTMC in macchine via via pi\`u ricche di memoria principale:
\begin{itemize}
	\item macchina virtuale con \SI{4}{GB} di RAM,
	\item macchina fisica con \SI{16}{GB} di RAM,
	\item server virtuale con \SI{64}{GB} di RAM,
	\item server virtuale con \SI{128}{GB} di RAM.
\end{itemize}
Utilizzando l'engine \emph{hybrid} (model checking ibrido) sul modello \ref{mod:14}, le prime tre macchine non sono riuscite a portare a termine la costruzione del modello, andando in crash per memoria insufficiente.
La quarta \`e invece riuscita a costruire un MTBDD da \SI{94}{GB} in 7325.603 secondi, costituito da 38814405 nodi (di cui 409378 raggiungibili), equivalente a una CTMC con 42156649 stati e 125726277 transizioni, ma si \`e dovuto interrompere il calcolo dopo circa quindici giorni di blocco.

Un tentativo con l'engine \emph{explicit} (model checking tradizionale) si \`e rivelato pi\`u fruttuoso, costruendo la matrice delle transizioni da \SI{71}{GB} in 699.519 secondi.
Tuttavia non \`e attualmente supportata la verifica di propriet\`a allo steady state (contenenti gli operatori $\mathbb{S}$ o $\mathbb{R}[\mathcal{S}]$) o contenenti operatori innestati (ad esempio $\mathcal{F} \mathcal{G} \phi$) per l'engine explicit.

\section{Propriet\`a generate da Bio-PEPA}\label{sez:propepa}
Relativamente al trattamento per la deficienza GAMT, sono stati generati automaticamente da Bio-PEPA:
\begin{itemize}
	\item reward transitori relativi alle reazioni ($\_sams$, $\_gamt$ e $\_uptake$), utili per contare quante volte avvengono;
	\item reward stazionari relativi alle specie ($\_SAM$, $\_SAH$, $\_GAA\_INT$, $\_GAA\_EXT$ e $\_CR$), che permettono di quantificarle;
	\item reward quadratici (ad esempio $\_SAM\_squared = \_SAM^2$), utili per il calcolo di varianze e scarti quadratici;
	\item reward basati sui contatori definiti ($\_UM\_SAM = \_SAM / 10$, $\_UM\_SAH = \_SAH / 10$, $\_UM\_GAA\_EXT = \_GAA\_EXT / 10$, $\_UM\_GAA\_INT = \_GAA\_INT / 10$ e $\_UM\_CR = \_CR / 10$).
\end{itemize}

Sono stati inoltre generati i predicati:
\begin{itemize}
	\item specie al massimo (ad esempio $\_SAM\_at\_maximum: \_SAM = MAX$), che rendono pi\`u leggibili le propriet\`a relative alla saturazione della specie;
	\item specie esaurite (ad esempio $\_SAM\_depleted: \_SAM = 0$), che rendono pi\`u leggibili le propriet\`a relative all'esaurimento.
\end{itemize}

La variabile $MAX$ \`e definita, nei modelli PRISM, come la somma di tutte le quantit\`a iniziali, per evitare di generare modelli infiniti.
Vengono generate altre tre variabili, nel file di propriet\`a, da inizializzare a valori arbitrari prima della verifica formale:
\begin{itemize}
	\item $T$ istante temporale;
	\item $i$ quantit\`a di una specie;
	\item $rew$ reward generico.
\end{itemize}
Vista la ridotta quantit\`a di informazioni aggiuntive ottenibili, rispetto alle simulazioni, dalle propriet\`a predefinite sono state inizializzate le variabili allo stesso valore per ogni propriet\`a: $T = 600$ (10 ore), $i = 10$ (\SI{1}{\mu M}), $rew$ non utilizzata.

Di seguito vengono elencate le propriet\`a, il tempo di verifica e l'esito, relativamente al solo modello \ref{mod:14}, che ha prodotto una CTMC da $42156649$ stati, $125726277$ transizioni e fattore di uniformizzazione pari a $17.4583$; il coefficiente d'errore per l'algoritmo di Fox-Glynn \`e pari a $1.25 \cdot 10^{-7}$.
I risultati vengono troncati alla quinta cifra decimale.

\begin{equation}
	\mathbb{P}_{=?} [ \mathcal{F}^{[T,T]} \_SAM=i ]
\end{equation}
``Qual \`e la probabilit\`a che ci siano $i$ molecole di S-adenosil metionina al tempo $T$?''\\
Risultato: 0\\
Tempo di elaborazione: 84309.627 secondi.

\begin{equation}
	\mathbb{P}_{=?} [ \mathcal{F}^{[T,T]} \_SAM\_at\_maximum ]
\end{equation}
``Qual \`e la probabilit\`a che l'S-adenosil metionina sia satura esattamente al tempo $T$?''\\
Risultato: 0.92757\\
Tempo di elaborazione: 57043.331 secondi.

\begin{equation}
	\mathbb{P}_{=?} [ \mathcal{F}^{ \leq T} \_SAM\_at\_maximum ]
\end{equation}
``Qual \`e la probabilit\`a che l'S-adenosil metionina sia satura entro tempo $T$?''\\
Risultato: 0.99999\\
Tempo di elaborazione: 67152.519 secondi.

\begin{equation}
	\mathbb{R}_{\{\_SAM\}=?} [ \mathcal{I}=T ]
\end{equation}
``Quanta S-adenosil metionina \`e presente al tempo $T$?''\\
Risultato: 584.92193\\
Tempo di elaborazione: 65439.354 secondi.

\begin{equation}
	\mathbb{P}_{=?} [ \mathcal{F}^{[T,T]} \_SAH=i ]
\end{equation}
``Qual \`e la probabilit\`a che ci siano $i$ molecole di S-adenosil omocisteina al tempo $T$?''\\
Risultato: 0\\
Tempo di elaborazione: 65624.93 secondi.

\begin{equation}
	\mathbb{P}_{=?} [ \mathcal{F}^{[T,T]} \_SAH\_at\_maximum ]
\end{equation}
``Qual \`e la probabilit\`a che l'S-adenosil omocisteina sia satura esattamente al tempo $T$?''\\
Risultato: 0\\
Tempo di elaborazione: 65311.243 secondi.

\begin{equation}
	\mathbb{P}_{=?} [ \mathcal{F}^{\leq T} \_SAH\_at\_maximum ]
\end{equation}
``Qual \`e la probabilit\`a che l'S-adenosil omocisteina sia satura entro tempo $T$?''\\
Risultato: 0\\
Tempo di elaborazione: 65306.103 secondi.

\begin{equation}
	\mathbb{R}_{\{\_SAH\}=?} [ \mathcal{I}=T ]
\end{equation}
``Quanta S-adenosil omocisteina \`e presente al tempo $T$?''\\
Risultato: 345.72163\\
Tempo di elaborazione: 65339.864.

\begin{equation}
	\mathbb{P}_{=?} [ \mathcal{F}^{[T,T]} \_GAA\_INT=i ]
\end{equation}
``Qual \`e la probabilit\`a che ci siano $i$ molecole di guanidinoacetato intracellulare al tempo $T$?''\\
Risultato: $2.69092 \cdot 10^{-13}$\\
Tempo di elaborazione: 65357.881 secondi.

\begin{equation}
	\mathbb{P}_{=?} [ \mathcal{F}^{[T,T]} \_GAA\_INT\_at\_maximum ]
\end{equation}
``Qual \`e la probabilit\`a che il guanidinoacetato intracellulare sia saturo esattamente al tempo $T$?''\\
Risultato: 0\\
Tempo di elaborazione: 65071.259 secondi.

\begin{equation}
	\mathbb{P}_{=?} [ \mathcal{F}^{\leq T} \_GAA\_INT\_at\_maximum ]
\end{equation}
``Qual \`e la probabilit\`a che il guanidinoacetato intracellulare sia saturo entro il tempo $T$?''\\
Risultato: 0\\
Tempo di elaborazione: 65088.718 secondi.

\begin{equation}
	\mathbb{R}_{\{\_GAA\_INT\}=?} [ \mathcal{I}=T ]
\end{equation}
``Quanto guanidinoacetato intracellulare \`e presente al tempo $T$?''\\
Risultato: 96.15829\\
Tempo di elaborazione: 65317.685 secondi.

\begin{equation}
	\mathbb{P}_{=?} [ \mathcal{F}^{[T,T]} \_GAA\_EXT=i ]
\end{equation}
``Qual \`e la probabilit\`a che ci siano $i$ molecole di guanidinoacetato plasmatico al tempo $T$?''\\
Risultato: $4.63631 \cdot 10^{-17}$\\
Tempo di elaborazione: 66352.333 secondi.

\begin{equation}
	\mathbb{P}_{=?} [ \mathcal{F}^{[T,T]} \_GAA\_EXT\_at\_maximum ]
\end{equation}
``Qual \`e la probabilit\`a che il guanidinoacetato plasmatico sia saturo esattamente al tempo $T$?''\\
Risultato: 0\\
Tempo di elaborazione: 65089.469 secondi.

\begin{equation}
	\mathbb{P}_{=?} [ \mathcal{F}^{\leq T} \_GAA\_EXT\_at\_maximum ]
\end{equation}
``Qual \`e la probabilit\`a che il guanidinoacetato plasmatico sia saturo entro tempo $T$?''\\
Risultato: 0\\
Tempo di elaborazione: 65337.322 secondi.

\begin{equation}
	\mathbb{R}_{\{\_GAA\_EXT\}=?} [ \mathcal{I}=T ]
\end{equation}
``Quanto guanidinoacetato plasmatico \`e presente al tempo $T$?''\\
Risultato: 108.12003\\
Tempo di elaborazione: 66145.619 secondi.

\begin{equation}
	\mathbb{P}_{=?} [ \mathcal{F}^{[T,T]} \_CR=i ]
\end{equation}
``Qual \`e la probabilit\`a che ci siano $i$ molecole di creatina al tempo $T$?''\\
Risultato: $4.09024 \cdot 10^{-161}$\\
Tempo di elaborazione: 65139.591 secondi.

\begin{equation}
	\mathbb{P}_{=?} [ \mathcal{F}^{[T,T]} \_CR\_at\_maximum ]
\end{equation}
``Qual \`e la probabilit\`a che la creatina sia satura esattamente al tempo $T$?''\\
Risultato: 0\\
Tempo di elaborazione: 65127.878 secondi.

\begin{equation}
	\mathbb{P}_{=?} [ \mathcal{F}^{\leq T} \_CR\_at\_maximum ]
\end{equation}
``Qual \`e la probabilit\`a che la creatina sia satura entro il tempo $T$?''\\
Risultato: 0\\
Tempo di elaborazione: 65093.444 secondi.

\begin{equation}
	\mathbb{R}_{\{\_CR\}=?} [ \mathcal{I}=T ]
\end{equation}
``Quanta creatina \`e presente al tempo $T$?''\\
Risultato: 332.72163\\
Tempo di elaborazione: 65481.934 secondi.

\begin{equation}
	\mathbb{R}_{\{\_UM\_SAM\}=?} [ \mathcal{I}=T ]
\end{equation}
``Quanta S-adenosil metionina \`e presente al tempo $T$, in $\mu M$?''\\
Risultato: 58.49219\\
Tempo di elaborazione: 65468.829 secondi.

\begin{equation}
	\mathbb{R}_{\{\_UM\_SAH\}=?} [ \mathcal{I}=T ]
\end{equation}
``Quanta S-adenosil omocisteina \`e presente al tempo $T$, in $\mu M$?''\\
Risultato: 34.57216\\
Tempo di elaborazione: 66046.642 secondi.

\begin{equation}
	\mathbb{R}_{\{\_UM\_GAA\_INT\}=?} [ \mathcal{I}=T ]
\end{equation}
``Quanto guanidinoacetato intracellulare \`e presente al tempo $T$, in $\mu M$?''\\
Risultato: 9.61583\\
Tempo di elaborazione: 65381.08 secondi.

\begin{equation}
	\mathbb{R}_{\{\_UM\_GAA\_EXT\}=?} [ \mathcal{I}=T ]
\end{equation}
``Quanto guanidinoacetato plasmatico \`e presente al tempo $T$, in $\mu M$?''\\
Risultato: 10.812000\\
Tempo di elaborazione: 65364.411 secondi.

\begin{equation}
	\mathbb{R}_{\{\_UM\_CR\}=?} [ \mathcal{I}=T ]
\end{equation}
``Quanta creatina \`e presente al tempo $T$, in $\mu M$?''\\
Risultato: 33.27216\\
Tempo di elaborazione: 65343.736 secondi.

\begin{equation}
	\mathbb{R}_{\{\_sams\}=?} [ \mathcal{C} \leq T ]
\end{equation}
``Quante volte \`e avvenuta la reazione ad opera della SAM sintasi entro il tempo $T$?''\\
Risultato: N/A\\
Tempo di elaborazione: N/A\\
Risultato ottenuto per via simulativa: 882.566.

\begin{equation}
	\mathbb{R}_{\{\_gamt\}=?} [ \mathcal{C} \leq T ]
\end{equation}
``Quante volte \`e avvenuta la reazione ad opera della guanidinoacetato metiltransferasi entro il tempo $T$?''\\
Risultato: N/A\\
Tempo di elaborazione: N/A\\
Risultato ottenuto per via simulativa: 332.971.

\begin{equation}
	\mathbb{R}_{\{\_uptake\}=?} [ \mathcal{C} \leq T ]
\end{equation}
``Quante volte il guanidinoacetato \`e entrato dentro gli eritrociti entro il tempo $T$?''\\
Risultato: N/A\\
Tempo di elaborazione: N/A\\
Risultato ottenuto per via simulativa: 392.443.

\section{Propriet\`a definite}\label{sez:propcustom}
Di seguito sono descritte propriet\`a pi\`u interessanti di quelle generate automaticamente (anche in questo caso verificate per $T = 600$ e $soglia = 10$).

Essendo l'S-adenosil metionina il reagente limitante, \`e necessario controllare e quantificare i casi in cui l'intera via metabolica vada in \emph{starvation}, bloccando la produzione di creatina:
\begin{equation}
\mathbb{P}_{=?} [\mathcal{F} \mathcal{G} \_SAM = 0]
\end{equation}
``Qual \`e la probabilit\`a che prima o poi l'S-adenosil metionina si esaurisca e rimanga a zero per sempre?''\\
Risultato: 0\\
Tempo di elaborazione: 342.842 secondi.

Anche nei casi in cui l'S-adenosil metionina non si esaurisca, \`e importante quantificare pure eventuali cali prestazionali e capacit\`a di recupero da cali temporanei:
\begin{equation}
\mathbb{P}_{=?} [\mathcal{F} \mathcal{G} \_SAM \leq soglia]
\end{equation}
``Qual \`e la probabilit\`a che prima o poi l'S-adenosil metionina scenda sotto la $soglia$ e vi rimanga per sempre?''\\
Risultato: 0\\
Tempo di elaborazione: 292.23 secondi.
\begin{equation}
\mathbb{P}_{=?} [\mathcal{G} (\_SAM = 0 \rightarrow \mathcal{F}^{\leq T} \_SAM > soglia)]
\end{equation}
``Qual \`e la probabilit\`a che, se si esaurisce l'S-adenosil metionina, questa riesca sempre a tornare oltre la $soglia$ entro un tempo $T$?''\\
Risultato: Propriet\`a non verificabile con l'engine explicit, n\'e per via simulativa\\
Tempo di elaborazione: N/A.

Vista la citotossicit\`a del guanidinoacetato, \`e interessante quantificare eventuali accumuli negli eritrociti (\`e importante tuttavia notare che il modello non rappresenta una possibile fuoriuscita del guanidinoacetato e che quindi l'esito della propriet\`a seguente potrebbe non essere significativo):
\begin{equation}
\mathbb{P}_{=?} [\mathcal{F}\mathcal{G}^{\leq T} \_GAA\_INT > soglia]
\end{equation}
``Qual \`e la probabilit\`a che il guanidinoacetato intracellulare rimanga sopra la $soglia$ almeno per un tempo $T$?''\\
Risultato: Propriet\`a non verificabile con l'engine explicit, n\'e per via simulativa\\
Tempo di elaborazione: N/A.

Per valutare l'efficacia della cura, \`e utile predire il tempo necessario a ripristinare livelli fisiologici di guanidinoacetato nel plasma:
\begin{equation}
\mathbb{P}_{=?} [\mathcal{F}^{\leq T} \_GAA\_EXT \leq soglia]
\end{equation}
``Qual \`e la probabilit\`a che il guanidinoacetato plasmatico scenda sotto la $soglia$ entro un tempo $T$?''\\
Risultato: $5.06063 \cdot 10^{-17}$\\
Tempo di elaborazione: 64474.518 secondi.
\\\\
	\chapter{Conclusioni}\label{cap:conclusioni}
	\section{Lavoro svolto}
	In questa tesi \`e stato descritto un approccio formale per l'analisi \emph{in silico} di reti metaboliche.
	Sono stati presentati strumenti matematici in grado di descrivere fedelmente con un modello ogni aspetto di un metabolismo generico, con particolare enfasi su catene di Markov a tempo continuo, algebre dei processi e logiche temporali, e le tecniche per utilizzare tali modelli tramite simulazioni e verifiche formali.
	
	\`E stata proposta un'applicazione di tale approccio allo studio della deficienza GAMT e di una cura in fase sperimentale, utilizzando gli strumenti software Bio-PEPA e PRISM.
	La cura presa in esame utilizza eritrociti ingegnerizzati da un apparato noto come Red Cell Loader per incapsulare l'enzima guanidinoacetato metiltransferasi allo scopo di sopperire alla mancanza della GAMT epatica dei pazienti affetti.
	Essendo la sperimentazione \emph{in vitro} non ancora completata, l'analisi \emph{in silico} proposta in questa tesi si \`e rivelata utile a indirizzare gli esperimenti futuri.
	
	Pi\`u in dettaglio, l'analisi simulativa effettuata ha dimostrato la necessit\`a di incapsulare, oltre alla GAMT, anche S-adenosil metionina sintasi clonata da \emph{E.\ coli}, al fine di sopperire alla scarsa affinit\`a della SAMS ematica e che porterebbe a un limitato effetto terapeutico, a causa della poca S-adenosil metionina a disposizione all'interno degli eritrociti.
	La verifica formale ha dimostrato che, una volta incapsulate sia SAMS che GAMT, non c'\`e rischio che il bioreattore costituito dagli eritrociti smetta di funzionare entro poche ore dall'infusione per esaurimento di S-adenosil metionina.
	
	\section{Necessit\`a di un approccio formale integrato}
		L'approccio simulativo:
		\begin{itemize}
			\item [$+$] \`e ampiamente diffuso grazie alla sua semplicit\`a di utilizzo;
			\item [$+$] \`e computazionalmente semplice, quindi produce risultati in tempi rapidi;
			\item [$+$] richiede un numero di risorse limitato;
			\item [$-$] non \`e in grado di catturare tutti i comportamenti del sistema in maniera esaustiva;
			\item [$-$] pu\`o accumulare errori in simulazioni particolarmente lunghe;
			\item [$-$] richiede una media di pi\`u simulazioni per aumentare la precisione;
			\item [$-$] \`e sensibile al tempo da simulare, aumentando le risorse richieste proporzionalmente al tempo;
			\item [$-$] rende difficile l'analisi di casi particolari, che potrebbero addirittura passare inosservati se non sono noti a priori.
		\end{itemize}
		
		L'approccio basato su verifica:
		\begin{itemize}
			\item [$+$] analizza esaustivamente tutti i comportamenti del sistema, rilevando anche casi particolari non noti a priori;
			\item [$+$] \`e esente da errori di approssimazione;
			\item [$+$] \`e computazionalmente indipendente dal limite temporale imposto per la soddisfacibilit\`a delle propriet\`a, richiedendo le stesse risorse per una propriet\`a istantanea o che dovr\`a verificarsi dopo miliardi di operazioni;
			\item [$-$] \`e poco diffuso a causa della sua difficolt\`a di utilizzo;
			\item [$-$] \`e computazionalmente complesso, richiedendo tempi elevati per ottenere un esito;
			\item [$-$] \`e avido di risorse, a meno dell'utilizzo di tecniche simboliche, non sempre efficaci;
		\end{itemize}
		
		Risulta quindi utile (se non auspicabile) utilizzare un approccio che integri sia la componente simulativa che quella basata su verifica.
		I formalismi di modellazione si prestano agevolmente ad entrambi gli approcci e la visione d'insieme fornita \`e spesso in grado di rivelare caratteristiche non osservabili con uno solo dei due approcci.
		
	\section{Sviluppi futuri}
	Relativamente al caso di studio presentato, si prospetta la necessit\`a di verificare la fedelt\`a delle simulazioni con gli esperimenti che verranno effettuati in futuro.
	Inoltre non \`e esclusa la possibilit\`a di investigare in maniera formale nuovi aspetti che verranno portati alla luce dai nuovi esperimenti.
	
	Per poter verificare le propriet\`a non modellabili dall'engine explicit di PRISM, pu\`o essere interessante introdurre approssimazioni che rendano i modelli di dimensione sufficientemente bassa da permettere la verifica tramite engine hybrid.
	In alternativa, quando l'engine explicit sar\`a in uno stato pi\`u maturo, potr\`a rivelarsi utile ripetere la verifica sui modelli non approssimati.
	
	\appendix
	
	\printbibliography % biblatex

	
	\ringraziamenti
		Si ringraziano i professori Jane Hillston e Stephen Gilmore per il loro aiuto sulla sintassi di Bio-PEPA.
		
		Si ringraziano i dott.ri Marco Cappellacci, Paolo Cecchini e Andrea Seraghiti e la Sezione di Scienze e Tecnologiedell'Informazione dell'Universit\`a di Urbino per aver messo a disposizione una macchina sufficientemente potente per l'applicazione delle tecniche descritte nel capitolo \ref{cap:modelchecking}.
		Si ringrazia infine la prof.ssa Simona Scacchi per la correzione delle bozze.
		
\end{document}